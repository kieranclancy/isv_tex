\bookheader{Deuteronomy}
\labelbook{Deut}

\bookpretitle{The Fifth Book of the Law called}
\booktitle{Deuteronomy}

\labelchapt{1}
\passage{The Setting of the Covenant}

\chapt{1}
\v{1}These are the words that Moses spoke to the assembly of\fnote{\fbackref{1:1} Lit. \fbib{to all}} Israel east\fnote{\fbackref{1:1} Lit. \fbib{Israel on the other side}; and so throughout the book} of the Jordan River,\fnote{\fbackref{1:1} The Heb. lacks \fbib{River}; and so throughout the book} in the Arabah desert, opposite Suph between Paran, Tophel, Laban, Hazeroth, and Di-zahab. \v{2}It takes eleven days to travel\fnote{\fbackref{1:2} The Heb. lacks \fbib{to travel}} from Horeb to Kadesh-barnea via Mount Seir.\fnote{\fbackref{1:2} This mountain, the modern \fbib{Jebel esh-sher\'{a}}, is located in the mountain range that extends south of the Dead Sea toward the Gulf of Aqaba, and is bordered by the Arabah Valley to the west.} \v{3}On the first day of the eleventh month,\fnote{\fbackref{1:3} I.e. the month of Shebat in the Hebrew calendar} in the fortieth year, Moses spoke to the Israelis about everything that the \divine{Lord} had commanded him concerning them. \v{4}This took place\fnote{\fbackref{1:4} The Heb. lacks \fbib{This took place}} after he defeated Sihon, king of the Amorites, who lived in Heshbon and Og, king of Bashan, who lived in Ashtaroth at Edrei.
\passage{Moses Reviews God's Instructions}

\v{5}East of the Jordan River, in the land of Moab, Moses began to expound this Law: \v{6}``The \divine{Lord} our God spoke to us in Horeb. He said, `You have been at this mountain long enough. \v{7}Break camp,\fnote{\fbackref{1:7} Lit. \fbib{Turn}} get going, and proceed to the hill country of the Amorites and all the nearby places in the Arabah desert, the highlands, the foothills, the Negev,\fnote{\fbackref{1:7} I.e. the southern regions of the Sinai peninsula; cf. Josh 10:40} the coastal plains, all of the land of the Canaanites, and Lebanon as far as the great river, the Euphrates. \v{8}Look! I've given you the land that lies ahead. Go in and possess the land that I, the \divine{Lord}, promised to give to your ancestors Abraham, Isaac, and Jacob, as well as to their descendants.'\,''
\passage{Moses Reviews the Selected Officials}

\v{9}``I also told you at that time that I won't be able to sustain you on my own. \v{10}The \divine{Lord} your God greatly multiplied your numbers, and today you are like the stars in the sky. \v{11}May the \divine{Lord}, the God of your ancestors, increase your numbers a thousand times more, and may he bless you, as he promised you. \v{12}How can I bear the burden of you and your bickering all by myself? \v{13}Choose for yourselves wise and discerning men, known to your tribes, and appoint them as your leaders. \v{14}You answered by saying that this plan is a good thing. \v{15}So I chose leaders from your tribes, wise and respected men, and I appointed them over you---commanders of thousands, hundreds, fifties, and tens. \v{16}I charged your judges at that time, `When you hold a hearing between brothers, judge fairly between a man and his brother or between foreigners. \v{17}When you hold a hearing, don't be partial\fnote{\fbackref{1:17} Lit. \fbib{don't recognize faces}} in judgment toward the least important or toward the great. Never fear men, because judgment belongs to God. If the matter is difficult for you, bring it to me for a hearing.' \v{18}I charged you at that time that you must do all of these things.''
\passage{Moses Reviews the Sending of the Scouts}
\passageinfo{(Numbers 13:1-15)}

\v{19}``Then we set out from Horeb and walked through that vast and dreadful desert, where you observed the road to the Amorite hill country. Just as the \divine{Lord} our God ordained for us, we finally arrived at Kadesh-barnea. \v{20}I told you at that time, `You have reached the hill country of the Amorites, which the \divine{Lord} our God is about to give us. \v{21}Look! The \divine{Lord} your God has given the land that lies\fnote{\fbackref{1:21} The Heb. lacks \fbib{that lies}} before you. Go and possess it, just as the \divine{Lord} God of your ancestors commanded you. Don't be afraid or discouraged.'

\v{22}``Then all of you approached me and said: `Let's send out men in advance of us so they can survey the land and bring back a report to us on how we'll go up to their cities.' \v{23}Because this suggestion\fnote{\fbackref{1:23} Lit. \fbib{word}} seemed good to me, I chose twelve men from among you, one from each tribe. \v{24}Then these men set out,\fnote{\fbackref{1:24} Lit. \fbib{Then they turned}} went up to the hill county, reached the Eshcol Valley, and surveyed it. \v{25}They hand-picked some of the fruit of the land, brought it down to us, and gave a report that said, `The land which the \divine{Lord} is about to give us is good.'\,''
\passage{Israel Rebels}

\v{26}``However, your ancestors didn't go up. Instead, they rebelled against the command\fnote{\fbackref{1:26} Lit. \fbib{mouth}} of the \divine{Lord} your God. \v{27}You murmured in your tents, `The \divine{Lord} hates us. He brought us out of the land of Egypt in order to deliver us to\fnote{\fbackref{1:27} Lit. \fbib{to give us into the hands of}} the Amorites so he could destroy us. \v{28}Where can we go? Our brothers discouraged us when they said that the people are bigger and taller than we are. Their cities are tall and fortified to the sky, and we also saw the Anakim\fnote{\fbackref{1:28} I.e. a race of giants that formerly populated Canaan; cf. Num 13:22, 33; Deut 9:2} there.'

\v{29}``Then I told you, `Don't be terrified or afraid of them. \v{30}The \divine{Lord} your God is the One who will be going ahead of you. He'll fight for you just as he did in Egypt before your eyes. \v{31}In the desert you saw that the \divine{Lord} carried you like a man carries his son, on every road you traveled until you reached this place.' \v{32}But despite this, you didn't trust in the \divine{Lord} your God, \v{33}who walked ahead of you along the way to scout a place for you to pitch camp---by fire at night and cloud by day---to lead you on the way you should go.''
\passage{Entrance is Denied}

\v{34}``When the \divine{Lord} heard your complaints, he became angry and declared, \v{35}`I swear that not one man of this evil generation will see the good land that I promised to give to your ancestors, \v{36}except Jephunneh's son Caleb. He will see it and I will give to him and to his descendants the land on which he has walked because he wholeheartedly followed the \divine{Lord}.'

\v{37}``The \divine{Lord} was also furious with me because of you. He said: `You will not enter the land.\fnote{\fbackref{1:37} Lit. \fbib{land there}} \v{38}However, Nun's son Joshua, your assistant, will go there. Encourage him, for he will cause Israel to take possession of it. \v{39}Your little ones---whom you said would be taken captive---and your children who do not yet\fnote{\fbackref{1:39} Lit. \fbib{this day}} know right from wrong will enter the land.\fnote{\fbackref{1:39} Lit. \fbib{land there}} I will give it to them and they themselves will possess it. \v{40}But as for you, prepare to set out for the desert on the way to the Reed\fnote{\fbackref{1:40} So MT; LXX reads \fbib{Red}} Sea.'

\v{41}``You responded to me and said, `We have sinned against the \divine{Lord}. We will now go up and fight according to what the \divine{Lord} our God commanded.' So each man put on his weapon for battle and recklessly started out for the hill country.''
\passage{The Amorites Defeat Israel}

\v{42}``Then the \divine{Lord} told me: `Tell them not to go up and fight because I will not be in their midst, or else you will be defeated before your enemies.'

\v{43}``I spoke to you but you didn't listen. Instead you rebelled against the command\fnote{\fbackref{1:43} Lit. \fbib{mouth}} of the \divine{Lord} and went up to the hill country. \v{44}The Amorites who lived in the hill country came out to engage you in battle. They pursued you like bees do and crushed you from Seir to Hormah. \v{45}You returned and cried out in the \divine{Lord}'s presence, but the \divine{Lord} didn't hear your voice or listen to you. \v{46}You remained in Kadesh for many days. It was a long time, indeed.''
\labelchapt{2}
\passage{Israel Passes through Edomite Territory}

\chapt{2}
\v{1}``We turned and set out for the desert on the road to the Reed\fnote{\fbackref{2:1} So MT; LXX reads \fbib{Red}} Sea, just as the \divine{Lord} had directed me. We traveled around Mount Seir for many days. \v{2}Then the \divine{Lord} told me, \v{3}`You've walked around this mountain long enough. Turn northward \v{4}and command this people, ``You are about to pass through the territory of your relatives, the descendants of Esau who live around Seir. They will be afraid of you so be very careful. \v{5}Don't fight them, because I won't give you any part of their land, not even the size of a footprint.\fnote{\fbackref{2:5} Lit. \fbib{the treading of the calf of your foot}} I have given Mount Seir to Esau as their property. \v{6}You may buy food to eat and water to drink from them, paying\fnote{\fbackref{2:6} The Heb. lacks \fbib{paying}} with cash.''\,' \v{7}Indeed, the \divine{Lord} your God blessed all the works of your hands. He knows about your travels through this vast desert. The \divine{Lord} your God was with you these past 40 years, so that you didn't lack anything. \v{8}So we bypassed our relatives, the descendants of Esau who live in Seir. We turned through the Arabah desert from Elath, and from Ezion-geber we traveled the desert road to Moab.''
\passage{Israel Passes through Moabite Territory}

\v{9}``Then the \divine{Lord} told me, `Don't harass Moab or provoke them to war, because I won't give you any part of their land. I have given Ar to the descendants of Lot as their property. \v{10}(The Emites, a people as powerful, numerous, and tall as the Anakim,\fnote{\fbackref{2:10} I.e. a race of giants that formerly populated Canaan; cf. Num 13:22, 33; Deut 9:2} lived there before. \v{11}Like the Anakim,\fnote{\fbackref{2:11} I.e. a race of giants that formerly populated Canaan; cf. Num 13:22, 33; Deut 9:2} they were thought of as Rephaim,\fnote{\fbackref{2:11} I.e. a race of giants that formerly populated Canaan; cf. Num 13:22, 33} but the Moabites called them Emites. \v{12}The Horites used to live in Seir before the descendants of Esau dispossessed them, exterminated them, and settled there instead, just as Israel will do in the land of its possession, which the \divine{Lord} gave them.) \v{13}Now get going and cross the Wadi\fnote{\fbackref{2:13} I.e. a seasonal stream or river that channels water during rain seasons but is dry at other times} Zered.' And so we crossed the Wadi\fnote{\fbackref{2:13} I.e. a seasonal stream or river that channels water during rain seasons but is dry at other times} Zered. \v{14}Now from the time we left Kadesh-barnea until we crossed the Wadi\fnote{\fbackref{2:14} I.e. a seasonal stream or river that channels water during rain seasons but is dry at other times} Zered was 38 years. All of that generation, the soldiers in the camp, were destroyed just as the \divine{Lord} swore they would be. \v{15}Indeed, the hand of the \divine{Lord} was against them to root them out from the camp until they were utterly destroyed.''
\passage{Israel Passes through Ammonite Territory}

\v{16}``And so all the soldiers among the people died. \v{17}Then the \divine{Lord} spoke to me, \v{18}`Today, you are about to cross the border of Moab at Ar. \v{19}When you come to the Ammonites, don't harass or provoke them to war, for I won't give any part of Ammonite land to you, since I have given it to the descendants of Lot as their property.

\v{20}```(Indeed, it was considered Rephaim\fnote{\fbackref{2:20} I.e. a race of giants that formerly populated Canaan} territory, since the Rephaim\fnote{\fbackref{2:20} I.e. a race of giants that formerly populated Canaan} used to lived there. The Ammonites called them Zamzummites, \v{21}a great people, numerous, and tall as the Anakim.\fnote{\fbackref{2:21} I.e. a race of giants that formerly populated Canaan; cf. Num 13:22, 33; Deut 9:2} But the \divine{Lord} destroyed the Rephaim,\fnote{\fbackref{2:21} I.e. a race of giants that formerly populated Canaan} so that the Ammonites dispossessed them and settled there instead. \v{22}This is what he did for the descendants of Esau who live in Seir, when he destroyed the Horites before them. So they dispossessed them and settled there in their place, where they live\fnote{\fbackref{2:22} The Heb. lacks \fbib{where they live}} to this day. \v{23}It was the same for the Avvites who lived in villages as far as Gaza. The Caphtorites, who came from Crete,\fnote{\fbackref{2:23} Lit. \fbib{Caphtor}} destroyed them and settled there in their place.) \v{24}Get ready and set out for the Wadi\fnote{\fbackref{2:24} I.e. a seasonal stream or river that channels water during rain seasons but is dry at other times} Arnon. Look! I've given into your control Sihon the Amorite, king of Heshbon, along with his land. Prepare to take possession by provoking him to war. \v{25}Starting today I will begin to instill fear and terror of you on the part of every nation under heaven who hears reports about you. They'll tremble in anguish before you.'\,''
\passage{Israel Defeats Sihon, King of Heshbon}

\v{26}``I sent messengers from the desert of Kedemoth to King Sihon of Heshbon with this message of peace: \v{27}`Let me pass through your territory. I'll stay on the main road. I won't turn to the right or left. \v{28}Sell me food for cash,\fnote{\fbackref{2:28} Lit. \fbib{silver}} so I can eat and give me water for cash,\fnote{\fbackref{2:28} Lit. \fbib{silver}} so I can drink. Just let me pass through on foot, \v{29}as the descendants of Esau who live in Seir did for me, as did the Moabites who live in Ar. I'll pass through,\fnote{\fbackref{2:29} The Heb. lacks \fbib{I'll pass through}} until I will have crossed the Jordan into the land that the \divine{Lord} our God is about to give us.' \v{30}But King Sihon of Heshbon did not allow us to pass through, because the \divine{Lord} your God had hardened his spirit and made him arrogant,\fnote{\fbackref{2:30} Lit. \fbib{and emboldened his heart}} in order to deliver him into your control today.

\v{31}``Then the \divine{Lord} told me, `See, I've begun to deliver Sihon and his territory over to you. Prepare to take possession of his land.'

\v{32}``Sihon came out to meet us, including his entire army, at the battle of Jahaz. \v{33}The \divine{Lord} our God delivered him to us, so we attacked him, his son, and his whole army. \v{34}We captured all his towns at that time. We utterly destroyed every town---the men, the women, and the children---leaving no survivors. \v{35}We only appropriated the livestock for our use, along with plunder from the cities that we captured. \v{36}From Aroer on the edge of Arnon Valley and from the town all the way to Gilead, there was no city that was too strong for us---the \divine{Lord} our God delivered them all to us. \v{37}You did not encroach onto Ammonite land, the banks of the Wadi\fnote{\fbackref{2:37} I.e. a seasonal stream or river that channels water during rain seasons but is dry at other times} Jabbok, the towns in the hill country, and all the other places that were forbidden\fnote{\fbackref{2:37} Lit. \fbib{commanded}} by the \divine{Lord} our God.''
\labelchapt{3}
\passage{Israel Defeats the King of Bashan}

\chapt{3}
\v{1}``We set out and went up along the road to Bashan. Then King Og of Bashan came out to meet us---he and his whole army---for a battle at Edrei. \v{2}Then the \divine{Lord} told me, `Don't fear him, because I've delivered him, his army, and his territory into your control. Do to him just as you have done to Sihon, king of the Amorites, who lived in Heshbon.'

\v{3}``So the \divine{Lord} our God also delivered into our control King Og of Bashan, along with his whole army. We attacked him until there were no survivors.\fnote{\fbackref{:3} Lit. \fbib{survivors left to him}} \v{4}Then we captured all his cities at that time. There was not a city left that we didn't capture from them---60 cities in all from the region of Argob, which is part of the kingdom of Og in Bashan. \v{5}All of these cities were fortified with high walls, gates, and bars. Furthermore, there were very many unwalled regions. \v{6}We utterly destroyed them, just as we did King Sihon of Heshbon, attacking them in every city---the men, women, and children. \v{7}But we kept for ourselves all of the livestock and plunder from the towns.

\v{8}``So at that time, we took control from the two Amorite kings the territory east of the Jordan from Wadi\fnote{\fbackref{3:8} I.e. a seasonal stream or river that channels water during rain seasons but is dry at other times} Arnon to Mount Hermon. \v{9}(The Sidonians called Hermon Sirion, but the Amorites called it Senir.) \v{10}We took control of\fnote{\fbackref{3:10} The Heb. lacks \fbib{We took control of}} all the cities of the plain, all of Gilead and Bashan as far as Salecah and Edrei, cities of the kingdom of Og in Bashan. \v{11}Only King Og of Bashan remained from the remnants of the Rephaim.\fnote{\fbackref{3:11} I.e. a race of giants that formerly populated Canaan; cf. Num 13:22, 33} In fact, his bed was made of iron. It's in Rabbah of the Ammonites, isn't it? It was nine cubits\fnote{\fbackref{3:11} I.e. about thirteen and a half feet long} long and four cubits\fnote{\fbackref{3:11} I.e. about six feet} wide.''
\passage{Moses Allots Land East of the Jordan}
\passageinfo{(Numbers 32:1-15)}

\v{12}``Of the land that we captured at that time, I've given its towns to the descendants of Reuben and the descendants of Gad from Aroer near the Wadi\fnote{\fbackref{3:12} I.e. a seasonal stream or river that channels water during rain seasons but is dry at other times} Arnon to half of the hill country of Gilead. \v{13}The remainder of Gilead and Bashan of the kingdom of Og, I've given to the half-tribe of Manasseh. (The whole region of Argob---that is, all of Bashan---is called the land of the Rephaim.) \v{14}Manasseh's son Jair captured all the Argob region as far as the territory of the descendants of Geshur and the descendants of Maacath. Bashan was named after him; that's why it is called Havvoth-jair to this day. \v{15}Furthermore, I've given Gilead to Machir. \v{16}And I've given Gilead to the descendants of Reuben and the descendants of Gad as far as the Arnon Valley, designating the middle of the valley as its boundary, including up to the Jabbok River as a boundary with the Ammonites. \v{17}The Arabah and the Jordan River are also a boundary from Chinnereth\fnote{\fbackref{3:17} I.e. the Sea of Galilee} to the Sea of the Arabah (that is, the Salt Sea),\fnote{\fbackref{3:17} I.e. the Dead Sea} below the slopes of Pisgah on the east.''
\passage{Moses Instructs the Men of War}

\v{18}``Then I commanded you at that time, `The Lord your God gave you this land as a possession. Those equipped for battle---every man a warrior---will cross before your fellow Israelis. \v{19}However, your women, children, and livestock---and I know you have many---may reside in your towns that I gave you \v{20}until the \divine{Lord} grants rest to your fellow Israelis like you. When they take possession of the territory that the \divine{Lord} your God is about to give them on the other side of the Jordan River, then each of you may return to the territory that I've allotted for you.'

\v{21}``I also charged Joshua at that time, `You witnessed everything that the \divine{Lord} your God did to the two kings. Indeed, the \divine{Lord} will do this to all the kingdoms which you are about to enter. \v{22}You are not to fear them, because the \divine{Lord} your God will fight for you.'\,''
\passage{Moses Pleads with God}

\v{23}``I pleaded with the \divine{Lord} at that time, \v{24}`\divine{Lord} God, you've begun to show your greatness and your strong power to your servant. For what god in heaven or on earth can equal your works and mighty deeds? \v{25}Let me cross over that I may see the good land on the other side of the Jordan River---the good hill country---as well as Lebanon.'

\v{26}``However, the \divine{Lord} was furious with me because of you. He did not listen to me. Instead, the \divine{Lord} said, `You are not to speak to me about this matter again! \v{27}Go up to the top of Pisgah and lift your eyes toward the west, north, south, and east. Look with your own eyes, since you won't be able to cross this Jordan River. \v{28}Therefore charge Joshua to be doubly strong, because he will lead this people\fnote{\fbackref{3:28} Lit. \fbib{He will cross over before this people}.} and cause them to inherit the land that you'll see.' \v{29}We then encamped in the valley opposite Beth-peor.''
\labelchapt{4}
\passage{Moses Presents the Privileges of the Covenant}

\chapt{4}
\v{1}``Now, Israel, listen to the statutes and the ordinances that I'm teaching you to observe so you may live and go in to take possession of the land that the \divine{Lord}, the God of your ancestors, is about to give you. \v{2}Do not add or subtract a thing to what I'm commanding you. Observe the commands of the \divine{Lord} your God.\fnote{\fbackref{4:2} Lit. \fbib{God that I'm commanding you}} \v{3}You saw with your own eyes what he did in Baal Peor. The \divine{Lord} your God exterminated from among you every man who followed Baal of Peor. \v{4}But all of you who are clinging to the \divine{Lord} your God are alive today. \v{5}See! I taught you the statutes and the ordinances, just as the \divine{Lord} God commanded. Therefore, observe them\fnote{\fbackref{4:5} The Heb. lacks \fbib{them}} when you enter the land you are about to possess. \v{6}Observe them carefully, because this will show your wisdom and discernment in the eyes of people who'll listen to all these decrees. Then they'll say: `Surely this great nation is a wise and discerning people.' \v{7}For what great nation has a god so near like the \divine{Lord} our God whenever we call on him? \v{8}And what great nation has all the decrees and righteous ordinances like all this teaching that I'm giving you today? \v{9}Only guard yourselves carefully so you won't forget the things that you saw and let them slip from your mind for the rest of your life. Tell them to your children and to your grandchildren. \v{10}The day you stood in the presence of the \divine{Lord} your God in Horeb, the \divine{Lord} told me, `Gather the people before me so they may hear my words, learn to revere me the whole time that they live in the land, and teach them\fnote{\fbackref{4:10} The Heb. lacks \fbib{them}} to their children.'\,''
\passage{Moses Warns against Idolatry}

\v{11}``When you approached and stood at the foot of the mountain---a mountain that was blazing with fire at its core\fnote{\fbackref{4:11} Lit. \fbib{heart}} while the sky was covered with thick, dark clouds--- \v{12}the \divine{Lord} your God spoke from the midst of the fire. You heard the sound of words, but you saw no form; there was only a voice. \v{13}He declared to you his covenant, which he commanded you to observe---the Ten Commandments that he wrote on two stone tablets. \v{14}The \divine{Lord} commanded me at that time to teach you to observe the statutes and ordinances in the land after you cross over to take possession of it.

\v{15}``Therefore, for your own sake, be very careful, since you did not see any form on the day that the \divine{Lord} your God spoke to you in Horeb from the midst of the fire. \v{16}Be careful!\fnote{\fbackref{4:16} The Heb. lacks \fbib{be careful}} Otherwise, you will be destroyed when you make carved images for yourself---all sorts of images in the form of man, woman, \v{17}any animal on earth, any winged bird that flies in the sky, \v{18}any creeping thing on the ground, or any fish in the sea.\fnote{\fbackref{4:18} Lit. \fbib{in the waters below ground}} \v{19}Do not gaze toward the heavens and observe the sun, the moon, the stars---the entire array of the sky---with the intent\fnote{\fbackref{4:19} The Heb. lacks \fbib{with the intent}} to worship and serve what the \divine{Lord} your God gave every nation.\fnote{\fbackref{4:19} I.e. as lights in the night sky; cf. Gen 1:14-18} \v{20}For the \divine{Lord} took you and brought you out of the iron-smelting furnace---out of Egypt---to be the people of his inheritance, as you are today.

\v{21}``But the \divine{Lord} was angry with me because of you. So he swore that I'll never cross the Jordan River to enter the good land that the \divine{Lord} your God is about to give you as an inheritance. \v{22}I'm going to die in this land and I won't cross the Jordan River, but you're about to cross over to possess that good land. \v{23}Be careful! Otherwise, you will forget the covenant of the \divine{Lord} your God, who established that covenant with you. Don't make carved images of any likeness in violation of everything that you were commanded by the \divine{Lord} your God. \v{24}Indeed, the \divine{Lord} your God is a consuming\fnote{\fbackref{4:24} So MT; LXX reads \fbib{is an all-consuming}} fire. He is a jealous God.''
\passage{Warnings against Angering God}

\v{25}``After you've borne children and grandchildren, have been there for a long time in the land, have become so corrupted that you make images of any form, and have done evil in the eyes of the \divine{Lord} your God, you will provoke him to anger. \v{26}Heaven and earth will testify against what has occurred\fnote{\fbackref{4:26} Lit. \fbib{I invoke heaven and earth against you}} today: you'll surely and swiftly be destroyed from the land that you are about to possess by crossing the Jordan River. You won't live long in it, because you'll certainly be exterminated. \v{27}Moreover, the \divine{Lord} will scatter you among the nations, and you'll be fewer in number in the nations where the \divine{Lord} your God will drive you. \v{28}There you'll serve gods made by human hands, serving\fnote{\fbackref{4:28} The Heb. lacks \fbib{serving}} trees and stones that cannot see, hear, eat, nor smell. \v{29}If from there you will seek the \divine{Lord} your God, then you will find him if you seek him with all your heart and soul. \v{30}In your distress, when all these things happen to you in days to come and you return to the \divine{Lord} your God, then you will hear his voice. \v{31}For God is compassionate. The \divine{Lord} your God won't fail you. He won't destroy you or forget the covenant that he confirmed with your ancestors.''
\passage{Who is Like the \divine{Lord}?}

\v{32}``Indeed, ask from one end of the heavens to the other about days of old, before your time, when God created mankind on the earth. Did we ever have anything as great as this, or ever hear of anything like it? \v{33}Has any people heard the voice of God speaking from the middle of a fire just as you did,\fnote{\fbackref{4:33}Lit. \fbib{heard}} and survived it? \v{34}Or has any god ever taken for himself one nation out from another nation with testings, signs, wonders, wars, awesome power,\fnote{\fbackref{4:34} Lit. \fbib{wars, a mighty hand and an outstretched arm,}} and magnificent, terrifying deeds\fnote{\fbackref{4:34} The Heb. lacks \fbib{deeds}} as the \divine{Lord} your God did in Egypt before your eyes?

\v{35}``You have been shown this in order to know that `the \divine{Lord} is God' and there is no one like him. \v{36}You have been made to hear his voice from heaven so you may be instructed.\fnote{\fbackref{4:36} Or \fbib{disciplined}} And he showed you his great fire here on earth, and you heard his voice from the middle of that fire. \v{37}Moreover, he loved your ancestors, chose their descendants after them, and brought you out of Egypt, accompanied by his presence and great power, \v{38}in order to drive out nations that are stronger and more powerful than you, to bring you into this land,\fnote{\fbackref{4:38} The Heb. lacks \fbib{this land}} and to give you their land as an inheritance, as it is today.

\v{39}``May you acknowledge and take to heart this day that the \divine{Lord} is God in the heavens above and over the earth below---there is no other God.\fnote{\fbackref{4:39} The Heb. lacks \fbib{God}} \v{40}May you observe his statutes and keep his commands that I'm giving you today, so that life may go well for you and for your descendants after you. That way, you'll live a long life in the land that the \divine{Lord} your God is about to give you permanently.''\fnote{\fbackref{4:40} Lit. \fbib{all the days}}
\passage{Cities of Refuge}

\v{41}Then Moses designated three cities on the east side of the Jordan, \v{42}where a person who accidentally killed someone could flee, if he killed his neighbor without having enmity toward him in the past. He may flee to one of these cities and live: \v{43}Bezer in the desert plain for the descendants of Reuben, Ramoth in Gilead for the descendants of Gad, and Golan in Bashan for the descendants of Manasseh.
\passage{Moses Reviews the Law}

\v{44}This is the Law that Moses reviewed in the presence of the Israelis. \v{45}These are the instructions, decrees, and ordinances that Moses declared to the Israelis when they came out of Egypt. \v{46}He did this\fnote{\fbackref{4:46} The Heb. lacks \fbib{He did this}} east of the Jordan, in the valley opposite Beth Peor, in the land of Sihon, king of the Amorites, who lived in Heshbon, and whom Moses and the Israelis defeated after leaving Egypt. \v{47}So they took possession of his land, as well as the land of King Og of Bashan. Both Amorite kings lived east of the Jordan---\v{48}from Aroer on the edge of the Wadi\fnote{\fbackref{4:48} I.e. a seasonal stream or river that channels water during rain seasons but is dry at other times} Arnon as far as Mount Sirion,\fnote{\fbackref{4:48} MT reads \fbib{Sion}; cf. Deut 3:9} which is also called Hermon, \v{49}and all the Arabah east of the Jordan as far as the Dead Sea\fnote{\fbackref{4:49} Lit. \fbib{the Sea of the Arabah}} below the slopes of Pisgah.
\labelchapt{5}
\passage{The Ten Commandments}
\passageinfo{(Exodus 20:1-17)}

\chapt{5}
\v{1}Moses called all of Israel together and told them: ``Listen, Israel! Today I'm going to announce God's laws and regulations so that you will learn them and take care to obey them. \v{2}When the \divine{Lord} our God made a covenant with us in Horeb, \v{3}it was not with our ancestors that the \divine{Lord} made this covenant, but with us---we who are here today---all of us who are now living. \v{4}The \divine{Lord} spoke to you face to face on the mountain from the fire. \v{5}I stood at that time as mediator\fnote{\fbackref{5:5} The Heb. lacks \fbib{as mediator}} between the \divine{Lord} and you to declare his\fnote{\fbackref{5:5} Lit. \fbib{the \divine{Lord}'s}} message to you, because you were afraid of the fire and would not go up the mountain. He said:
\begin{bulletlist}
\itemb{\heb{'}\fnote{\fbackref{5:6-21} The Heb. letters to the left denote numbers 1-10}} \v{6}```I am the \divine{Lord} your God, who brought you out of the land of Egypt---from the house of slavery. \v{7}You are to have no other gods as a substitute for me.\fnote{\fbackref{5:7} Lit. \fbib{gods besides me}}
\itemb{\heb{b}} \v{8}```You are not to craft for yourselves an idol resembling what is in the skies above, or on earth beneath, or in the water sources under the earth. \v{9}You are not to bow down to them in worship or serve them, because I, the \divine{Lord} your God, am a jealous God, visiting the guilt of parents\fnote{\fbackref{5:9} Lit. \fbib{fathers}} on children, to the third and fourth generation\fnote{\fbackref{5:9} So LXX. The Heb. lacks \fbib{generation}} of those who hate me, \v{10}but showing gracious love to the thousands of those who love me and keep my\fnote{\fbackref{5:10} The MT is written \fbib{his} but is to be read \fbib{my}} commandments.
\itemb{\heb{g}} \v{11}```You are not to misuse the name of the \divine{Lord} your God,\fnote{\fbackref{5:11} Lit. \fbib{to take in vain the name of the \divine{Lord} your God}; i.e. for a worthless purpose} because the \divine{Lord} will not leave unpunished the one who misuses his name.\fnote{\fbackref{5:11} Lit. \fbib{who takes his name in vain} i.e. for a worthless purpose}
\itemb{\heb{d}} \v{12}```Observe the Sabbath day, maintaining its holiness,\fnote{\fbackref{5:12} Lit. \fbib{day as holy;} i.e. to set apart the day as holy} just as the \divine{Lord} your God commanded. \v{13}Six days you are to labor and do all your work, \v{14}but the seventh day is a Sabbath to the \divine{Lord} your God. You are not to do any work---neither you, your son, nor your daughter,\fnote{\fbackref{5:14} Lit. \fbib{your sons and your daughters}} your male and female servants, your oxen and donkeys, nor any of your livestock, nor any foreigner who lives among you---\fnote{\fbackref{5:14} Lit. \fbib{lives within your gates}} so that your male and female servants may rest as you do. \v{15}You are to remember that you were a slave in the land of Egypt, but the \divine{Lord} your God brought you out from there with great power and a show of force.\fnote{\fbackref{5:15} Lit. \fbib{with a mighty hand and an outstretched arm}} Therefore, the \divine{Lord} your God has commanded you to observe the Sabbath day.
\itemb{\heb{h}} \v{16}```Honor your father and your mother, just as the \divine{Lord} your God commanded you, so that you will live long and things will go well for you in the land that the \divine{Lord} your God is giving you.
\itemb{\heb{w}} \v{17}```You are not to commit murder.
\itemb{\heb{z}} \v{18}```You are not to commit adultery.
\itemb{\heb{.h}} \v{19}```You are not to steal.
\itemb{\heb{.t}} \v{20}```You are not to give false testimony against your neighbor.
\itemb{\heb{y}} \v{21}```You are not to desire\fnote{\fbackref{5:21} Lit. \fbib{to covet}; i.e. to set your heart on} your neighbor's wife nor crave your neighbor's house,\fnote{\fbackref{5:21} Or \fbib{neighbor's family dynasty}} his fields, his male and female servants, his ox, his donkey, nor anything else that pertains to your neighbor.'\,''
\end{bulletlist}
\passage{Moses Recalls God's Warnings}

\v{22}``The \divine{Lord} declared these commands in a loud voice to your entire assembly on the mountain from out of the fire\fnote{\fbackref{5:22} LXX Sam Pentateuch read \fbib{dark}; cf. Deut 4:11} and dark clouds,\fnote{\fbackref{5:22} Lit. \fbib{cloud and thick darkness}} and nothing more was added. He inscribed them on two tablets of stone and gave them to me. \v{23}When you heard the voice from the darkness while the mountain was blazing, all the leaders and elders of your tribes came to me and said: \v{24}`The \divine{Lord} our God truly has displayed his glory and power, for we heard him\fnote{\fbackref{5:24} The Heb. lacks \fbib{him}} from out of the fire today. We have witnessed how God spoke to human beings, yet they lived. \v{25}Now therefore, why should we die? This great fire will consume us. If we continue to listen to the voice of the \divine{Lord} our God any longer, we'll die. \v{26}For what mortal man\fnote{\fbackref{5:26} Lit. \fbib{For who among all flesh}} has heard the voice of the living God speaking out of the fire like we did, and lived? \v{27}As for you, go near and listen to everything that the \divine{Lord} our God will say to you, then repeat it\fnote{\fbackref{5:27} Lit. \fbib{Then tell everything that the \divine{Lord} our God will speak}} to us, and we'll listen and obey.'

\v{28}``The \divine{Lord} heard what you said. He told me: `I've heard what this people said. Everything they said was good. \v{29}If only they would commit\fnote{\fbackref{5:29} Lit. \fbib{only their heart would incline}} to fear me and keep all my commands, then it will go well with them and their children forever.\v{30}Go and tell them to return to their tents, \v{31}but you stand here with me and I'll speak to you all the commands, decrees, and laws that you must teach them to observe in the land that I'm giving you to possess. \v{32}You must be careful to do what the \divine{Lord} your God commanded you, turning neither to the left nor to the right. \v{33}You are to walk in every pathway that the \divine{Lord} your God commanded you, so that life\fnote{\fbackref{5:33} Lit. \fbib{it}} may go well for you, and so that you will prolong your days in the land that you will possess.'\,''
\labelchapt{6}
\passage{The Covenant of Love}

\chapt{6}
\v{1}``Now these are the commands, decrees, and ordinances that the \divine{Lord} commanded me\fnote{\fbackref{6:1} The Heb. lacks \fbib{me}} to teach you. Obey them in the land you are entering to possess, \v{2}so that you, your children, and your grandchildren may fear the \divine{Lord} your God. Keep all his decrees and commandments that I'm giving you every day of your life, so you may live a long time. \v{3}Listen, Israel! Be careful to obey, so that life\fnote{\fbackref{6:3} Lit. \fbib{it}} may go well for you and that you may increase greatly. Just as the \divine{Lord} God of your ancestors told you, you'll have a land flowing with milk and honey.

\v{4}``Listen, Israel! The \divine{Lord} is our God, the \divine{Lord} alone.\fnote{\fbackref{6:4} Or \fbib{The \divine{Lord} our God, the \divine{Lord} is one.}} \v{5}You are to love the \divine{Lord} your God with all your heart, all your soul, and all your strength. \v{6}Let these words that I'm commanding you today be always\fnote{\fbackref{6:6} The Heb. lacks \fbib{always}} on your heart. \v{7}Teach them repeatedly to your children. Talk about them while sitting in your house or walking on the road, and as you lie down or get up. \v{8}Tie them as reminders\fnote{\fbackref{6:8} Lit. \fbib{signs}} on your forearm, bind them on your forehead,\fnote{\fbackref{6:8} Lit. \fbib{them as frontlets between your eyes}} \v{9}and write them on the door frames of your house and on your gates.''
\passage{Serve the \divine{Lord} Only}

\v{10}``When the \divine{Lord} your God brings you to the land that he promised to your ancestors Abraham, Isaac, and Jacob, he will give you large and beautiful cities that you didn't build, \v{11}houses filled with every good thing that you didn't supply, wells that you didn't dig, and vineyards and olive groves that you didn't plant. When you eat and are satisfied, \v{12}be careful not to forget the \divine{Lord} your God, who brought you out of the land of Egypt and slavery.\fnote{\fbackref{6:12} Lit. \fbib{Egypt, out of the house of slavery}} \v{13}Fear the \divine{Lord} your God, serve him, and make your oaths in his name. \v{14}Do not follow other gods from the gods of the nations\fnote{\fbackref{6:14} Lit. \fbib{peoples}} around you, \v{15}because the \divine{Lord} your God who is among you is a jealous God. He will turn his anger against you and destroy you from the surface of the land.''
\passage{Do What is Right}

\v{16}``Don't test the \divine{Lord} your God like you did in Massah. \v{17}Be sure to observe the commands of the \divine{Lord} your God, his testimonies, and his decrees that he gave you. \v{18}Do what is good and right in the \divine{Lord}'s sight so it may go well with you. Then you'll enter and possess the good land that the \divine{Lord} your God promised to your ancestors, \v{19}expelling all your enemies before you, as the \divine{Lord} said.''
\passage{Remember What the \divine{Lord} has Done}

\v{20}``When your son asks you in the future, `What is the meaning of the instructions, decrees, and ordinances that the \divine{Lord} our God commanded you?' \v{21}tell him, `We were slaves to Pharaoh in Egypt, but the \divine{Lord} brought us out of Egypt with great power. \v{22}Before our very eyes, the \divine{Lord} did great and terrible signs and wonders in Egypt---to Pharaoh and to his entire household. \v{23}But as for us, he brought us out from there to bring us into the land and give it to us, as he promised our ancestors. \v{24}Then the \divine{Lord} commanded us to observe all these decrees and to fear the \divine{Lord} our God for our own good, so that he may keep us alive as we are today. \v{25}It will be credited as\fnote{\fbackref{6:25} The Heb. lacks \fbib{credited as}} righteousness for us if we're careful to obey the entire Law in the presence of the \divine{Lord} our God, as he commanded.'\,''
\labelchapt{7}
\passage{Instructions Regarding the Tribal Nations}

\chapt{7}
\v{1}``When the \divine{Lord} your God brings you into the land that you are entering to possess, he will drive out many nations before you: the Hittites, Girgashites, Amorites, Canaanites, Perizzites, Hivites, and Jebusites--- seven nations who are more numerous and stronger than you. \v{2}So when the \divine{Lord} your God delivers them to you and you have defeated them, then utterly destroy them. You are not to make any covenant with them nor be gracious to them. \v{3}You are not to intermarry with them. You are not to give your daughters to their sons nor take their daughters for your sons, \v{4}because they will turn your children from me to serve other gods so that the \divine{Lord}'s anger blazes against you and swiftly destroys you by fire. \v{5}This is what you are to do to them: tear down their altars, break their pillars, cut down their ritual pillars, and burn their carved idols in fire, \v{6}because you are a holy people to the \divine{Lord} your God. The \divine{Lord} your God chose you to be his people, his treasured possession from all the nations\fnote{\fbackref{7:6} Lit. \fbib{peoples}} on the face of the earth.''
\passage{The \divine{Lord} Keeps His Covenant}

\v{7}``It wasn't because you were more numerous than other nations\fnote{\fbackref{7:7} Lit. \fbib{peoples}} of the earth that the \divine{Lord} committed himself to you and chose you. In fact, you were the least numerous of all the nations.\fnote{\fbackref{7:7} Lit. \fbib{peoples}} \v{8}But the \divine{Lord} loved you and kept his oath that he made to your ancestors. The \divine{Lord} brought you out with great power from slavery,\fnote{\fbackref{7:8} Lit. \fbib{from the house of slaveries}} from the control of Pharaoh, king of Egypt. \v{9}Know that the \divine{Lord} your God is God, the trusted God who faithfully keeps his covenant to the thousandth generation of those who love him and obey his commands. \v{10}But for the one who hates him, he will repay him by destroying him. He will not delay dealing with someone who hates him. \v{11}Therefore, keep the commands, decrees, and ordinances that I am instructing you to obey today.''
\passage{The \divine{Lord} Blesses Obedience}

\v{12}``If you pay attention to these laws and obey them, then the \divine{Lord} your God will continue his covenant of gracious love with you that he promised with an oath to your ancestors. \v{13}He'll love you and increase your numbers. He'll bless the fruit of your womb, the fruit of your land (the grain, new wine, and oil), the offspring of your herds, and the lambs of your flock in the land that the \divine{Lord} promised your ancestors he would give you. \v{14}You'll be blessed among all the nations. There'll be no infertility among you, not even\fnote{\fbackref{7:14} Lit. \fbib{you, neither infertility}} among your herds. \v{15}The \divine{Lord} will turn aside every disease from you. He won't inflict on you the terrible diseases you knew in Egypt, but will inflict them instead on all who hate you. \v{16}You are to utterly destroy everyone whom the \divine{Lord} your God will deliver to you. Don't have pity on them nor serve their gods. Otherwise, they will become a snare for you.''
\passage{The \divine{Lord} will Fight for You}

\v{17}``You may say to yourselves, `These nations are more numerous than we are. How can we dispossess them?' \v{18}But you mustn't fear them. Be sure to remember what the \divine{Lord} your God did to Pharaoh and all of Egypt. \v{19}Your eyes saw the great trials, the signs and wonders, and the awesome power with which\fnote{\fbackref{7:19} Lit. \fbib{wonders, with a mighty hand and an outstretched arm}} the \divine{Lord} your God brought you out. The \divine{Lord} your God will do the same to all the people whom you fear. \v{20}He'll\fnote{\fbackref{7:20} Lit. \fbib{The \divine{Lord} your God will}} send plagues against them until the survivors who hide from you have perished. \v{21}Don't tremble before them, because the \divine{Lord} your God, who is among you, is a great and awesome God. \v{22}He\fnote{\fbackref{7:22} Lit. \fbib{The \divine{Lord} your God}} slowly will dislodge these nations before you, but he won't destroy them quickly, so the wild animals\fnote{\fbackref{7:22} Lit. \fbib{the beasts of the field}} won't multiply around you. \v{23}But the \divine{Lord} your God will deliver them over to you, throwing them into great confusion, until they are destroyed. \v{24}He will deliver kings into your control and you are to wipe out the memory of them\fnote{\fbackref{7:24} Lit. \fbib{will cause their names to perish}} from under heaven. No one will be able to stand before you. You are utterly to destroy them. \v{25}Burn the images of their gods in the fire. Desire neither the silver nor the gold that adorns them, nor take them for yourselves, so you won't be ensnared by them, because the gold and silver\fnote{\fbackref{7:25} Lit. \fbib{because it}} are detestable to the \divine{Lord} your God. \v{26}Don't bring any detestable thing to your house, because you yourself will be utterly destroyed along with these detestable things. You must absolutely abhor and detest all of\fnote{\fbackref{7:26} The Heb. lacks \fbib{all of}} it, because it has been devoted to destruction.''
\labelchapt{8}
\passage{Remember the \divine{Lord}'s Provisions}

\chapt{8}
\v{1}``Be careful to observe every command that I'm instructing you today, in order that you may live, increase, and enter and take possession of the land that the \divine{Lord} promised by an oath to your ancestors. \v{2}Remember how the \divine{Lord} your God led you all the way these 40 years in the desert to humble\fnote{\fbackref{8:2} Or \fbib{afflict}} and test you in order to make known what was in your heart---whether or not you would keep his commands. \v{3}He humbled\fnote{\fbackref{8:3} Or \fbib{afflicted}} you, causing you to be hungry, yet he fed you with manna that neither you nor your ancestors had known, in order to teach you that human beings are not to live by food alone---instead human beings are to live by every word that proceeds from the mouth of the \divine{Lord}.

\v{4}``The clothes you wore\fnote{\fbackref{8:4} Lit. \fbib{clothes from on you}} did not wear out, nor did your feet blister during these 40 years. \v{5}Be convinced in your heart that as a father disciplines his son, so the \divine{Lord} your God disciplines you. \v{6}Observe the commands of the \divine{Lord} your God by walking in his ways and by fearing\fnote{\fbackref{8:6} Or \fbib{revering}} him, \v{7}because the \divine{Lord} your God is bringing you to a good land---a land with rivers and deep springs flowing to the valleys and hills. \v{8}It's a land filled\fnote{\fbackref{8:8} The Heb. lacks \fbib{filled}} with wheat, barley, vines, fig trees, and pomegranates. It's a land filled\fnote{\fbackref{8:8} The Heb. lacks \fbib{filled}} with olive oil and honey--- \v{9}a land without scarcity. You'll eat food in it and lack nothing. It's a land where its rocks are iron and you can dig copper from its mountains.''
\passage{Remember the Source of Blessings}

\v{10}``When you have eaten and are satisfied, bless the \divine{Lord} your God for the good land that he has given you. \v{11}Be careful! Otherwise, you will forget the \divine{Lord} your God by failing to keep his commands, ordinances, and statutes that I'm commanding you this day. \v{12}Otherwise, when you eat and are satisfied, when you have built beautiful houses and lived in them, \v{13}when your cattle and oxen have multiplied, and when your silver and gold have increased, \v{14}then you will become arrogant. You'll neglect the \divine{Lord} your God, \v{15}who brought you out of the land of Egypt---from the house of slavery---and who led you through the vast and dangerous desert---that parched land without water---with its poisonous snakes and scorpions. He brought water out of solid rock for you \v{16}and fed you in the desert with manna that neither you nor your ancestors had known to humble and test you so that things may go well with you later. \v{17}You may say to yourselves, `I have become wealthy by my own strength and by my own ability.'\fnote{\fbackref{8:17} Lit. \fbib{by the power of my hand}} \v{18}But remember the \divine{Lord} your God, because he is the one who gives you the ability to produce wealth, in order to confirm his covenant that he promised by an oath to your ancestors, as is the case today. \v{19}If you neglect the \divine{Lord} your God, follow other gods, and serve and worship them, I testify to you today that you will certainly be destroyed. \v{20}Just like the nations whom the \divine{Lord} destroyed before you, so will you be destroyed, because you did not listen to the voice of the \divine{Lord} your God.''
\labelchapt{9}
\passage{When the \divine{Lord} Fulfills His Promise}

\chapt{9}
\v{1}``Listen, Israel! Today you are about to cross the Jordan to enter and dispossess greater and mightier nations than you, who live in\fnote{\fbackref{9:1} The Heb. lacks \fbib{who live in}} large cities that are fortified to the sky. \v{2}The Anakim\fnote{\fbackref{9:2} Or \fbib{giants}; cf. Num 13:22, 33} are strong and tall, and you know them. You've heard it said, `Who can stand up against the Anakim?'\fnote{\fbackref{9:2} Or \fbib{giants}; cf. Num 13:22, 33} \v{3}But know today that the \divine{Lord} your God is going ahead of you as a consuming fire. He will destroy and subdue them before you. He will dispossess and destroy them quickly, just as the \divine{Lord} told you. \v{4}After the \divine{Lord} has expelled them before you, you are not to say to yourselves, `The \divine{Lord} caused me to enter and possess this land because of my righteousness.' \v{5}On the contrary, it is because of the wickedness of these nations that the \divine{Lord} is dispossessing them before you to confirm what the \divine{Lord} promised by an oath to your ancestors Abraham, Isaac, and Jacob. \v{6}Know that it is not because of your righteousness that the \divine{Lord} your God is giving to you this good land to inherit, since you are a stubborn people.''
\passage{Israel Broke the Covenant}

\v{7}``Remember---and don't ever forget---how you provoked the \divine{Lord} your God in the desert. From the day that you came out of the land of Egypt until you came to this place, you have been rebelling against the \divine{Lord}. \v{8}At Horeb you continually rebelled against the \divine{Lord} so that he\fnote{\fbackref{9:8} Lit. \fbib{the \divine{Lord}}} was angry enough to destroy you. \v{9}Then I went up to the mountain to receive the two stone Tablets of the Covenant that the \divine{Lord} had established with you. I stayed on the mountain for 40 days and nights without eating food or drinking water. \v{10}Then the \divine{Lord} gave me the two stone tablets on which God inscribed with his own finger all the words that the \divine{Lord} spoke to you on the mountain from the middle of the fire that day when you were all assembled together. \v{11}At the end of 40 days and nights, the \divine{Lord} gave to me the two stone Tablets of the Covenant.

\v{12}``Then the \divine{Lord} told me, `Get going! Go down from here at once! Your people whom you brought out of Egypt have become corrupt. They have turned quickly from the way that I commanded them and have cast an idol for their use.'

\v{13}``Then the \divine{Lord} told me, `I have examined this people, and they\fnote{\fbackref{9:13} Lit. \fbib{and the people}} are stubborn indeed. \v{14}Let me alone! I will destroy them, blot out their name from heaven, and then I'll make you into a nation that will be mighty and more numerous than they are.'

\v{15}``So I turned and went down from the mountain while the mountain was on fire. The two Tablets of the Covenant were in both of my hands. \v{16}Then I saw how you had really sinned against the \divine{Lord} your God! You had made for yourselves a calf---a cast idol. You had turned aside quickly from the way that the \divine{Lord} your God had commanded. \v{17}So I grabbed the two tablets and threw them out of my hands, breaking them before your eyes. \v{18}I fell down in the \divine{Lord}'s presence, just as I had the first 40 days and nights. I didn't eat food or drink water because of your sin. You had sinned by committing this evil in full view of the \divine{Lord}, thereby provoking him to anger. \v{19}I feared the anger and wrath of the \divine{Lord} against you, because he was irate enough to destroy you. But the \divine{Lord} also listened to me at that time. \v{20}It was as had been the case with Aaron---the \divine{Lord} was very angry and about to destroy him, but I prayed for Aaron at that time. \v{21}So when you made the calf that made you sin, I grabbed it, burned it with fire, crushed it, and ground it thoroughly until it was pulverized to powder. Then I threw the powder into the river that was flowing from the mountain.''
\passage{Moses Interceded for Israel}

\v{22}``You provoked the \divine{Lord} again at Taberah, Massah, and Kibroth-hattaavah. \v{23}When the \divine{Lord} sent you from Kadesh-barnea and told you, `Go possess the land that I gave you,' instead you disobeyed what the \divine{Lord} your God said. You didn't trust him or listen to his voice. \v{24}You have been rebelling against the \divine{Lord} since the day I knew you. \v{25}I fell down in the \divine{Lord}'s presence for 40 days and nights, because the \divine{Lord} said he was ready to destroy you. \v{26}So I prayed to the \divine{Lord} and said, `Oh \divine{Lord} my God, don't destroy your people and your inheritance whom you redeemed by your power.\fnote{\fbackref{9:26} Lit. \fbib{redeemed in your greatness}} You brought them out from Egypt in a powerful way. \v{27}Remember your servants Abraham, Isaac, and Jacob. Don't pay attention to the stubbornness, wickedness, and sinfulness of this people. \v{28}Otherwise, the people of the land from which you brought us will say, ``The \divine{Lord} wasn't able to bring them into the land that he had promised them. So he brought them out to kill them in the desert because he hated them.'' \v{29}But they are your people and inheritance, whom you brought out by your mighty strength\fnote{\fbackref{9:29} Lit. \fbib{arm}} and awesome power.'\,''
\labelchapt{10}
\passage{A Copy of the Ten Commandments}

\chapt{10}
\v{1}``At that time, the \divine{Lord} told me, `Chisel two tablets of stone for yourself just like the first ones, and then come up to me on the mountain. Also make for yourself a wooden chest. \v{2}I'll write on the tablets what was\fnote{\fbackref{10:2} Lit. \fbib{tablets of words that were}} on the first tablets that you broke, and then you are to place them into the wooden chest.' \v{3}So I made a chest out of acacia wood and chiseled two tablets of stone just like the first ones. Then I went up the mountain with the two tablets in my hands. \v{4}Then the \divine{Lord}\fnote{\fbackref{10:4} Lit. \fbib{he}} inscribed on the tablets what he wrote before---that is, the Ten Commandments that the \divine{Lord} declared to you on the mountain from the middle of the fire during the day of the assembly. And the \divine{Lord} gave them to me. \v{5}Then I turned, went down the mountain, and placed the tablets in the chest that I had made. They are there now, just as the \divine{Lord} commanded me.''
\passage{Aaron Dies and the Descendants of Levi are Appointed}

\v{6}``The Israelis traveled from the wells of the descendants of Jaakan to Moserah. Aaron died and was buried there. His son Eleazar succeeded him as priest. \v{7}From there they moved on to Gudgodah and from Gudgodah to Jotbathah, a land with flowing streams. \v{8}At that time the \divine{Lord} set apart the tribe of Levi to carry the Ark of the Covenant of the \divine{Lord}, to stand in the \divine{Lord}'s presence, to serve, and to bless his name until this day. \v{9}That is why the descendants of Levi do not have a portion and an inheritance among their relatives. As for\fnote{\fbackref{10:9} The Heb. lacks \fbib{As for}} the \divine{Lord}, he is their inheritance, just as the \divine{Lord} your God told them. \v{10}When I stood on the mountain for 40 days and 40 nights as I did the first time, the \divine{Lord} listened to me once again. The \divine{Lord} was not willing to destroy you. \v{11}So the \divine{Lord} told me, `Get up and proceed to lead\fnote{\fbackref{10:11} Lit. \fbib{to a journey before}} the people, so they may enter and take possession of the land that I promised to give their ancestors by an oath.'\,''
\passage{Love the \divine{Lord}}

\v{12}``Now Israel, what does the \divine{Lord} your God desire from you? Only this: fear him,\fnote{\fbackref{10:12} Lit. \fbib{the \divine{Lord} your God}} walk in all his ways, love him, serve him\fnote{\fbackref{10:12} Lit. \fbib{the \divine{Lord} your God}} with all your heart and in all your life,\fnote{\fbackref{10:12} Or \fbib{soul}} \v{13}and observe his\fnote{\fbackref{10:13} Lit. \fbib{observe the \divine{Lord}'s}} commands and statutes that I'm commanding you today for your own good. \v{14}You see, heaven---even the highest heavens---belongs to the \divine{Lord}, along with the earth and all that is in it, \v{15}yet the \divine{Lord} committed himself to love your ancestors---and did so! He chose you---their descendants after them---from all the nations, as it is\fnote{\fbackref{10:15} The Heb. lacks \fbib{it is}} today. \v{16}Therefore, circumcise your heart and stop being stubborn. \v{17}For the \divine{Lord} your God is the God of all gods, the \divine{Lord} of all lords, the great God, mighty and awesome, who does not show favoritism or take bribes. \v{18}He executes justice for the orphan and the widows, loves the foreigner, and gives them food and clothing.''
\passage{Love Others}

\v{19}``You are to love the foreigner, because you were foreigners in the land of Egypt. \v{20}You are to fear the \divine{Lord} your God and serve him. Cling to him and swear by his name. \v{21}He is the one you are to praise, because he is\fnote{\fbackref{10:21} Lit. \fbib{He is your praise,}} your God who carried out those great and awesome things for you that you witnessed. \v{22}Your ancestors went down to Egypt with 70 people, but the \divine{Lord} your God has now made you as numerous as the stars in the sky.''
\labelchapt{11}
\passage{Remember God's Power}

\chapt{11}
\v{1}``Therefore love the \divine{Lord} your God and be very careful to keep his injunctions, statutes, ordinances, and commands all the time.\fnote{\fbackref{11:1} Lit. \fbib{days}} \v{2}Keep in mind today that I am not speaking to your children, who neither were aware of nor did they witness the discipline of the \divine{Lord} your God---that is, his great and far-reaching power, \v{3}including: the signs and works that he did within Egypt to Pharaoh, king of Egypt, and to all his land; \v{4}what he did to the Egyptian army, to its horses, and to its chariots, when he caused the water of the Reed\fnote{\fbackref{11:4} So MT; LXX reads \fbib{Red}} Sea to engulf them as they pursued you; how the \divine{Lord} destroyed them, even to this day; \v{5}what he did for you in the desert until you came to this place; \v{6}and what he did to Eliab's sons Dathan and Abiram, descendants of Reuben, when the ground opened up and swallowed them, their households, their tents, and every living thing belonging to them in the full sight\fnote{\fbackref{11:6} Lit. \fbib{the middle}} of Israel. \v{7}Your very own eyes saw all the great things that the \divine{Lord} did.''
\passage{Possessing a Fertile Land}

\v{8}``Keep all the commands that I'm giving\fnote{\fbackref{11:8} Lit. \fbib{commanding}} you today, so you can be strong enough to enter and possess the land that you are crossing over to inherit \v{9}and so you'll live long in the land that the \divine{Lord} your God promised by an oath to give your ancestors and their descendants---a land flowing with milk and honey, \v{10}since the land that you are about to enter to inherit isn't like the land of Egypt that you just left, where you plant a seed and irrigate it with your feet like a vegetable garden. \v{11}Instead, the land that you are crossing over to inherit is a land of hills and valleys that drinks water supplied by rain from heaven, \v{12}a land about which the \divine{Lord} your God is always concerned, because the eyes of the \divine{Lord} are continuously on it throughout the entire year.''\fnote{\fbackref{11:12} Lit. \fbib{on it from the beginning of the year until the end of the year}}
\passage{Delights of a Bountiful Land}

\v{13}``If you carefully observe the commands that I'm giving you today---that is, to love the \divine{Lord} your God and serve him with all your heart and soul--- \v{14}then he\fnote{\fbackref{11:14} So with LXX, SP, V; MT reads \fbib{I}} will send rain on the land in its season (the early and latter rains)\fnote{\fbackref{11:14} I.e. winter and spring rains} and you'll gather grain, new wine, and oil. \v{15}He\fnote{\fbackref{11:15} So with LXX, SP, V; MT reads \fbib{I}} will provide grass in the fields for your livestock, and you'll eat and be satisfied. \v{16}Be careful! Otherwise, your hearts will deceive you and you will turn away to serve other gods and worship them. \v{17}The wrath of God will burn against you so that he will restrain the heavens and it won't rain. The ground won't yield its produce and you'll be swiftly destroyed from the good land that the \divine{Lord} is about to give you. \v{18}Take these commands to heart and keep them in mind, tying them as reminders on your arm and as bands on your forehead. \v{19}Teach them to your children, talking about them while sitting in your house, walking on the road, or when you are about to lie down or get up. \v{20}Also write them upon the doorposts of your house and gates,\fnote{\fbackref{11:18-20} cf. Deut 6:7-8} \v{21}so that you and your children may live a long time in the land that the \divine{Lord} promised to give your ancestors---as long as the sky remains above the earth.''
\passage{Boundaries of the Land}

\v{22}``If you carefully observe all of these commands that I'm giving you to do---to love the \divine{Lord} your God, to walk in all his ways, and to cling to him--- \v{23}then the \divine{Lord} will dispossess all these nations before you and you'll dispossess nations that are even greater and stronger than you. \v{24}Every place upon which the soles of your feet tread will be yours as boundaries---from the desert to Lebanon and from the River (that is, from the Euphrates) to the Mediterranean\fnote{\fbackref{11:24} Lit. \fbib{Western}} Sea. \v{25}No one will be able to stand against you. The \divine{Lord} your God will instill terror and fear of you throughout the entire land wherever you go, just as he promised you. \v{26}Look! I'm about to grant you a blessing and a curse--- \v{27}a blessing if you obey the commands of the \divine{Lord} your God that I'm giving you today, \v{28}or a curse if you don't obey the commands of the \divine{Lord} your God, by turning from the way that I'm commanding you today and following other gods whom you have not known.''
\passage{Declaration of the Blessings and Curses}

\v{29}``When the \divine{Lord} brings you to the land that you are about to enter to inherit, repeat the blessings on Mount Gerizim and the curses on Mount Ebal. \v{30}They're across the Jordan River to the west in the land of the Canaanites who live in the Arabah opposite Gilgal near the Oak of Moreh, aren't they? \v{31}For you are about to cross the Jordan River to go in and possess the land that the \divine{Lord} your God is about to give you to inherit and live in. \v{32}Be careful to obey all the statutes and ordinances that I'm placing before you today.''
\labelchapt{12}
\passage{Destroying Altars to False Gods}

\chapt{12}
\v{1}``These are the statutes and ordinances that you must carefully observe in the land that the \divine{Lord} God of your ancestors has given you to possess every day that you live on the earth. \v{2}Be sure you destroy there all the places where the nations that you're going to dispossess serve their gods---upon the high mountains and hills and under every leafy tree. \v{3}Tear down their altars, then cut down their sacred poles\fnote{\fbackref{12:3} Lit. \fbib{their Ashram}; i.e. cultic pillars} and burn them. Cut down the carved images of their gods in order to destroy their names from that place.''
\passage{Sacrifice at the Central Sanctuary}

\v{4}``You must not act like this with respect to the \divine{Lord} your God. \v{5}Instead, you must seek to enter only the place that the \divine{Lord} your God will choose among your tribes. There he will establish his name and live. \v{6}Bring your burnt offerings there, along with your sacrifices, your tithes, your hand-carried gifts, your offerings in fulfillment of promises, your freely-given offerings, and the firstborn of your herds and flocks. \v{7}Then you and your household will eat in the presence of the \divine{Lord} your God and rejoice with all the things you have made by your own effort\fnote{\fbackref{12:7} Lit. \fbib{hand}} and with which he\fnote{\fbackref{12:7} Lit. \fbib{the \divine{Lord}}} blessed you.

\v{8}``You must not act as we have been doing here today, where everyone acts as they see fit, \v{9}since you haven't arrived yet to your allotted place\fnote{\fbackref{12:9} Lit. \fbib{allotted resting place and inheritance}} that the \divine{Lord} your God is about to give you. \v{10}But after you have crossed the Jordan River and settled in the land that the \divine{Lord} your God is giving you to inherit, and after you have received relief from the enemies around you and are living securely, \v{11}then bring to the place that the \divine{Lord} your God will choose as a dwelling place---where he will establish his name---everything that I'm commanding you: your burnt offerings, your sacrifices, your tithes, your hand-carried gifts, and all your best offerings in fulfillment of promises that you pledged to the \divine{Lord}.

\v{12}``Rejoice in the presence of the \divine{Lord} your God---you, your sons and daughters, your male and female servants, and the descendant of Levi who is in your city---because there is no territorial allotment\fnote{\fbackref{12:12} Lit. \fbib{no portion and possession}} for him as you have. \v{13}Be careful not to offer burnt offerings at any location you happen to see\fnote{\fbackref{12:13} Lit. \fbib{you see}} \v{14}instead of at the place the \divine{Lord} will choose in one of the tribal areas. There you may offer burnt offerings, and there you may do everything that I'm commanding you.''
\passage{Instructions Pertaining to Food}

\v{15}``You may slaughter and eat as much meat as you desire,\fnote{\fbackref{12:15} Lit. \fbib{with all the desire of your soul}} according to the blessing of the \divine{Lord} your God, when he provides for you in all your cities.\fnote{\fbackref{12:15} Lit. \fbib{gates}} Both ritually unqualified and qualified people\fnote{\fbackref{12:15} Lit. \fbib{unclean and clean}; and so throughout the book} may eat it as they would gazelle and deer. \v{16}However, you are not to consume the blood;\fnote{\fbackref{12:16} Cf. Acts 15:20, 29} instead, you are to pour it out on the ground as you would water.

\v{17}``You won't be allowed to eat your tithe of grain, new wine, or oil, the firstborn of your herd and flock, your voluntary offerings that you pledged, your free-will offerings, or the works of your hands in your own cities. \v{18}You'll eat only in the presence of the \divine{Lord} your God at the place that he\fnote{\fbackref{12:18} Lit. \fbib{that the \divine{Lord} your God}} will choose---you, your sons and your daughters, your male and female servants, and the descendant of Levi who is in your cities.\fnote{\fbackref{12:18} Lit. \fbib{gates}} Rejoice in the presence of the \divine{Lord} your God in everything you undertake.\fnote{\fbackref{12:18} Lit. \fbib{in every work of your hand}} \v{19}Be careful not to forget the descendant of Levi while you live\fnote{\fbackref{12:19} Lit. \fbib{all your days in the land}} in the land. \v{20}When the \divine{Lord} your God enlarges your territory---just as he told you---and you say `I want to eat meat' since you desire to eat it,\fnote{\fbackref{12:20} Lit. \fbib{meat}} you may do so as much as you please.\fnote{\fbackref{12:20} Lit. \fbib{may eat flesh with all the desire of your soul}}

\v{21}``If the place where the \divine{Lord} your God chooses to establish his name is distant from you, then you may slaughter from your herd and your flock what the \divine{Lord} has provided for you, as he instructed you. You may consume them in your cities\fnote{\fbackref{12:21} Lit. \fbib{gates}} as much as you please. \v{22}You may eat them just as you would gazelle and deer. Ritually unqualified and qualified people may eat them. \v{23}Only be sure to refrain from eating blood, because blood is the source of\fnote{\fbackref{12:23} The Heb. lacks \fbib{source of}} life and you are not to consume blood with the meat. \v{24}You are not to eat it; instead, you are to pour it on the ground as you would water. \v{25}You are not to eat it, so that life may go well for you and for your children after you. Then you'll do what is right in the eyes of the \divine{Lord}.

\v{26}``You may carry and bring only your consecrated gifts and offerings in fulfillment of promises to the place that the \divine{Lord} will choose. \v{27}You must offer your burnt offerings---both the meat and the blood---on the altar of the \divine{Lord} your God. You are to offer the blood by pouring it on the altar of the \divine{Lord} your God while you consume the meat. \v{28}Be sure to observe all these words that I'm commanding you, in order that life may go well for you and your children after you forever, for this is good and right in the eyes of the \divine{Lord} your God.''
\passage{Don't Become Ensnared}

\v{29}``When the \divine{Lord} your God eliminates the nations that you are about to dispossess so you can live in their land, \v{30}after they have been destroyed in your sight, be careful not to be ensnared as they were. Otherwise, you will seek their gods and ask yourselves, `How do these nations serve their gods? I will do likewise.' \v{31}You must not do the same to the \divine{Lord} your God, because they practiced in the presence of their gods every sort of abomination that the \divine{Lord} hates. Moreover, they sacrificed\fnote{\fbackref{12:31} Lit. \fbib{they burned in fire}} their sons and daughters to their gods. \v{32}\fnote{\fbackref{12:32} This v. is 13:1 in MT}Now as to everything I'm commanding you, you must be careful to observe it. Don't add to or subtract from it.''
\labelchapt{13}
\passage{Dealing with False Prophets}

\chapt{13}
\v{1}\fnote{\fbackref{13:1} This v. is 13:2 in MT}``A prophet or a diviner of dreams may arise among you, give you an omen or a miracle \v{2}that takes place, and then he may tell you, `Let's follow other gods (whom you have not known) and let's serve them.' Even though the sign or portent comes to pass, \v{3}you must not listen to the words of that prophet or that diviner of dreams. For the \divine{Lord} your God is testing you, to make known whether or not you'll continue to love the \divine{Lord} your God with all your heart and soul. \v{4}You must follow the \divine{Lord} your God, fear him, observe his commandments, listen to his voice, serve him, and cling to him. \v{5}That prophet or diviner of dreams must be executed, because he advocated rebellion against the \divine{Lord} your God, who brought you from the land of Egypt and redeemed you from the house of slavery, and because he lured you from the way in which the \divine{Lord} your God instructed you to live. Purge the evil from among you.''
\passage{Dealing with Idolaters}

\v{6}``Your own blood brother,\fnote{\fbackref{13:6} Lit. \fbib{your brother, the son of your mother}} your son, your daughter, your beloved wife, or your friend who is like your soul mate may entice you quietly. He may tell you, `Let's go and serve other gods' (whom neither you nor your ancestors have known \v{7}from the gods of the people that surround you---whether near or far from you---from one end of the earth to the other). \v{8}You are not to yield to him, listen to him, look with pity on him, show compassion to him, or even cover up for him. \v{9}Instead, you are surely to execute him. You must be the first to put him to death with your own hand, and then the hands of the whole community. \v{10}Stone him to death, because he sought to lure you from the \divine{Lord} your God, who brought you from the land of Egypt, from the land of slavery. \v{11}Then all Israel will hear about it, be afraid, and won't do this evil thing again among you.

\v{12}``You may hear in one of your towns that the \divine{Lord} your God is giving you to inhabit \v{13}that worthless men\fnote{\fbackref{13:13} Lit. \fbib{that men, sons of Belial}} have come from among you to entice those who live in the towns. They may say, `Let's go and serve other gods that you haven't known.' \v{14}You must thoroughly investigate and inquire if it is true that this detestable thing exists among you. If it is so,\fnote{\fbackref{13:14} The Heb. lacks \fbib{if it is so}} \v{15}then put the inhabitants of the town to death by the sword. Devote everything in it to divine destruction---even its livestock---by the sword. \v{16}Gather whatever you've taken as spoils at the public square of the town, then burn the town, along with whatever you've taken, as an offering to the \divine{Lord} your God. It will remain a permanent mound of ruins, never to be rebuilt again. \v{17}Moreover, you must never take any item from those condemned things, so the \divine{Lord} may yet relent from his burning anger and extend compassion, have mercy, and cause you to increase in number---as he promised by an oath to your ancestors--- \v{18}if you obey the voice of the \divine{Lord} your God by observing all his commands that I'm commanding you today. Do what is right in the sight of the \divine{Lord} your God.''
\labelchapt{14}
\passage{Refrain from Cutting Yourselves}

\chapt{14}
\v{1}``You are children of the \divine{Lord} your God. You must not lacerate yourselves or shave your foreheads on account of the dead, \v{2}because you are a holy people to the \divine{Lord} your God, and the \divine{Lord} chose to make you his precious possession from among all the nations\fnote{\fbackref{14:2} Lit. \fbib{peoples}} of the earth.''
\passage{Refrain from Unclean Food}

\v{3}``You must not eat any detestable food. \v{4}These are the animals that you may eat: ox, sheep, goat, \v{5}deer, gazelle, roebuck, wild goat, ibex, antelope, and mountain sheep. \v{6}You may eat every animal with a divided hoof---those with split cloven hooves---that chews the cud. \v{7}However, you must not eat these animals that chew the cud or have a divided hoof: the camel, hare, and rock badger. Even though they chew the cud, their hooves are not divided. Therefore, they are unclean for you. \v{8}And also the pig, because even though its hoof is divided, it does not chew the cud. It is therefore unclean for you. You must not eat their meat or even touch their carcasses.

\v{9}``You may choose to eat from these creatures in the water: you may eat anything with fin and scale, \v{10}but you may not eat anything without fin and scale, since it is unclean to you.

\v{11}``You may eat all clean birds, \v{12}but you must not eat any of these: the eagle, vulture, osprey, \v{13}buzzard, any kind of kite, \v{14}any kind of raven, \v{15}the ostrich, night hawk, seagull, any kind of falcon, \v{16}the little owl, great owl, horned owl, \v{17}pelican, carrion vulture, cormorant, \v{18}stork, any kind of heron, the hoopoe, and the bat. \v{19}Any winged, swarming insect is unclean to you---they are not to be eaten. \v{20}You may eat every bird that is clean.

\v{21}``You must not eat any carcass,\fnote{\fbackref{14:21} I.e., that dies of itself or in the wild} but you may give it to the alien in your cities\fnote{\fbackref{14:21} Lit. \fbib{in your gates}} so he may either consume it or sell it to a foreigner, since you are a people that is holy to the \divine{Lord} your God.

``You must not cook a young goat in its mother's milk.''
\passage{Remember to Tithe}

\v{22}``Be sure to tithe annually from everything you plant that yields a harvest in the field. \v{23}Then in the presence of the \divine{Lord} your God, in the place where he'll choose to establish his name, you may consume the tithe of your grain, your new wine, your oil, and the firstborn of your livestock and flock, so that you'll learn to revere the \divine{Lord} your God all your life. \v{24}Now the way may be distant from you, so that you are unable to transport your tithe because you have been blessed by the \divine{Lord} your God and the place where the \divine{Lord} your God chooses to establish his name may be distant from you. \v{25}In that case, convert it into cash, secure the money,\fnote{\fbackref{14:25} Lit. \fbib{bind the money with your hand}} and then bring it to the place where the \divine{Lord} will choose. \v{26}You may spend the money to your heart's content to buy livestock, flocks, wine, strong drink, and whatever you desire. You and your household may eat there and rejoice in the presence of the \divine{Lord} your God.''
\passage{The Levitical Tithe}

\v{27}``But you must not forget the descendant of Levi in your town,\fnote{\fbackref{14:27} Lit. \fbib{gates}} because there is no tribal allotment\fnote{\fbackref{14:27} Lit. \fbib{a portion and inheritance}} for him as there is for you. \v{28}Every third year, bring all the tithes of your produce of that year and store them in your cities \v{29}so the descendants of Levi---who have no tribal allotment as you do---foreigners, orphans, and widows who live in your cities may come, eat, and be satisfied. That way, the \divine{Lord} your God will bless you in everything you do.''\fnote{\fbackref{14:29}. Lit. \fbib{in the work of your hand that you do}}
\labelchapt{15}
\passage{The \divine{Lord}'s Remission}

\chapt{15}
\v{1}``You must cancel your debts at the end of every seventh year. \v{2}This is the way to conduct remission: every creditor must cancel the loan that his friend borrowed, and he must not pressure his friend or brother to repay it,\fnote{\fbackref{15:2} The Heb. lacks \fbib{to repay it}} because remission to the \divine{Lord} will be proclaimed. \v{3}You may exact payment from a foreigner, but cancel whatever your brother owes you. \v{4}Moreover, there will be no poor person among you, for the \divine{Lord} will surely bless you in the land that he\fnote{\fbackref{15:4} Lit. \fbib{the \divine{Lord}}} is about to give you to possess. \v{5}Only be certain to obey the voice of the \divine{Lord} your God. Carefully observe all of these commands that I'm commanding you today, \v{6}because the \divine{Lord} your God will bless you just as he promised. You are to lend to many nations, but not to borrow. Also, you will rule over many nations, but they will not rule over you.''
\passage{Care for the Poor}

\v{7}``If there should be a poor man among your relatives\fnote{\fbackref{15:7} Lit. \fbib{brothers}} in one of the cities of the land that the \divine{Lord} your God is about to give you, don't be hard-hearted or tight-fisted toward your poor relative.\fnote{\fbackref{15:7} Lit. \fbib{brother}} \v{8}Instead, be sure to open your hand to him and lend him enough to lessen his need. \v{9}Be careful not to think this wicked thought to yourselves: `The seventh year, the year of remission, is drawing near{\ldots}' and you show ill will\fnote{\fbackref{15:9} Lit. \fbib{and your eyes are evil}} toward your poor relative\fnote{\fbackref{15:9} Lit. \fbib{brother}} and not give to him. He may then call to the \divine{Lord} on account of you, and you will be guilty of sin. \v{10}You must certainly give to him and not feel regret for doing so.\fnote{\fbackref{15:10} Lit. \fbib{for giving to him}} Because of this, the \divine{Lord} your God will bless all your works and everything you do. \v{11}Since poor people won't cease to exist in the land, I'm commanding you: Be sure to display generosity\fnote{\fbackref{15:11} Lit. \fbib{to open your hand}} to your poor and needy relatives in your land.''
\passage{Releasing Slaves}

\v{12}``When a fellow Hebrew male or female slave is sold to you and serves you for six years, then in the seventh year you are to set them\fnote{\fbackref{15:12} Lit. \fbib{him}; and so throughout the chapter} free. \v{13}But when you set them free, don't send them away empty-handed. \v{14}Provide for them liberally from your flock, threshing floor, and wine vat. As the \divine{Lord} your God has blessed you, so give to them. \v{15}Don't ever forget that you were a slave in the land of Egypt, yet the \divine{Lord} your God redeemed you. Therefore, I'm giving you these commands today.

\v{16}``If that slave\fnote{\fbackref{15:16} Lit. \fbib{he}} should say to you, `I won't leave you,' because he loves you and your household, and it was good for him to be with you, \v{17}then take an awl and pierce through his earlobe into the door. He then will be your slave forever. You are to do the same for your female slaves. \v{18}Don't view this as a hardship for yourself when you set him free, for he will have served you for six years---twice the time of a paid worker. Then the \divine{Lord} will bless you in all that you do.''
\passage{Offering the Firstborn Male Animals}

\v{19}``Set apart for the \divine{Lord} your God every firstborn male among your herd and flock. You are not to put the firstborn of your ox to work or shear the firstborn of your flock. \v{20}Instead, in the presence of the \divine{Lord} your God, you and your household are to eat them every year at the place the \divine{Lord} will choose. \v{21}If it has a blemish---lameness, blindness, or any kind of defect---you must not sacrifice it to the \divine{Lord} your God. \v{22}In your cities,\fnote{\fbackref{15:22} Lit. \fbib{gates}} both the unclean and the clean together are to eat it together,\fnote{\fbackref{15:22} Or \fbib{completely}} as the gazelle and the deer, \v{23}but you are not to eat its blood. Pour it on the ground like water.''
\labelchapt{16}
\passage{Celebrate the Passover}

\chapt{16}
\v{1}``Observe the month of Abib, keeping the Passover to the \divine{Lord} your God, because the \divine{Lord} your God brought you out of Egypt during the night in the month of Abib. \v{2}Then sacrifice sheep and cattle for the Passover to the \divine{Lord} your God at the place where the \divine{Lord} your God will choose to establish his name. \v{3}You must not eat any yeast with it. Instead, for seven days eat bread without yeast---the bread of affliction---because you left the land of Egypt in haste. Remember the day you went out of the land of Egypt for the rest of your lives. \v{4}Yeast is not to be seen in any of your territories for seven days. The meat is not to remain from the evening of the first day until morning.

\v{5}``You must not sacrifice the Passover in just any of your cities\fnote{\fbackref{16:5} Lit. \fbib{gates}} that the \divine{Lord} your God is about to give you. \v{6}Instead, you are to sacrifice the Passover in the evening at dusk---at the time of day you left Egypt---at the place where your God will choose to establish his name. \v{7}Boil and eat the Passover meal\fnote{\fbackref{16:7} The Heb. lacks \fbib{the Passover meal}} at the place that the \divine{Lord} your God will choose. In the morning you may go back to your tents. \v{8}Eat bread without yeast for six days. Then on the seventh day, hold an assembly to the \divine{Lord} your God. Don't do any work.''
\passage{Celebrate the Festival of Weeks}

\v{9}``Count off seven weeks from when the sickle is first put to standing grain. \v{10}Then observe the Festival of Weeks in the presence of the \divine{Lord} your God by giving your tribute and the freewill offering of your hands in proportion to the manner in which the \divine{Lord} your God blessed you. \v{11}Rejoice in the presence of the \divine{Lord} your God with your son, daughter, male and female slaves, the descendant of Levi who is in your city,\fnote{\fbackref{16:11} Lit. \fbib{gate}} the stranger, the orphan, and the widow among you, at the place where the \divine{Lord} your God will choose to establish his name. \v{12}Remember that you were slaves in Egypt, so keep and observe these statutes.''
\passage{Celebrate the Festival of Tents}

\v{13}``Celebrate the Festival of Tents\fnote{\fbackref{16:13} Or \fbib{Tents}} for seven days after you harvest from your threshing floor and your wine press. \v{14}Rejoice in your festival---you, your son, your daughter, your male and female slaves, and the descendants of Levi, foreigners, orphans, and widows, who live in your cities.\fnote{\fbackref{16:14} Lit. \fbib{gates}} \v{15}For seven days you are to celebrate in the presence of the \divine{Lord} your God at the place where the \divine{Lord} will choose, because the \divine{Lord} your God will bless you in all your harvest and in everything you do, and your joy will be complete.

\v{16}``Every male must appear in the presence of the \divine{Lord} your God three times a year at the place where he will choose: for the Festival of Unleavened Bread, the Festival of Seven Weeks, and the Festival of Tents.\fnote{\fbackref{16:16} Or \fbib{Tents}} He must not appear in the \divine{Lord}'s presence empty-handed, \v{17}but each one must appear\fnote{\fbackref{16:17} The Heb. lacks \fbib{must appear}} with his own gift, proportional to the blessing that the \divine{Lord} your God has given you.''
\passage{Pursue Justice}

\v{18}``Appoint judges and civil servants according to your tribes in all your cities\fnote{\fbackref{16:18} Lit. \fbib{gates}} that the \divine{Lord} your God is about to give you, so they may judge the people impartially.\fnote{\fbackref{16:18} Lit. \fbib{people with righteous judgment}} \v{19}You must not twist justice, show favoritism, or take bribes, because a bribe blinds the eyes of the wise and subverts the speech of the righteous. \v{20}You are to pursue justice---and only justice---so you may live and possess the land that the \divine{Lord} your God is about to give you.''
\passage{Prohibited Practices}

\v{21}``You are not to set up a sacred pole\fnote{\fbackref{16:21} Lit. \fbib{Asherah}; i.e. a cultic pillar} beside the altar of the \divine{Lord} your God that you will build. \v{22}Furthermore, you are not to erect for yourselves a sacred stone pillar, because the \divine{Lord} your God detests these things.\chapt{17}
\v{1}You are not to sacrifice to the \divine{Lord} your God an ox or a sheep that has a defect or any flaw in it, because that is detestable to the \divine{Lord} your God.''
\labelchapt{17}
\passage{Death to the Idolater}

\v{2}``You may discover that a man or woman living in one of your cities that the \divine{Lord} your God is about to give you has done evil in the eyes of the \divine{Lord} your God by transgressing his covenant. \v{3}He may be following and serving other gods by bowing down to them---that is, to the sun, the moon, or to any of the heavenly host\fnote{\fbackref{17:3} Or \fbib{any of the stars or planets}, if referring to astronomical bodies; or \fbib{supernatural beings}, if referring to fallen or unfallen angelic armies} (something I did not command). \v{4}When it is reported to you or you hear of it, you are to investigate it thoroughly. When the truth has been established that this detestable thing has been done in Israel, \v{5}summon the man or the woman who did this evil thing to your city gates, and then stone the man or the woman to death. \v{6}Based on the testimony\fnote{\fbackref{17:6} Lit. \fbib{mouth}} of two or three witnesses, they must surely die, but they are not to die based on the testimony of one person. \v{7}Let the witnesses\fnote{\fbackref{17:7} Lit. \fbib{the hands of the witnesses}} be the first to begin executing them, then the rest of\fnote{\fbackref{17:7} Lit. \fbib{the hand of all}} the people are to follow. By doing this you will purge evil from among you.''
\passage{Deciding Difficult Cases}

\v{8}``If a case is too difficult for you to decide with respect to bloodshed,\fnote{\fbackref{17:8} Lit. \fbib{blood versus blood}} civil claims,\fnote{\fbackref{17:8} Lit. \fbib{justice versus justice}} assault and battery,\fnote{\fbackref{17:8} Lit. \fbib{wound versus wound}} or other matters of dispute within your courts,\fnote{\fbackref{17:8} Lit. \fbib{gates}} bring\fnote{\fbackref{17:8} Lit. \fbib{stand and go up}} it to the place that the \divine{Lord} your God will choose. \v{9}Present the case\fnote{\fbackref{17:9} The Heb. lacks \fbib{Present the case}} to the Levitical priest or the judge at that time. When you have inquired and they have announced the verdict, \v{10}carry out the verdict that was declared to you at the place that the \divine{Lord} will choose. Carefully observe all of their instructions to you \v{11}in accordance with what the Law says and in accordance with the verdict that will be handed to you. You must not deviate from the verdict that they declare to you either to the right or to the left. \v{12}If a man presumptuously disregards the priest who is serving the \divine{Lord} your God there, or the judge, that person must die so you will purge evil from Israel. \v{13}Then all the people who hear will be afraid and will not act presumptuously again.''
\passage{Duties of the Future King}

\v{14}``When you have come to the land that the \divine{Lord} your God is about to give you, and you have taken possession of it and have settled in it, then you will say, `I will appoint a king over me like all the nations around me.' \v{15}You will certainly set a king over you, whom the \divine{Lord} your God will choose from among your relatives, but you must not place a foreign king over you who is not from your relatives. \v{16}He must not amass horses for himself or cause the people to return to Egypt to obtain more horses, because the \divine{Lord} said you must never return that way again. \v{17}Also, he must not accumulate wives for himself (otherwise, his affection will become diverted), nor accumulate for himself excessive quantities of\fnote{\fbackref{17:17} The Heb. lacks \fbib{quantities of}} silver and gold. \v{18}When he occupies his royal throne, he must make a copy of this Law for himself from a scroll used by the Levitical priests. \v{19}It is to remain with him the rest of his life so he may learn to fear the \divine{Lord} his God and observe all the words of this Law and these statutes, in order to fulfill them. \v{20}He is not to exalt himself over his relatives, nor turn aside from the commandment---neither to the right nor to the left---so that he and his sons may reign long in Israel.''
\labelchapt{18}
\passage{Provision for the Descendants of Levi}

\chapt{18}
\v{1}``The Levitical priests---the whole tribe of Levi---will not have a portion or an inheritance within Israel. Instead, they will eat the burnt offerings of the \divine{Lord}, because that is their inheritance. \v{2}But they will not have an inheritance among their relatives, because the \divine{Lord} alone is their inheritance---as he promised them.''
\passage{Provision for the Priests}

\v{3}``A portion of what the people offer in sacrifice, whether cattle or sheep, is to be due the priests. They must set aside the shoulder, jowls, and stomach for the priest. \v{4}Give them the first gatherings of your grain, wine, and oil, as well as wool from the shearing of your flock. \v{5}For the \divine{Lord} your God has chosen them and their descendants\fnote{\fbackref{18:5} Lit. \fbib{sons}} from among your tribes to stand and serve in the name of the \divine{Lord} all their lives.''\fnote{\fbackref{18:5} Lit. \fbib{days}}
\passage{Provision for the Itinerant Levite}

\v{6}``Any descendant of Levi who wishes to do so may come from any city or part of Israel where he resides to the place that the \divine{Lord} will choose. \v{7}There he may serve in the name of the \divine{Lord} his God. Like his fellow descendants of Levi who stand there in the \divine{Lord}'s presence, \v{8}he may eat the same share as they do regardless of what he receives from his ancestral estate.''
\passage{Detestable Practices}

\v{9}``When you enter the land that the \divine{Lord} your God is about to give you, don't learn the detestable practices of those nations there. \v{10}There must never be found among you anyone who sacrifices\fnote{\fbackref{18:10} Lit. \fbib{passes}} his son or daughter in fire, practices divination, interprets omens, practices sorcery, \v{11}casts spells, or who is a medium, an occultist, or a necromancer. \v{12}Whoever practices these things is detestable to the \divine{Lord}, and the \divine{Lord} your God will expel them before you because of these things. \v{13}You must be completely faithful to the \divine{Lord} your God, \v{14}because those nations that you are about to dispossess listen to those who practice witchcraft and divination. But the \divine{Lord} does not allow you to act this way.''
\passage{Discerning the True Prophet}

\v{15}``The \divine{Lord} your God will raise up a prophet like me for you from among your relatives. You must listen to him, \v{16}because this is what you asked from the \divine{Lord} your God at Horeb when you were assembled together: `Don't let us\fnote{\fbackref{18:16} Lit. \fbib{me}} hear the voice of the \divine{Lord} our God again, or even see this great fire---otherwise, we\fnote{\fbackref{18:16} Lit. \fbib{I}} will die.'

\v{17}``Then the \divine{Lord} told me: `What they have suggested is good. \v{18}I will raise up a prophet like you from among their relatives, and I will place my words in his mouth so that he may expound everything that I have commanded to them. \v{19}But if someone will not listen to those words that the prophet\fnote{\fbackref{18:19} Lit. \fbib{he}} speaks in my name, I will hold him accountable. \v{20}Even then, if the prophet speaks presumptuously in my name, which I didn't authorize him to speak, or if he speaks in the name of other gods, that prophet must die.' \v{21}Now you may ask yourselves, `How will we be able to discern that the \divine{Lord} has not spoken?' \v{22}Whenever a prophet speaks in the name of the \divine{Lord} and the oracle does not come about or the word is not fulfilled, then the \divine{Lord} has not spoken it. The prophet will have spoken presumptuously, so you need not fear him.''
\labelchapt{19}
\passage{Cities of Refuge}

\chapt{19}
\v{1}``When the \divine{Lord} your God destroys those nations whose lands he\fnote{\fbackref{19:1} Lit. \fbib{the \divine{Lord} your God}} is about to give you, you must dispossess them and live in their cities and houses. \v{2}You must reserve\fnote{\fbackref{19:2} Or \fbib{set apart}} three cities within the land that the \divine{Lord} your God is about to give you to possess. \v{3}Build roads throughout the land that the \divine{Lord} your God is providing as an inheritance, and then divide it into three districts so that any killer may flee there.

\v{4}``Now this is the situation for any killer who flees there to live: suppose he strikes his friend unwittingly, not having hated him previously. \v{5}For instance,\fnote{\fbackref{19:5} The Heb. lacks \fbib{for instance}} he may have accompanied his friend to go to a forest to cut trees. Then he swung his axe to cut some wood, but the ax head flew off the handle\fnote{\fbackref{19:5} Lit. \fbib{tree}} and hit\fnote{\fbackref{19:5} Lit. \fbib{found}} his friend, so that he died. The killer\fnote{\fbackref{19:5} The Heb. lacks \fbib{the killer}} may flee to one of these cities to live. \v{6}Since the distance may be great, the angry avenger may overtake the killer he is pursuing and kill him, in which case there will be no justice in his death, because he did not hate his friend\fnote{\fbackref{19:6} Lit. \fbib{hate him}} previously. \v{7}Therefore I am commanding you to reserve\fnote{\fbackref{19:7} Or \fbib{set apart}} three cities.''
\passage{Increase the Cities of Refuge}

\v{8}``Now if the \divine{Lord} enlarges your territories just as he promised your ancestors and gives you all the land that he promised,\fnote{\fbackref{19:8} Lit. \fbib{promised to give your ancestors}} \v{9}and if you are careful to observe all these commands that I am commanding you today---to love the \divine{Lord} your God and to walk daily in his ways---then add three more cities in addition to these three cities. \v{10}You must not shed innocent blood on your land that the \divine{Lord} your God is about to give you as an inheritance. Otherwise, you'll be guilty of murder.''
\passage{Refuse Cold-Blooded Murderers}

\v{11}``However, if a person hates his neighbor, lies in wait for him, rises up against him, and attacks him so that he dies, and then he flees to one of those cities, \v{12}then the elders of his own city are to send for him, remove him from there, and deliver him to the related avenger for execution. \v{13}Have no pity on him, but totally purge the shedding of innocent blood from Israel so that life may go well with you.''
\passage{Boundary Markers}

\v{14}``When you inherit the land that the \divine{Lord} your God is about to give you, don't move your neighbor's boundary marker from where it was placed long ago.''
\passage{Laws about Witnesses}

\v{15}``The testimony of one person alone is not to suffice to convict anyone of any iniquity, sin, or guilt. But the matter will stand on the testimony of two or three witnesses. \v{16}When a malicious witness takes the stand\fnote{\fbackref{19:16} Lit. \fbib{witness stands}} against a man and accuses him, \v{17}then both must stand with their dispute in the \divine{Lord}'s presence, the priests, and the judges at that time. \v{18}The judges will investigate thoroughly. If the false witness lies in testifying against his relative, \v{19}do to him just as he intended to do to his relative. By doing this you will purge evil from your midst. \v{20}When others hear of this, they will be afraid and will not do such an evil deed again in your midst. \v{21}Your eyes must not show pity---life for life, eye for eye, tooth for tooth, hand for hand, and foot for foot.''
\labelchapt{20}
\passage{Rules of War}

\chapt{20}
\v{1}``When you go to war against your enemies and observe more horses, chariots, and soldiers\fnote{\fbackref{20:1} Lit. \fbib{people}} than you have, don't be afraid of them, for the \divine{Lord} your God who brought you out of the land of Egypt is with you. \v{2}As you draw near for battle, let the priest approach and speak to the army.\fnote{\fbackref{20:2} Lit. \fbib{people}; and so throughout the chapter} \v{3}He will say to them, `Listen, Israel! You're about to go into battle today against your enemies. Don't be faint-hearted. Don't be afraid, don't panic, and don't be terrified to face them. \v{4}For the \divine{Lord} your God will be with you, fighting on your behalf against your enemies in order to grant you victory.'

\v{5}``Furthermore, let the officials ask the army, `Is there a man here\fnote{\fbackref{20:5} The Heb. lacks \fbib{here}} who has built a new house but has not yet dedicated it? Let him go back home. Otherwise, he may die in battle and another man dedicate it. \v{6}And is there a man here\fnote{\fbackref{20:6} The Heb. lacks \fbib{here}} who has planted a vineyard and not yet benefited from it? Let him go home. Otherwise, he may die in battle and another man use it. \v{7}And is there a man here\fnote{\fbackref{20:7} The Heb. lacks \fbib{here}} who is engaged to a woman and has not yet married her? Let him go back home. Otherwise, he may die in battle and another man marry her.'

\v{8}``Let the officials also speak to the army, `Is there a man here\fnote{\fbackref{20:8} The Heb. lacks \fbib{here}} who is afraid and faint-hearted? Let him go back home. Otherwise, he may demoralize his fellow soldier.'\fnote{\fbackref{20:8} Lit. \fbib{his brother}}

\v{9}``When the officials have finished speaking to the army, they must appoint officers to lead the troops.''
\passage{Rules of Peace}

\v{10}``When you approach a city to wage war against it, extend terms of peace. \v{11}If it agrees to peace and welcomes you, then all the people found in it will serve you as forced laborers. \v{12}But if they refuse to make peace with you and instead choose war, then attack it. \v{13}The \divine{Lord} your God will deliver it into your control, and you must execute every male. \v{14}The women, children, all the livestock in the city, and all of the spoil and plunder will belong to you. Appropriate the spoil of your enemies, which the \divine{Lord} your God will give you. \v{15}Do this to all the cities that are distant from you---that is, to those cities that are not in neighboring nations.''
\passage{Destruction of the Canaanites}

\v{16}``You are not to leave even one person alive in the cities of these nations that the \divine{Lord} your God is about to give you as an inheritance. \v{17}You must completely destroy the Hittites, the Amorites, the Canaanites, the Perizzites, the Hivites, and the Jebusites, just as the \divine{Lord} your God commanded you, \v{18}so they won't teach you to do all the detestable things that they do for their gods. If you do what they teach you, you will sin against the \divine{Lord} your God.''
\passage{Preservation of Fruit Trees}

\v{19}``When you attack a city and have to fight against it for many days, don't destroy its trees by cutting them down with an ax. You may eat from them, but you must not cut them down. Are the trees of the field human beings, that you would come and attack them? \v{20}However, you may cut down the trees whose fruit\fnote{\fbackref{20:20} The Heb. lacks \fbib{whose fruit}} you know isn't edible, in order to build siege works against the city that waged war with you, until it falls.''
\labelchapt{21}
\passage{Atonement for Unsolved Murder}

\chapt{21}
\v{1}``If a murder victim is found fallen in the open country of the land that the \divine{Lord} your God is about to give you to possess, and it is not known who killed him, \v{2}then let your elders and judges go out and measure the distance from the dead body to the neighboring cities. \v{3}Then the elders of the city nearest the body are to take a heifer that hasn't been put to work or hasn't pulled a yoke \v{4}and\fnote{\fbackref{21:4} Lit. \fbib{The elders of the city}} are to lead the heifer to a flowing stream in a valley that has never been tilled or planted. They are to break the heifer's neck there. \v{5}Then the priests of the sons of Levi are to step forward, because the \divine{Lord} your God chose them to serve and pronounce blessings in his name.\fnote{\fbackref{21:5} Lit. \fbib{in the name of the \divine{Lord} your God}} Every case of dispute and assault is to be subject to their ruling. \v{6}All the elders of the city nearest the dead body are to wash their hands over the heifer whose neck was broken in the valley, \v{7}and they are to make this declaration: `Our hands didn't shed this blood, nor were we witnesses to the crime. \v{8}Make atonement for your people Israel, whom you have redeemed, \divine{Lord}, and don't charge the blood of an innocent man against them.'\fnote{\fbackref{21:8} Lit. \fbib{against your people Israel}} Then the blood that has been shed will be atoned for. \v{9}This is how you will remove the guilt of innocent blood from among you, for you must do what is right in the sight of the \divine{Lord}.''
\passage{Marriage to a War Captive}

\v{10}``If you go to battle against your enemies, and the \divine{Lord} your God delivers them into your control, you may take some prisoners captive. \v{11}If you see among the prisoners a beautiful woman and you desire her, then you may take her as your wife. \v{12}Bring her to your house, but shave her head and trim her nails. \v{13}Remove her prisoner's clothing and let her remain for a month in your house, mourning her parents. After that, you may\fnote{\fbackref{21:13} Lit. \fbib{may go in to her,}} become her husband and she is to become your wife. \v{14}If you aren't pleased with her and you send her away, you must not sell her for money or mistreat her, since you will have dishonored her.''
\passage{Preferential Treatment Prohibited}

\v{15}``If a man has two wives where one is loved but the other is unloved, and both\fnote{\fbackref{21:15} Lit. \fbib{the one who is loved and who is not loved}} of them bear him sons, but the firstborn is the son of the unloved wife, \v{16}then when he bequeaths his possessions to his sons, he must not give preference to the firstborn of the beloved wife over the firstborn of the unloved wife. \v{17}Instead, he must acknowledge the firstborn of the unloved wife by giving him double of everything he owns, because he is really the first fruit of his father's\fnote{\fbackref{21:17} The Heb. lacks \fbib{father's}} strength. The right of the firstborn belongs to him.''
\passage{Death to a Rebellious Son}

\v{18}``If a man has a stubborn son who does not obey his parents,\fnote{\fbackref{21:18} Lit. \fbib{obey the voice of his father or the voice of his mother}} and although they try to discipline him, he still refuses to pay attention to them, \v{19}then his parents\fnote{\fbackref{21:19} Lit. \fbib{his father and his mother}} are to seize him and bring him before the elders at the gate of his city. \v{20}Then they are to declare to the elders of their city: `Our son is stubborn and rebellious. He does not obey us. He lives wildly and is a drunkard.' \v{21}Then all the men of his city are to stone him with stones so that he dies. This is how you will remove this evil from among you. Then all Israel will hear of it and will be afraid.''
\passage{Burial of the Executed}

\v{22}``If a man is guilty of a capital offense, is executed, and then is impaled on a tree, \v{23}his body must not remain overnight on the tree. You must bury him that same day, because cursed of God is the one who has been hanged on a tree. Don't defile your land that the \divine{Lord} is about to give you as your inheritance.''
\labelchapt{22}
\passage{Hospitality to Neighbors}

\chapt{22}
\v{1}``When you see the ox or sheep of your fellow countryman\fnote{\fbackref{22:1} Lit. \fbib{brother's} and so throughout the chapter} straying, don't go away and leave it. Instead, be sure to return it to him.\fnote{\fbackref{22:1} Lit. \fbib{brother} and so throughout the chapter} \v{2}If your fellow countryman doesn't live near you or you don't know who he is, bring the animal\fnote{\fbackref{22:2} Lit. \fbib{bring it}} to your house and let it remain with you until he\fnote{\fbackref{22:2} Lit. \fbib{brother}} claims it. Then return it to him. \v{3}Do the same for his donkey, his garment, and for anything lost that belongs to your fellow countryman. When you find it, you must not ignore it. \v{4}When you see the donkey or the ox of your fellow countryman fallen on the road, don't ignore them. Instead be sure to help them get up.''
\passage{Miscellaneous Laws}

\v{5}``A woman is not to wear what is appropriate to a man, nor is a man to put on a woman's garment, because anyone who does this is detestable to the \divine{Lord} your God.

\v{6}``When you encounter a bird's nest along the road, whether in a tree or on the ground, and the mother bird is sitting on its chicks\fnote{\fbackref{22:6} Lit. \fbib{on the young}} or eggs, don't take the mother along with its young.\fnote{\fbackref{22:6} Lit. \fbib{sons}} \v{7}You may take the young, but be sure to release the mother, so that life will go well for you and that you may have a long life.

\v{8}``When you build a new house, install a parapet along your roof so that if someone falls from the roof, you won't bring guilt of bloodshed on your house.''
\passage{Principles of Distinction}

\v{9}``Don't plant two kinds of seeds in your vineyard. Otherwise, the entire crop will have to be forfeited, both the seed that you have sown and the produce from it.

\v{10}``Don't plow with an ox and a donkey yoked together.

\v{11}``Don't wear material made from wool and linen mixed together.

\v{12}``Sew tassels for yourself on the four corners of the garment with which you cover yourself.''
\passage{Integrity in Marriage}

\v{13}``Suppose a man marries a wife, but after having sexual relations with her, he despises her, \v{14}invents charges against her, and defames her by saying, `I have married this woman, but when I had sexual relations with her I found that she wasn't a virgin.' \v{15}Then the father of the young lady, along with her mother, is to bring evidence of the young lady's virginity to the elders at the gate. \v{16}The father of the young lady is to then say to the elders: `I have given my daughter to this man as a wife, but he despises her. \v{17}Now look, he has invented charges against her by saying, ``I haven't found your daughter to be a virgin.'' But here is the proof of my daughter's virginity.' Then they are to spread the cloth before the elders of the city. \v{18}The elders of that city will then take the man, punish him, \v{19}fine him 100 shekels of silver, and then give them to the young lady's father, because he defamed a virgin of Israel. She is to remain his wife and he can't divorce her as long as he lives. \v{20}But if this charge is true, and the evidence of the young lady's virginity wasn't found, \v{21}they are to bring her to the door of her father's house. Then the men of the city are to stone her with boulders until she dies for doing a detestable thing in Israel---acting like a prostitute while in her father's house. By doing this, you will remove this evil from among you.

\v{22}``If a man is caught having sexual relations with a married woman, then both of them must die---the man who had sex with the woman and the woman herself---so that this evil will be removed from Israel.

\v{23}``If a man meets a young virgin lady in the city who is engaged to be married and has sexual relations with her, \v{24}then the two must be brought to the city gate and there they must be stoned to death---the girl because she was in a city but did not cry out for help, and the man who abused a woman who was engaged to another man. By doing this you are to remove this evil from among you.

\v{25}``If a man meets a girl in the country who is engaged to be married and then rapes\fnote{\fbackref{22:25} Lit. \fbib{overwhelms}} her, the man alone---the one who had sexual relations with her---must die. \v{26}As for the young lady, don't do anything to her. The young lady did nothing worthy of death. This case is similar to when a man attacks his countryman and kills him. \v{27}Since he found her in the country, the engaged girl may have cried out, but there was no one to rescue her.

\v{28}``However, if a man meets a girl who isn't engaged to be married, and he seizes her, rapes her, and is later found out, \v{29}then the man who raped her must give 50 shekels of silver to the girl's father. Furthermore, he must marry her. Because he violated her, he is to not divorce her as long as he lives.

\v{30}\fnote{\fbackref{22:30} This v. is 23:1 in MT}``A man must not marry his father's wife, so that he will not dishonor his father's memory.''\fnote{\fbackref{22:30} Lit. \fbib{wing}; or \fbib{skirt}}
\labelchapt{23}
\passage{Qualifications for Assembling}

\chapt{23}
\v{1}\fnote{\fbackref{23:1} This v. is 2 in MT, and so throughout the chapter.}``No man whose testicles have been crushed\fnote{\fbackref{23:1} Or \fbib{wounded}} or whose penis has been cut off may participate in the assembly of the \divine{Lord}. \v{2}Furthermore, no one born due to an illicit sexual relationship may participate in the assembly of the \divine{Lord}, including his descendants to the tenth generation. \v{3}``No Ammonite or Moabite may participate in the assembly of the \divine{Lord}, and none of their descendants is to be admitted to the assembly of the \divine{Lord}, to the tenth generation, \v{4}because they didn't come to meet you with food and water along the way as you were coming out of Egypt. Instead, they hired Beor's son Balaam from Pethor in Aram-naharaim\fnote{\fbackref{23:4} I.e. Mesopotamia} to curse you. \v{5}However, the \divine{Lord} your God didn't listen to Balaam. The \divine{Lord} your God turned Balaam's\fnote{\fbackref{23:5} Lit. \fbib{his}} curse into a blessing, because the \divine{Lord} your God loves you. \v{6}Don't seek a peace treaty with them as long as you live. \v{7}Don't detest Edomites, since they are related to you. Don't detest Egyptians, either, because you were strangers in their land. \v{8}Their grandchildren\fnote{\fbackref{23:8} Lit. \fbib{Children born in the third generation}} may participate in the assembly of the \divine{Lord}.''
\passage{Community Sanitation}

\v{9}``When you are encamped for battle against your enemies, be on guard against every form of impropriety. \v{10}If someone among you becomes unclean due to nocturnal emissions, he must leave the camp and stay outside. \v{11}As evening approaches he must wash himself with water. Then at sunset, he may return to the camp.

\v{12}``Choose a place outside the camp for a latrine. \v{13}Include a spade among your equipment so that when you squat to relieve yourself, you can dig a hole and then cover your excrement. \v{14}For the \divine{Lord} your God is on the move within your camp to deliver you and to hand your enemies over to you. Therefore your camp must be holy so that he will not see anything indecent among you and turn away from you.''
\passage{Treatment of Slaves}

\v{15}``Don't hand over a slave who escaped from his master when he runs to you. \v{16}Let him live among you wherever he chooses in any of your cities that he likes. Don't mistreat him.''
\passage{Cultic Prostitution Prohibited}

\v{17}``There are to be no cultic prostitutes among the daughters or the sons of Israel. \v{18}Don't bring the earnings of a female prostitute nor the income of a male prostitute into the house of the \divine{Lord} your God as payment for any vow. Both of these are detestable to the \divine{Lord} your God.''
\passage{Fair Dealings}

\v{19}``Don't charge interest to your relatives, whether for money, food, or for anything that has been loaned at interest. \v{20}You may charge interest to a foreigner, but don't charge interest to your relatives, so the \divine{Lord} your God may bless you in everything you undertake in the land that you are about to enter and possess.

\v{21}``When you make a vow to the \divine{Lord} your God, don't delay paying it, because the \divine{Lord} your God will certainly demand payment from you, and then you will be guilty of sin. \v{22}But if you refrain from making a vow, then you won't be guilty. \v{23}Be sure you do whatever you promise, because you have given your word voluntarily to the \divine{Lord} your God.

\v{24}``When you enter your countrymen's vineyard, you may eat the grapes to your satisfaction, but don't take any in a basket. \v{25}When you enter your countrymen's grain fields, you may pluck the grain with your hand, but don't put a sickle to his standing grain.''
\labelchapt{24}
\passage{Various Laws}

\chapt{24}
\v{1}``If a man chooses to enter into marriage with a woman, but she finds herself displeasing to him because he has found something objectionable\fnote{\fbackref{24:1} Lit. \fbib{naked}; i.e. \fbib{indecent}} about her, he must draw up divorce papers, hand them to her, and then send her out of his house. \v{2}If she goes out from his house, becomes the wife of another man, \v{3}and this second husband\fnote{\fbackref{24:3} Lit. \fbib{this other man}} dislikes her, he, also, must draw up divorce papers, hand them to her, and then send her away from his house. Should the second husband die, \v{4}her first husband who married her and divorced her earlier must not remarry her,\fnote{\fbackref{24:4} Lit. \fbib{not take her to live with him as wife}} because she was defiled, since this is detestable to the \divine{Lord}. Don't defile the land that the \divine{Lord} your God is about to give you as a possession.

\v{5}``When a man is newly married, he must not be sent out to war or have a related duty placed on him. Let him stay home for one year and be happy with his wife whom he has married.

\v{6}``Don't take a pair of millstones, especially the upper millstone, as collateral for a loan, because this means taking a man's\fnote{\fbackref{24:6} Lit. \fbib{taking his}} livelihood.

\v{7}``If a man is found kidnapping his relative, a fellow Israeli, and mistreats or sells him, that kidnapper must die. By doing this, you will remove this evil from among you.

\v{8}``In cases of leprosy, be very careful to observe exactly what the Levitical priests instructed you. Carefully follow what I have commanded them. \v{9}Remember what the \divine{Lord} your God did to Miriam along the way as you were coming out of Egypt.''
\passage{Respecting the Poor}

\v{10}``When you loan something to your neighbor, don't enter his house to seize what he offered as collateral. \v{11}Stay outside and let the man to whom you made the loan bring it\fnote{\fbackref{24:11} Lit. \fbib{the collateral}} out to you. \v{12}If he is a poor man, don't go to sleep with his collateral in your possession.\fnote{\fbackref{24:12} The Heb. lacks \fbib{in your possession}} \v{13}Be sure to return his garment\fnote{\fbackref{24:13} Lit. \fbib{collateral}} to him at sunset so that he may sleep with it, and he will bless you. It will be a righteous deed in the presence of the \divine{Lord} your God. \v{14}Don't take advantage of a hired person who is poor and needy, whether he's your fellow citizen or a foreigner who lives in your city. \v{15}Pay his wages that same day before the sun sets, because he is poor and his livelihood\fnote{\fbackref{24:15} Lit. \fbib{life}} depends on it. Otherwise, he may cry out to the \divine{Lord} against you, and you will incur guilt.''
\passage{Practicing Justice}

\v{16}``Fathers are not to be put to death on account of their children's sin; nor are children to die on account of their fathers' sin. Each person is to be put to death for his own sin.

\v{17}``Don't deny justice to a foreigner or to an orphan, nor take a widow's garment as collateral for a loan. \v{18}Remember to observe this because you were slaves in Egypt, and the \divine{Lord} your God redeemed you from there. That is why I am commanding you to do this.

\v{19}``When you are reaping in the field, and you overlook a sheaf, don't return to get it. Let it remain for the foreigner, the orphan, or the widow, in order that the \divine{Lord} your God may bless everything you undertake. \v{20}When you harvest the olives from your trees, don't go back to the branches a second time. What remains is for the foreigner, the orphan, or the widow. \v{21}When you harvest the grapes in your vineyard, don't go back a second time. What remains are for the foreigner, the orphan, or the widow. \v{22}Remember to do this because you were slaves in the land of Egypt. That is why I'm commanding you to do this.''
\labelchapt{25}
\passage{Limitations on Punishment}

\chapt{25}
\v{1}``When there is a conflict between individuals, let them come to court to judge the case, decide who is innocent, and condemn the guilty person. \v{2}If the guilty person deserves a beating, the judge will make him lie down and be beaten in his presence with the number of lashes fit for his crime. \v{3}But he must not be beaten more than 40 lashes, because if he receives more than 40 lashes, your brother will be humiliated in your eyes.

\v{4}``Don't muzzle an ox while it is threshing grain.''
\passage{Levirate Marriage}

\v{5}``When two brothers are living together and one of them dies without leaving a son, his widow must not be married outside the family to a foreigner. Instead, the brother-in-law must go to her, take her as his wife, and by doing so perform the duty of a brother-in-law. \v{6}The firstborn whom she will bear will continue the name of the dead brother, so his name will not be erased from Israel. \v{7}But if the man does not want to marry his brother's widow, then she\fnote{\fbackref{25:7} Lit. \fbib{the brother's wife}} must go to the elders at the city gate and declare, `My husband's brother refuses to perform the duty of a brother-in-law in order to preserve the name of his brother in Israel. He is not willing to perform the duty of a brother-in-law.' \v{8}Then the elders of the city are to summon him and speak with him. If he insists on saying, `I don't want to marry her,' \v{9}then she is to approach her brother-in-law in the presence of the elders, remove his sandal, spit in his face, and say in response, `May this be done to the man who does not preserve the lineage\fnote{\fbackref{25:9} Lit. \fbib{house}} of his brother.' \v{10}Then his family name in Israel will be known `as the family of the one whose sandal was removed.'\,''
\passage{Limiting a Wife's Defense}

\v{11}``If two men are fighting together, and the wife of one comes to rescue her husband from the grasp of his assailant, and she reaches out and seizes his genitals, \v{12}you are to cut off her hand. Don't show any pity.''
\passage{Honest Weights}

\v{13}``Don't have different weights in your bag---one heavy and one light. \v{14}Don't have different measuring devices in your house---one large and one small. \v{15}You must have honest weights and measuring devices,\fnote{\fbackref{25:15} Lit. \fbib{and an honest and fair device}} so you may live long in the land that the \divine{Lord} your God is about to give you, \v{16}for anyone who does these things---anyone who deals dishonestly---is detestable to the \divine{Lord} your God.''
\passage{Annihilation of the Amalekites}

\v{17}``Remember what the Amalekites did to you along the road while you were coming out of Egypt, \v{18}how when you were very tired and weary, they lay in wait for you on the road and eliminated everyone who was lagging behind. They had no fear of God. \v{19}Therefore, when the \divine{Lord} your God has given you rest from all your enemies who surround you in the land that he\fnote{\fbackref{25:19} Lit. \fbib{that the \divine{Lord} your God}} is about to give you to possess as an inheritance, you must completely erase the memory of the Amalekites from under heaven. Don't forget!''
\labelchapt{26}
\passage{Gift of the First Produce}

\chapt{26}
\v{1}``When you arrive in the land that the \divine{Lord} your God is about to give you as an inheritance, take possession of it and settle in it. \v{2}Gather all the first produce of the ground that you harvest from your land that the \divine{Lord} your God is about to give you, place it in a basket, and bring it to the place where the \divine{Lord} your God will choose to establish his name. \v{3}Approach the priest who is in charge at that time and say to him, `I acknowledge today to the \divine{Lord} your God that I've arrived in the land that the \divine{Lord} promised our ancestors to give us.' \v{4}Then the priest will take the basket from you\fnote{\fbackref{26:4} Lit. \fbib{your hand}} and place it in front of the altar of the \divine{Lord} your God. \v{5}Then you are to affirm and declare in the presence of the \divine{Lord} your God:

\begin{poetry}
\poeml `A wandering Aramean was my ancestor, who went down to Egypt and traveled there with very few family members,\fnote{\fbackref{26:5} The Heb. lacks \fbib{family members}} yet there he became a great, powerful, and populous nation. \v{6}But the Egyptians oppressed us, afflicted us, and assigned us to hard labor. \v{7}So we cried out to the \divine{Lord} God of our ancestors, and he\fnote{\fbackref{26:7} Lit. \fbib{the \divine{Lord}}} heard our cries and observed our affliction, trouble, and oppression. \v{8}The \divine{Lord} brought us out of Egypt with his awesome power,\fnote{\fbackref{26:8} Lit. \fbib{his mighty hand and outstretched arm}} with great terror, signs, and wonders. \v{9}And then we arrived at this place, and he gave this land to us, flowing with milk and honey. \v{10}Now, look---I brought the first produce of the land that you, \divine{Lord}, have given me.'
\end{poetry}

Then set it in the presence of the \divine{Lord} your God and worship him.\fnote{\fbackref{26:10} Lit. \fbib{worship before the \divine{Lord} your God}} \v{11}Rejoice with the descendants of Levi and the foreigner among you at all the good things that the \divine{Lord} your God has given you and your family.''
\passage{Levitical Tithes}

\v{12}``When you have finished your harvest, reserve the tithe in the third year (the year of the tithe), and give the entire tithe to the descendants of Levi, to the foreigners, to the orphans, and to the widows, so they may eat and be satisfied in your cities. \v{13}Then declare in the presence of the \divine{Lord} your God:

\begin{poetry}
\poeml `I've removed the holy offering from my house and given it to the descendants of Levi, to the foreigners, to the orphans, and to the widows just as you have commanded me. I haven't violated or forgotten your commands. \v{14}I haven't eaten any part of it while mourning, nor removed any part of it while unclean, nor offered any of it to the dead. I've obeyed the voice of the \divine{Lord} my God and did all that he commanded me. \v{15}Look down from your holy habitation in heaven and bless your people Israel and the land that you have given us, just as you promised our ancestors---a land flowing with milk and honey.'\,''
\end{poetry}
\passage{Living for the Glory of God}

\v{16}``The \divine{Lord} your God is commanding you this very day to observe these statutes and judgments. Be careful to obey them with all your heart and soul. \v{17}You have declared this very day that the \divine{Lord} will be your God. You are to walk in his ways, keep his statutes, commands, and judgments, and obey his voice. \v{18}The \divine{Lord} affirmed this day that you are his prized possession. Therefore observe his commands, \v{19}so he may elevate you far above all the nations that he has made. Then you will live to the praise, fame, and glory of God,\fnote{\fbackref{26:19} The Heb. lacks \fbib{of God}} and so be a nation that is holy to the \divine{Lord} your God, as he has promised.''
\labelchapt{27}
\passage{Stone Memorials}

\chapt{27}
\v{1}Moses and the elders of Israel gave these orders to the people: ``Observe all of the commandments\fnote{\fbackref{27:1} So LXX. MT reads \fbib{commandment}} that I'm giving\fnote{\fbackref{27:1} Lit. \fbib{commanding}} you today. \v{2}On the day you cross over the Jordan River to the land that the \divine{Lord} your God is about to give you, set up large stones and coat them with plaster. \v{3}Then inscribe on them all the words of this law when you've crossed over into the land that the \divine{Lord} your God is about to give you---a land flowing with milk and honey---just as the \divine{Lord} God of your ancestors promised you.

\v{4}``When you have crossed the Jordan River, set up these stones about which I'm commanding you today on Mount Ebal, and coat them with plaster. \v{5}Then build an altar there to the \divine{Lord} your God, an altar of stones that hasn't been worked with iron tools. \v{6}Build the altar to the \divine{Lord} your God with uncut stones, then offer a burnt offering to him.\fnote{\fbackref{27:6} Lit. \fbib{the \divine{Lord} your God}} \v{7}Offer a burnt offering there, then eat and rejoice in the presence of the \divine{Lord} your God. \v{8}Inscribe on the stones plainly and distinctly\fnote{\fbackref{27:8} Lit. \fbib{and make good}} all the words of this Law.''

\v{9}Then Moses and the Levitical priests spoke to Israel. They said, ``Be quiet and listen, Israel! Today you have become the people of the \divine{Lord} your God. \v{10}Listen to his voice\fnote{\fbackref{27:10} Lit. \fbib{voice of the \divine{Lord} your God}} and carry out his commands and statutes that I'm giving\fnote{\fbackref{27:10} Lit. \fbib{commanding}} you today.''
\passage{Penalties for Disobedience}

\v{11}Moses gave the people these commands that day:

\begin{poetry}
\poeml \v{12}``When you cross the Jordan River, these tribes\fnote{\fbackref{27:12} The Heb. lacks \fbib{tribes}} are to stand on Mount Gerizim to bless the people---Simeon, Levi, Judah, Issachar, Joseph, and Benjamin. \v{13}The tribes of\fnote{\fbackref{27:13} The Heb. lacks \fbib{the tribes of}} Reuben, Gad, Asher, Zebulun, Dan, and Naphtali are to stand on Mount Ebal to pronounce the curse. \v{14}The descendants of Levi are to declare in a loud voice to every Israeli: \\
\poeml \v{15}```Cursed is the one\fnote{\fbackref{27:15} Lit. \fbib{man}; and so through v. 26} who makes a sculptured or cast image---a detestable thing to the \divine{Lord}, the work of a craftsman---and sets it up secretly.' \\
\poeml ``Then all the people are to respond by saying, `Amen!' \\
\poeml \v{16}``Cursed is the one who treats his father and mother with dishonor.' \\
\poeml ``Then all the people are to respond by saying, `Amen!' \\
\poeml \v{17}```Cursed is the one who moves his neighbor's boundary stone.' \\
\poeml ``Then all the people are to respond by saying, `Amen!' \\
\poeml \v{18}```Cursed is the one who misleads a blind person on the road. \\
\poeml ``Then all the people are to respond by saying, `Amen!' \\
\poeml \v{19}```Cursed is the one who perverts justice due the foreigner, the orphan, or the widow.' \\
\poeml ``Then all the people are to respond by saying, `Amen!' \\
\poeml \v{20}``Cursed is the one who has sexual relations with his father's wife, because he has disgraced his father.\fnote{\fbackref{27:20} Lit. \fbib{has uncovered his father's garment}} \\
\poeml ``Then all the people are to respond by saying, `Amen!' \\
\poeml \v{21}```Cursed is the one who has sexual relations with any animal. \\
\poeml ``Then all the people are to respond by saying, `Amen!' \\
\poeml \v{22}```Cursed is the one who has sexual relations with his sister, the daughter of his father or mother. \\
\poeml ``Then all the people are to respond by saying, `Amen!' \\
\poeml \v{23}```Cursed is the one who has sexual relations with his mother-in-law. \\
\poeml ``Then all the people are to respond by saying, `Amen!' \\
\poeml \v{24}```Cursed is one who strikes his neighbor secretly. \\
\poeml ``Then all the people are to respond by saying, `Amen!' \\
\poeml \v{25}```Cursed is one who accepts a bribe to kill an innocent person. \\
\poeml ``Then all the people are to respond by saying, `Amen!' \\
\poeml \v{26}```Cursed is the one who doesn't uphold the words of this Law and observe them. \\
\poeml ``Then all the people are to respond by saying, `Amen!'\,''
\end{poetry}
\labelchapt{28}
\passage{Rewards for Obedience}

\chapt{28}
\v{1}``Indeed, if you diligently obey\fnote{\fbackref{28:1} Or \fbib{listen to}} the \divine{Lord} your God to carry out all his commands that I'm giving you today, then the \divine{Lord} your God will set you high above all the nations of the earth. \v{2}Moreover, all these blessings will come upon you in abundance,\fnote{\fbackref{28:2} Lit. \fbib{and will overtake you}} if you obey the \divine{Lord} your God:

\v{3}``Blessed will you be in the city and blessed will you be in the country.

\v{4}``Blessed will your children\fnote{\fbackref{28:4} Lit. \fbib{shall the fruit of your womb}} be, as well as the produce of your land, the offspring of your beasts and cattle, and the offspring of your flock.

\v{5}``Blessed will be your grain\fnote{\fbackref{28:5} The Heb. lacks \fbib{grain}} basket and your kneading bowl.

\v{6}``Blessed will you be in your comings and goings.''

\v{7}``The \divine{Lord} will make your enemies, who rise against you and attack from one direction, to flee from you in seven directions.

\v{8}``The \divine{Lord} will send blessings for you with regard to your barns and everything you undertake. Indeed, he will bless you in the land that the \divine{Lord} your God is about to give you.

\v{9}``The \divine{Lord} will assign you to be a holy people\fnote{\fbackref{28:9} Or \fbib{nation}} for himself, just as he promised you, as long as you keep his\fnote{\fbackref{28:9} Lit. \fbib{of the Lord your God}} commands and walk in his ways.

\v{10}``Then all the people of the earth will observe that the name of the \divine{Lord} is proclaimed\fnote{\fbackref{28:10} Lit. \fbib{called}} among you, and they will fear you.

\v{11}``The \divine{Lord} will show his abundant goodness with respect to your children,\fnote{\fbackref{28:11} Lit. \fbib{the fruit of your womb}} the offspring of your animals, and the produce of your farmland that he\fnote{\fbackref{28:11} Lit. \fbib{the \divine{Lord}}} promised your ancestors he would give you.

\v{12}``The \divine{Lord} will open his rich\fnote{\fbackref{28:12} Or \fbib{good}} treasury, the heavens, to release rain upon your land in season and bless everything you undertake so that you'll lend to many nations but won't borrow.

\v{13}``The \divine{Lord} your God will make you the head and not the tail---placing you above and not beneath---if you obey the commands of the \divine{Lord} your God that I'm giving you today to keep and observe. \v{14}Do not deviate from any of his commands that I'm giving you today---neither to the right nor the left---to follow and serve other gods.''
\passage{Reversal of Blessings}

\v{15}``But if you don't obey the \divine{Lord} your God and faithfully carry out all his commands and statutes that I'm giving you today, then all these curses will come upon you and overwhelm you.

\v{16}``Cursed will you be in the city and cursed will you be in the country.

\v{17}``Cursed will be your grain\fnote{\fbackref{28:17} The Heb. lacks \fbib{grain}} basket and your kneading bowl.

\v{18}``Cursed will your children\fnote{\fbackref{28:18} Lit. \fbib{shall the fruit of your womb}} be, as well as the produce of your land, the offspring of your beasts and cattle, and the offspring of your flock.

\v{19}``Cursed will you be in your comings and goings.''
\passage{Diseases and Drought}

\v{20}``The \divine{Lord} will send the curse among you, will confuse you, and will rebuke you in everything you undertake until you are destroyed and perish quickly because of your evil deeds, since you will have forsaken him.\fnote{\fbackref{28:20} Lit. \fbib{me}} \v{21}The \divine{Lord} will cause you to be ill with long-lasting diseases until you are wiped out from the land that you are entering to possess. \v{22}The \divine{Lord} will afflict you with tuberculosis, fever, inflammation, high fever, drought, blight, and mildew. These will attack you until you are completely destroyed. \v{23}The sky above your head will become bronze while the ground beneath you will become iron. \v{24}The \divine{Lord} will change the rain on your land to powder and dust. It will come down from the sky until you are exterminated.''
\passage{From Defeat to Exile}

\v{25}``The \divine{Lord} will cause you to be defeated\fnote{\fbackref{28:25} Lit. \fbib{be struck down}} by your enemies. You'll go out against them in one direction, but you'll flee from them in seven directions. Consequently, you'll be in a state of great terror throughout all the kingdoms of the earth. \v{26}Your dead bodies will be food for the birds of the sky and the wild animals of the earth, with no one to chase them away.

\v{27}``The \divine{Lord} will afflict you with the boils of Egypt, with tumors, skin disease, and festering rashes, and none of them will be curable. \v{28}The \divine{Lord} will afflict you with insanity, blindness, and mental confusion.\fnote{\fbackref{28:28} Lit. \fbib{and confusion of the heart}} \v{29}As a result, you'll wander aimlessly in broad daylight just as a blind person wanders in darkness. You won't prosper in life.\fnote{\fbackref{28:29} Lit. \fbib{in your ways}} Instead you'll be oppressed and plundered all day long, with no deliverer.

\v{30}You'll be engaged to a woman, but another man will rape\fnote{\fbackref{28:30} Lit. \fbib{violate}} her. You'll build a house but you won't live in it. You'll plant a vineyard but you won't harvest\fnote{\fbackref{28:30} Or \fbib{enjoy}} it. \v{31}Your ox will be slaughtered in front of you, and you won't be able to eat it. Your donkey will be stolen from you while you watch and won't be returned to you. Your flock of sheep will be handed to your enemies and there will be no deliverer. \v{32}Your sons and daughters will be given to another people while you watch, and you won't be able to approach them at all,\fnote{\fbackref{28:32} Lit. \fbib{all the day}} and you'll be powerless to help.\fnote{\fbackref{28:32} Lit. \fbib{and there will be no power in your hand}}

\v{33}``A people whom you don't know will devour what your land and labor produces. You'll be only oppressed and discouraged continuously \v{34}until you are driven insane from what your eyes will see.

\v{35}``The \divine{Lord} will inflict you with incurable boils on your knees and legs, and from the sole of your foot to the top of your head.

\v{36}``The \divine{Lord} will banish you and your king whom you will appoint over you to go to a nation that neither you nor your ancestors have known, and there you'll serve other gods of wood and stone. \v{37}You'll become a desolation and a proverb, and you'll be mocked among the people where the \divine{Lord} will drive you.''
\passage{Complete Reversal}

\v{38}``You'll plant many seeds in a field, but your harvest will be small because the locust will consume it. \v{39}You'll plant a vineyard, but you won't drink wine or harvest any grapes, because worms will consume it. \v{40}You'll have olive trees throughout your territory, but you won't be able to anoint yourself with oil, because the olives will drop off the trees. \v{41}You'll bear sons and daughters, but they won't belong to you, because they'll go into captivity. \v{42}Whirling locusts will consume every tree and the produce of your land. \v{43}The foreigner in your midst will be elevated higher and higher over you, while you are brought low little by little. \v{44}He will lend to you, but you won't lend to him. He'll be the head, but you'll be the tail. \v{45}All these curses will come upon you and will overwhelm you until you are exterminated, because you didn't obey\fnote{\fbackref{28:45} Lit. \fbib{listen to the voice}} the \divine{Lord} your God to keep his commands and statutes, which he had commanded you. \v{46}These curses\fnote{\fbackref{28:46} Heb. lacks \fbib{curses}} will serve as a sign and wonder for you and your descendants\fnote{\fbackref{28:46} Lit. \fbib{seed}} as long as you live.''\fnote{\fbackref{28:46} Lit. \fbib{until eternity}}
\passage{Servitude and Bondage}

\v{47}``Because you didn't serve the \divine{Lord} your God joyfully and wholeheartedly,\fnote{\fbackref{28:47} Lit. \fbib{and with gladness of heart}} despite the abundance of everything you have, \v{48}you'll serve your enemies whom the \divine{Lord} your God will send against you. You will serve in famine and in drought,\fnote{\fbackref{28:48} Or \fbib{in hunger and thirst}} in nakedness, and in lack of everything. They'll\fnote{\fbackref{28:48} Lit. \fbib{he}} set a yoke of iron upon your neck until they\fnote{\fbackref{28:48} Lit. \fbib{he}} have exterminated you.

\v{49}``The \divine{Lord} will raise a distant nation against you from the other side of the earth. Swooping down like a vulture, \v{50}it will be a nation whose language you don't understand, whose\fnote{\fbackref{28:50} Lit. \fbib{a nation}} stern appearance\fnote{\fbackref{28:50} Or \fbib{face}} neither shows regard\fnote{\fbackref{28:50} Lit. \fbib{who does not lift faces}} nor extends grace to anyone whether old or young. \v{51}Its army\fnote{\fbackref{28:51} Heb. lacks \fbib{army}} will consume the offspring of your animals and the produce of your soil until you are exterminated. They\fnote{\fbackref{28:51} Lit. \fbib{it}} will leave you without your grain, wine, oil, the increase of your cattle, and the lamb of your flock, until you are completely destroyed. \v{52}They'll\fnote{\fbackref{28:52} Lit. \fbib{it}} besiege all your cities until your high and fortified walls in which you have trusted collapse throughout the land. Indeed, they will besiege all your cities, which the \divine{Lord} your God gave you.''
\passage{Cannibalism}

\v{53}``You'll eat your own children\fnote{\fbackref{28:53} Lit. \fbib{eat the fruit of your womb}}---the flesh of your sons and daughters, whom the Lord your God gave you---on account of the siege and the distress with which your enemy will oppress you. \v{54}Even the compassionate man among you---the very sensitive one---will look with evil in his eyes toward his brother, his beloved wife, and his surviving sons, whom he spared. \v{55}He will withhold from each of them the flesh of his sons that he is eating---since there will be nothing left---on account of the siege and distress with which your enemy will oppress you in all your cities. \v{56}The most tender and sensitive lady among you, who doesn't venture to touch the soles of her feet to the ground on account of her daintiness, will look with hostility in her eyes against her beloved husband, her sons, and her daughters. \v{57}She will eat her afterbirth\fnote{\fbackref{28:57} Lit. \fbib{will begrudge that which comes out from between her feet}} and her newborn children\fnote{\fbackref{28:57} Lit. \fbib{sons whom she will bear}} secretly---since there will be nothing left---on account of the siege and distress with which your enemy will oppress you in your cities.''
\passage{Reduction in Population}

\v{58}``If you aren't careful to observe all the words of this Law that have been written in this book, instructing you\fnote{\fbackref{28:58} The Heb. lacks \fbib{instructing you}} to fear this glorious and awesome name of the \divine{Lord} your God, \v{59}then he\fnote{\fbackref{28:59} Lit. the \fbib{\divine{Lord}}} will inflict extraordinary plagues on you and your children, great and lasting plagues, and severe and lasting illnesses. \v{60}He will inflict\fnote{\fbackref{28:60} Lit. \fbib{will return}} on you all the diseases of Egypt that you dreaded, and they won't be curable.\fnote{\fbackref{28:60} Lit. \fbib{they will cling to you}} \v{61}Moreover, the \divine{Lord} will inflict you with illnesses and plagues that were not written in this Book of the Law, until you are exterminated. \v{62}Because you will not have obeyed\fnote{\fbackref{28:62} Lit. \fbib{listen to the voice of}} the \divine{Lord} your God, very few of you will be left---instead of you being as numerous as the stars in the heavens. \v{63}Just as the \divine{Lord} delighted to prosper and increase you, so now the \divine{Lord} will delight to destroy, exterminate, and banish you from the land that you are about to enter to possess.''
\passage{Scattering among the Nations}

\v{64}``He'll\fnote{\fbackref{28:64} Lit. \fbib{the \divine{Lord}}} scatter you among the nations\fnote{\fbackref{28:64} Lit. \fbib{peoples}} from one end of the earth to the other,\fnote{\fbackref{28:64} Lit. \fbib{end of the earth}} and there you'll serve other gods made of wood and stones, which neither you nor your ancestors have known. \v{65}Among those nations you'll have no rest. There'll be no resting place for the soles of your feet. Instead, the \divine{Lord} will give you an anxious heart, failing eyesight, and a despairing spirit. \v{66}You'll cling to life, being fearful by both night and day, with no assurance of survival. \v{67}In the morning you'll say, `I wish it were evening.' Yet in the evening you'll say, ``I wish it were morning,'' on account of what you'll dread\fnote{\fbackref{28:67} Lit. \fbib{the dread of your heart that you will dread}} and what you'll see.\fnote{\fbackref{28:67} Lit. \fbib{the vision of your eyes that you will see}} \v{68}Finally, the \divine{Lord} will bring you back to Egypt by ship, a place that I said you'll never see again. There you'll try to sell yourselves to your enemies as male and female slaves, but no one will buy you.''
\labelchapt{29}
\passage{Remembering the Exodus and Conquest}

\chapt{29}
\v{1}\fnote{\fbackref{29:1} This v. is 28:69 in MT}These are the terms of the covenant that the \divine{Lord} commanded Moses to make with the Israelis in the land of Moab in addition to the covenant that he made with them in Horeb. \v{2}\fnote{\fbackref{29:2} This v. is 29:1 in MT, and so throughout the chapter}Moses called all Israel together and addressed them: ``You saw everything that the \divine{Lord} did before your eyes in the land of Egypt to Pharaoh, to all his servants,\fnote{\fbackref{29:2} Lit. \fbib{his slaves}} and to his whole country. \v{3}Those great feats that you saw with your own eyes are signs and great wonders. \v{4}Yet to this day, the \divine{Lord} hasn't given you a heart that understands, eyes that perceive, and ears that discern. \v{5}Though I've led you for 40 years in the desert, neither your clothes nor your shoes have worn out. \v{6}You didn't have bread to eat or wine or anything intoxicating to drink, so that you would learn\fnote{\fbackref{29:6} Or \fbib{know}} that I am the \divine{Lord} your God. \v{7}Then you reached this place, where King Sihon of Heshbon and King Og of Bashan had come out to meet and fight with us, but we defeated them. \v{8}We captured their land and handed it as an inheritance to the descendants of Reuben, the descendants of Gad, and half the tribe of Manasseh. \v{9}Therefore, keep the terms of this covenant, carrying them out so that you'll be wise in everything you do.''
\passage{Entering into a Covenant Relationship}

\v{10}``All of you are standing today in the presence of the \divine{Lord} your God---the heads of your tribes, your elders, your magistrates, all the men of Israel, \v{11}along with your children, your wives, even the foreigner in your camp, including the woodchopper and the water drawer---\v{12}to enter into a covenant with the \divine{Lord} your God and into the oath that he\fnote{\fbackref{29:12} Lit. \fbib{the Lord your God}} is about to make with you today, \v{13}so that he will elevate you to be a people for him. And he will be God to you, just as he promised you and swore to your ancestors Abraham, Isaac, and Jacob. \v{14}Now, I'm not making this covenant and oath with you alone, \v{15}but with whoever is here with us standing in the presence of the \divine{Lord} our God today, as well as with those who aren't here with us today.''
\passage{Incurring the Judgment of God}

\v{16}``Now, you know how we lived in the land of Egypt and how we traveled through the territory of\fnote{\fbackref{29:16} The Heb. lacks \fbib{the territory of}} other nations. \v{17}You have seen their detestable practices, their idols of wood, stone, silver, and gold that they had with them. \v{18}Be alert so there is no man, woman, family, or a tribe whose heart is turning away from the \divine{Lord} your God to go and serve the gods of those nations. Be alert so there will be no root among you that produces poisonous and bitter fruit,\fnote{\fbackref{29:18} Lit. \fbib{wormwood}; i.e. bitter things} \v{19}because when such a person\fnote{\fbackref{29:19} Lit. \fbib{he}} hears the words of this oath, he will bless himself and say:

\begin{poetry}
\poeml `I will have a peaceful life, even though I'm determined to be stubborn.'\fnote{\fbackref{29:19} The quotation possibly ends here.} By doing this he will be sweeping away both watered and parched ground alike.'
\end{poetry}

\v{20}``The \divine{Lord} won't forgive such a person.\fnote{\fbackref{29:20} Lit. \fbib{him}} Instead, the zealous anger of the \divine{Lord} will blaze against him. All the curses that were written in this book will fall on him. Then the \divine{Lord} will wipe out his memory\fnote{\fbackref{29:20} Lit. \fbib{name}} from under heaven. \v{21}The \divine{Lord} will set him apart from all the tribes of Israel for destruction,\fnote{\fbackref{29:21} Lit. \fbib{evil}} according to the curses of the covenant that were written in this Book of the Law.''
\passage{A Reminder of Sodom and Gomorrah}

\v{22}``Then the generation to come---your descendants after you and the foreigners who come from afar---will see plagues and illnesses infecting the land that the \divine{Lord} will inflict on it. \v{23}The whole land will be covered\fnote{\fbackref{29:23} The Heb. lacks \fbib{will be covered}} with salt pits and burning sulfur, with nothing planted, nothing sprouting, and producing no vegetation---overthrown like Sodom, Gomorrah, Admah, and Zeboiim, when the \divine{Lord} overthrew them in his raging fury. \v{24}All the nations will ask, `Why did the \divine{Lord} do this to this land? What is the meaning of this fierce and great anger?' \v{25}Then they will answer themselves,\fnote{\fbackref{29:25} The Heb. lacks \fbib{themselves}}

\begin{poetry}
\poeml `Because they've abandoned the covenant of their \divine{Lord}, the God of their ancestors that he had made with them when he brought them out of Egypt. \v{26}They followed and worshipped other gods whom they had not known and whom he did not assign to them. \v{27}For this reason, the anger of the \divine{Lord} raged against this land, to bring upon it all the curses that were written in this book. \v{28}The \divine{Lord} uprooted them from the land in his anger, wrath, and great fury, deporting them to another land, and that's the way things are today.'
\end{poetry}

\v{29}``The secret things belong to the \divine{Lord} our God, but what has been revealed belongs to us and to our children forever, so that we might observe the words of this Law.''
\labelchapt{30}
\passage{Restoration after the Exile}

\chapt{30}
\v{1}``When all these things happen to you---both the blessings and the curses that I've presented to you---and you take them seriously\fnote{\fbackref{30:1} Lit. \fbib{you cause them to return to your heart}} in all the nations where the \divine{Lord} your God will deport you, \v{2}and when you---you and your descendants, that is---will have returned to him\fnote{\fbackref{30:2} Lit. \fbib{the \divine{Lord} your God}} and obeyed all the commands that I'm giving you today with all your heart and soul, \v{3}then the \divine{Lord} your God will restore your fortunes and will show compassion to you. He will gather you from among the nations\fnote{\fbackref{30:3} Lit. \fbib{peoples}} where he\fnote{\fbackref{30:3} Lit. \fbib{the \divine{Lord} your God}} had scattered you. \v{4}Even if the \divine{Lord} had banished you to the ends of the heavens, the \divine{Lord} your God will gather you from there \v{5}and he'll\fnote{\fbackref{30:5} Lit. \fbib{\divine{Lord} your God}} bring you to the land that your ancestors inherited. You'll possess it, you'll prosper, and you'll greatly multiply more than your ancestors did. \v{6}Then the \divine{Lord} your God will circumcise both your hearts and those\fnote{\fbackref{30:6} Lit. \fbib{and the heart}} of your descendants so that you can love him\fnote{\fbackref{30:6} Lit. \fbib{the \divine{Lord} your God}} with your heart and with your soul and therefore live. \v{7}Then the \divine{Lord} your God will inflict all these curses on your enemies and on those who hate and persecute you.''
\passage{Prosperity in Obedience}

\v{8}``So now, return and obey the \divine{Lord} your God and observe all his commands that I'm giving you today, \v{9}and the \divine{Lord} your God will prosper you abundantly in all that you do, along with your children,\fnote{\fbackref{30:9} Lit. \fbib{fruit of your womb}} your livestock, and the produce of your fields, because the \divine{Lord} your God will again be delighted with you for good, just as he was delighted with your ancestors, \v{10}if you obey him\fnote{\fbackref{30:10} Lit. \fbib{the \divine{Lord} your God}} and keep his commands and statutes that are written in this Book of the Law, and if you return to him\fnote{\fbackref{30:10} Lit. \fbib{the \divine{Lord} your God}} with all your heart and soul. \v{11}Indeed, these commands that I'm giving you today are neither confusing nor unattainable for you. \v{12}They aren't in the heavens, so you have to ask, `Who'll go up to the heavens for us and get it for us so we can hear it and act on it?' \v{13}And they aren't beyond the seas either, so you have to ask, `Who'll cross the sea and get it for us so we can hear it and act on it?' \v{14}No, the word is very near you---it's within your mouth and heart for you to attain.''
\passage{Destruction in Disobedience}

\v{15}``Look! Today I have set before you life and what is good, along with death and what is evil. \v{16}That's why I'm commanding you today to love the \divine{Lord} your God by walking in his ways and by observing his commands, statutes, and ordinances, so that you may live long, increase, and so that the \divine{Lord} your God may bless you in the land that you are about to enter to possess. \v{17}But if you turn your heart away, and do not obey, but instead if you stray away to worship and serve other gods, \v{18}I'm declaring to you today that you will surely be destroyed. You won't live long\fnote{\fbackref{30:18} Lit. \fbib{your days will not be long}} in the land that you are crossing the Jordan River to enter and possess. \v{19}I call heaven and earth to testify against you today! I've set life and death before you today: both blessings and curses. Choose life, that it may be well with you---you and your children. \v{20}Love the \divine{Lord} your God, obey his voice, and cling to him, because he is your life---even your long life---so that you may live in the land that the \divine{Lord} promised to give Abraham, Isaac, and Jacob.''
\labelchapt{31}
\passage{Moses Commissions Joshua}

\chapt{31}
\v{1}Moses went and explained these things to everyone in Israel. \v{2}Then he concluded, ``I'm now 120 years old. I'm not able to get around anymore, \v{3}and the \divine{Lord} told me, `You won't be crossing the Jordan River.' But the \divine{Lord} your God is crossing over before you. He will destroy these nations in front of you and you will dispossess them. As for Joshua, he will cross over before you, just as the \divine{Lord} promised. \v{4}The \divine{Lord} will do to them just as he did to Sihon and Og, the kings of the Amorites, and to their land when he destroyed them. \v{5}The \divine{Lord} will hand them over to you, so you can do to them what I've instructed you to do. \v{6}Be strong and courageous. Don't fear or tremble before them, because the \divine{Lord} your God will be the one who keeps on walking with you---he won't leave you or abandon you.''

\v{7}Then Moses called on Joshua and told him in the presence of everyone in Israel, ``Be strong and courageous, because you'll bring this people to the land that the \divine{Lord} your God had promised to give your ancestors. You will be the one who causes them to possess it. \v{8}Indeed, the \divine{Lord} is the one who will keep on walking in front of you. He'll be with you and won't leave you or abandon you, so never be afraid and never be dismayed.''
\passage{Moses Entrusts the Law to the Levitical Priests}

\v{9}Then Moses wrote down this Law and gave it to the Levitical priests who carry the Ark of the Covenant of the \divine{Lord} and to all of Israel's leaders.\fnote{\fbackref{31:9} Lit. \fbib{elders}} \v{10}Then he\fnote{\fbackref{31:10} Lit. \fbib{Moses}} gave these\fnote{\fbackref{31:10} The Heb. lacks \fbib{these}} orders: ``At the end of seven years, the year designated for release,\fnote{\fbackref{31:10} Or \fbib{remission}} during the Festival of Tents,\fnote{\fbackref{31:10} Or \fbib{Tents}} \v{11}when all of Israel comes to appear in the presence of the \divine{Lord} your God at the place that he'll choose, read this Law aloud to them.\fnote{\fbackref{31:11} Lit. \fbib{to all Israel}} \v{12}Gather the people---the men, women, children, and the foreigners that live in your cities---so they may hear and fear the \divine{Lord} your God, and so they may be careful to obey the words contained in this Law. \v{13}Their children who don't know will hear and learn to fear the \divine{Lord} your God as long as you live in the land that you are crossing the Jordan River to possess.''
\passage{Moses and Joshua Present Themselves to the \divine{Lord}}

\v{14}Then the \divine{Lord} told Moses: ``Look! Because your time to die is approaching, call Joshua, present yourselves at the Tent of Meeting, and then I will commission him.'' Moses and Joshua complied\fnote{\fbackref{31:14} Lit. \fbib{walked}} and presented\fnote{\fbackref{31:14} Or \fbib{stood}} themselves at the Tent of Meeting. \v{15}So the \divine{Lord} appeared at the tent in a pillar of cloud that stood above the entrance.\fnote{\fbackref{31:15} Lit. \fbib{above the Tent of Meeting}}

\v{16}Then the \divine{Lord} told Moses, ``Look! You are about to join your ancestors.\fnote{\fbackref{31:16} Lit. \fbib{to lie down with}} Afterwards, this people will rebel\fnote{\fbackref{31:16} Heb. \fbib{rise up}} and commit prostitution with the foreign gods of the land that they are about to enter to possess. They will abandon me and break my covenant that I made with them.\fnote{\fbackref{31:16} Lit. \fbib{him}} \v{17}When that happens,\fnote{\fbackref{31:17} Lit. \fbib{On that day}} my anger will burn against them,\fnote{\fbackref{31:17} Lit. \fbib{him}} because they will have abandoned me. I'll hide my face from them, they will be consumed, and many evils and distresses will find them. When this happens,\fnote{\fbackref{31:17} \fbib{On that day}} they\fnote{\fbackref{31:17} Lit. \fbib{he}} will say, `These troubles have happened to us because God isn't among us.' \v{18}I'll surely hide my face in that day on account of the evil that they\fnote{\fbackref{31:18} Lit. \fbib{he}} will have done for they\fnote{\fbackref{31:18} Lit. \fbib{he}} turned to other gods.''
\passage{Moses Instructed to Teach a Song}

\v{19}``Now write this song and teach it to the Israelis. Put this song in their very mouths, so that it will be a witness for me against the Israelis, \v{20}because after I've brought them to the land flowing with milk and honey that I promised to their ancestors by an oath, they'll eat, grow fat, and then they'll turn to other gods and serve them, while despising me and breaking my covenant. \v{21}Then, when many evils and troubles will have come upon them, this song will serve as a witness against them, since their descendants won't fail to sing it. I know the plan that they are devising even before I bring them into the land that I promised them\fnote{\fbackref{31:21} The Heb. lacks \fbib{them}} by an oath.''

\v{22}So Moses wrote the song that very day and taught it to the Israelis. \v{23}Then the \divine{Lord} charged Nun's son Joshua, ``Be strong and courageous, because you'll bring the Israelis to the land that I promised to them by an oath. I'll be with you.''

\v{24}When Moses had finished writing the words of this Law in a book, \v{25}he\fnote{\fbackref{31:25} Lit. \fbib{Moses}} gave this\fnote{\fbackref{31:25} The Heb. lacks \fbib{this}} charge to the descendants of Levi who carried the Ark of the Covenant of the \divine{Lord}: \v{26}``Take the book of this Law and set it beside the Ark of the Covenant of the \divine{Lord} your God. Let it remain there with you as witness against you, \v{27}because indeed I know your rebellion and stubbornness. Note that even while I'm still alive, you've been rebelling against the \divine{Lord}---how much more so after my death! \v{28}Gather together the leaders\fnote{\fbackref{31:28} Lit. \fbib{elders}} of your tribes and your foremen so I can speak these words in their hearing and call heaven and the earth as witnesses against them, \v{29}because I know that after my death, you'll surely act wickedly and turn from the road that I've instructed you. As a result, evil will fall on you in days to come, because you'll act wickedly in the sight of the \divine{Lord}, causing him to become angry due to your behavior.'' \v{30}So Moses spoke the words of this song---to the very end---in front of the entire assembly of Israel.
\labelchapt{32}
\passage{The Song of Moses}

\begin{poetry}
\poeml \chapt{32}
\v{1}Hear, heavens, and I will speak! \\
\poemll    Listen, earth, to the words of my mouth! \\
\poeml \v{2}May my instructions descend like rain \\
\poemll    and may my words flow like dew, \\
\poeml as light rain upon the grass, \\
\poemll    and as showers upon new plants. \\
\poeml \v{3}For I'll proclaim the name of our \divine{Lord}. \\
\poemll    Ascribe greatness to our God! \\
\poeml \v{4}Flawless is the work of the Rock, \\
\poemll    because all his ways are just. \\
\poeml A faithful God---never unjust--- \\
\poemll    righteous and upright is he. \\
\poeml \v{5}But those who are not his children \\
\poemll    acted corruptly against him; \\
\poemlll       they are a defective and perverted generation. \\
\poeml \v{6}This is not the way to repay the \divine{Lord}, is it, \\
\poemll    you foolish and witless people? \\
\poeml Is he not your father, \\
\poemll    who bought you, formed you, and established you?
\passage{An Exhortation to Remember God's Work}
\poeml \v{7}Remember the days of old, \\
\poemll    reflect on the years of previous generations. \\
\poeml Ask your father, \\
\poemll    and he'll tell you; \\
\poemlll       your elders will inform you. \\
\poeml \v{8}When the Most High gave nations as their inheritance, \\
\poemll    when he separated the human race, \\
\poeml he set boundaries for the people \\
\poemll    according to the number of the children of God.\fnote{\fbackref{32:8} So with LXX and DSS 4QDeut. MT reads \fbib{the Israelis}} \\
\poeml \v{9}For the \divine{Lord}'s portion is his people; \\
\poemll    Jacob is his allotted portion.
\passage{The \divine{Lord}'s Work on Behalf of Israel}
\poeml \v{10}The \divine{Lord}\fnote{\fbackref{32:10} Lit. \fbib{He}} found him\fnote{\fbackref{32:10} I.e. Jacob as a personification of national Israel; and so throughout the song} in a desert land, \\
\poemll    in a barren, eerie\fnote{\fbackref{32:10} Lit. \fbib{howling}} wilderness. \\
\poeml He surrounded, cared for, and guarded him \\
\poemll    as the pupil of his eye. \\
\poeml \v{11}Like an eagle stirs its nest, \\
\poemll    hovering near its young, \\
\poeml spreading out his wings to take him \\
\poemll    and carry him on his pinions, \\
\poeml \v{12}the \divine{Lord} alone guided him. \\
\poemll    There was no foreign god with him. \\
\poeml \v{13}He mounted him on a high place above the earth, \\
\poemll    feeding him from the produce of the field. \\
\poeml He nourished\fnote{\fbackref{32:13} Or \fbib{nursed}} him with honey from the rock \\
\poemll    and with oil from the flint rock, \\
\poeml \v{14}with curds from cattle and with milk from sheep, \\
\poemll    with the fat of lambs, with rams from Bashan, \\
\poeml with the fat of goats, with the finest\fnote{\fbackref{32:14} Lit. \fbib{kernel}} of wheat--- \\
\poemll    and from the juice of grapes you drank wine.
\passage{Israel's Rebellion}
\poeml \v{15}Jacob dined until satisfied;\fnote{\fbackref{32:15} So DSS Q Sam and LXX; the Heb. lacks \fbib{Jacob dined until satisfied}} \\
\poemll    Jeshurun\fnote{\fbackref{32:15} I.e. a poetic term for national Israel; the Heb. name means \fbib{Upright One}} grew fat and kicked. \\
\poeml He\fnote{\fbackref{32:15} Lit. \fbib{You}} grew fat, coarse, and gross, \\
\poemll    so that he abandoned the God who made him \\
\poemlll       and spurned the Rock that was his salvation. \\
\poeml \v{16}They provoked him to jealousy over foreigners \\
\poemll    and to anger over detestable things. \\
\poeml \v{17}They sacrificed to demons--- \\
\poemll    not to the real God--- \\
\poeml gods whom they didn't know, \\
\poemll    new neighbors who had recently appeared, \\
\poemlll       whom your ancestors never feared. \\
\poeml \v{18}You\fnote{\fbackref{32:18} I.e. the nation of Israel personified in the second person sing. pronoun, and so throughout the verse} neglected the Rock that fathered you; \\
\poemll    you abandoned God, who was awaiting your birth.\fnote{\fbackref{32:18} Or \fbib{who was giving birth to you}}
\passage{The \divine{Lord}'s Response}
\poeml \v{19}The \divine{Lord} saw it and became jealous,\fnote{\fbackref{32:19} So DSS, LXX. MT reads \fbib{and was repulsed}} \\
\poemll    provoked by his sons and daughters. \\
\poeml \v{20}So he said: \\
\poeml ``Let me hide my face from them. \\
\poemll    I will observe what their end will be, \\
\poeml because they are a perverted generation, \\
\poemll    children within whom there is no loyalty. \\
\poeml \v{21}They provoked me to jealousy over non-gods, \\
\poemll    and to be angry over their vanity. \\
\poeml Now I'll provoke them to jealousy over a non-people; \\
\poemll    and over a foolish nation I'll provoke them to anger. \\
\poeml \v{22}For a fire breaks out in my anger--- \\
\poemll    burning to the deepest part of\fnote{\fbackref{32:22} The Heb. lacks \fbib{part of}} the afterlife,\fnote{\fbackref{32:22} Lit. \fbib{Sheol}} \\
\poeml consuming the earth and its produce \\
\poemll    and igniting the foundations of the mountains. \\
\poeml \v{23}I'll bury them in misfortunes \\
\poemll    and bring them to an end with my arrows. \\
\poeml \v{24}Emaciated from famine, \\
\poemll    feverish from plague, \\
\poemll    and destroyed by bitterness, \\
\poeml I'll send fanged beasts against them, \\
\poemll    along with poisonous snakes that glide through the dust. \\
\poeml \v{25}Outside, the sword will cause bereavement; \\
\poemll    within,\fnote{\fbackref{32:25} Lit. \fbib{within the room}} there will be terror \\
\poemlll       for the young man and virgin alike, \\
\poemlll       also for the nursing infant and the aged man.''\fnote{\fbackref{32:25} Lit. \fbib{and a man of gray hair}} \\
\poeml \v{26}``I said, \\
\poeml `I will scatter them,\fnote{\fbackref{32:26} Or \fbib{will break them to pieces}} \\
\poemll    erasing their memory from the human race,\fnote{\fbackref{32:26} Lit. \fbib{from among men}} \\
\poeml \v{27}if it weren't for dreading the taunting of their enemies--- \\
\poemll    otherwise, their adversary might misinterpret and say, \\
\poeml ``Our power is great. \\
\poemll    It isn't the \divine{Lord} who made all of this happen.''\,'\,''
\passage{Moses Warns Israel}
\poeml \v{28}They are a nation devoid of purpose \\
\poemll    and without insight. \\
\poeml \v{29}O, that they were wise to understand this \\
\poemll    and consider their future!\fnote{\fbackref{32:29} Lit. \fbib{end}} \\
\poeml \v{30}How can one person\fnote{\fbackref{32:30} The Heb. lacks \fbib{person}} chase a thousand of them \\
\poemll    and two put a myriad\fnote{\fbackref{32:30} Or \fbib{put countless ones}; Lit. \fbib{put ten thousand}} to flight, \\
\poeml unless their Rock delivers them \\
\poemll    and the \divine{Lord} gives them up? \\
\poeml \v{31}For their rock isn't like our Rock, \\
\poemll    as even\fnote{\fbackref{32:31} Lit. \fbib{and}} our enemies admit.\fnote{\fbackref{32:31} Or \fbib{concede}} \\
\poeml \v{32}Instead,\fnote{\fbackref{32:32} Lit. \fbib{Because}} their vine is from the vines of Sodom \\
\poemll    and the vineyards of Gomorrah. \\
\poeml Their grapes are poisonous, \\
\poemll    their clusters bitter. \\
\poeml \v{33}Their wine is the venom of serpents, \\
\poemll    a poisonous cobra.
\passage{The \divine{Lord}'s Response}
\poeml \v{34}``Is this not kept in reserve, \\
\poemll    sealed up with me in my treasury? \\
\poeml \v{35}To me belong vengeance and recompense. \\
\poemll    In due time their feet will slip, \\
\poeml because their time of calamity is near \\
\poemll    and the things prepared for them draw near. \\
\poeml \v{36}For the \divine{Lord} will vindicate his people \\
\poemll    and bring comfort to his servants, \\
\poeml because he will observe that their power\fnote{\fbackref{32:36} Lit. \fbib{hand}} has waned, \\
\poemll    when neither prisoner\fnote{\fbackref{32:36} Or \fbib{slave}} nor free person remain. \\
\poeml \v{37}``He will say, `Where are their gods, \\
\poemll    the rock in which they took refuge? \\
\poeml \v{38}Who ate the fat of their offerings \\
\poemll    and drank the wine that was their drink offering? \\
\poeml Let them rise and help you \\
\poemll    and be your hiding place!' \\
\poeml \v{39}``Look now! I AM,\fnote{\fbackref{32:39} So LXX; MT reads \fbib{I, I myself, am he}} \\
\poemll    and there is no other god besides me. \\
\poeml I myself cause death \\
\poemll    and I sustain life; \\
\poeml I wound severely \\
\poemll    and I also heal; \\
\poemlll       from my power\fnote{\fbackref{32:39} Lit. \fbib{hand}} no one can deliver. \\
\poeml \v{40}``I solemnly swear\fnote{\fbackref{32:40} Lit. \fbib{raise my hand}} to heaven--- \\
\poemll    I say `As certainly as I'm alive and living forever, \\
\poeml \v{41}I'll whet my shining sword, \\
\poemll    with my hands in firm grasp of judgment. \\
\poeml I'll show vengeance on my adversary \\
\poemll    and repay those who keep on hating me. \\
\poeml \v{42}I'll make my arrows drunk with blood. \\
\poemll    My sword will devour flesh, \\
\poemlll       along with the blood of the slain, \\
\poemll    and I'll take their enemy leaders captive.' \\
\poeml \v{43}``Sing for joy, nations! \\
\poemll    Sing for joy,\fnote{\fbackref{32:43} The Heb. lacks \fbib{Sing for joy}} people who belong to him! \\
\poeml For he'll avenge the blood of his servants, \\
\poemll    turn on his adversary, \\
\poemlll       and cleanse both his land and his people.''
\end{poetry}
\passage{Moses' Final Counsel}

\v{44}So Moses and Nun's son Joshua came and recited all the words of this song while the people were assembled. \v{45}When Moses had finished addressing all of these words to all Israel, \v{46}he told them, ``Take to heart my entire testimony against you today. Command your children to observe carefully every word of this Law, \v{47}because they're not just empty words for you---they are your very life. Through these instructions you will live long in the land that you are about to cross over the Jordan River to possess.''
\passage{Moses Forbidden to Enter Canaan}

\v{48}Later that day, the \divine{Lord} told Moses, \v{49}``Ascend this Abarim mountain range\fnote{\fbackref{32:49} The Heb. lacks \fbib{range}} toward Mount Nebo in the land of Moab across from Jericho, and look out over the land of Canaan that I'm about to give to the Israelis as a possession. \v{50}You will die on the mountain that you are about to ascend and be taken to be with your ancestors, just as your brother Aaron died on Mount Hor and was taken to be with his ancestors. \v{51}Both of you acted unfaithfully against me among the Israelis at Meribah-kadesh in the desert of Zin, when you failed to uphold my holiness among the Israelis. \v{52}You'll see the land from a distance, but you won't be able to enter the land that I am about to give to the Israelis.''
\labelchapt{33}
\passage{Moses Reviews the Tribes of Israel}

\chapt{33}
\v{1}This is the blessing with which Moses, the man of God, blessed the Israelis before his death. \v{2}He said:

\begin{poetry}
\poeml ``The \divine{Lord} came from Sinai. \\
\poemll    Rising from Seir upon us,\fnote{\fbackref{33:2} So LXX. MT reads \fbib{them}} \\
\poeml he shone forth from Mount Paran, \\
\poemll    accompanied\fnote{\fbackref{33:2} Lit. \fbib{came} or \fbib{brought}} by a myriad\fnote{\fbackref{33:2} Or \fbib{by countless}; Lit. \fbib{by ten thousands}} of his holy ones, \\
\poemlll       with flaming fire from his right hand for them. \\
\poeml \v{3}Indeed, lover of people, \\
\poemll    all of his holy ones are in your control.\fnote{\fbackref{33:3} Lit. \fbib{hand}} \\
\poeml They gather at your feet \\
\poemll    to do as you have instructed.\fnote{\fbackref{33:3} Lit. \fbib{to do your word}} \\
\poeml \v{4}Moses commanded with the Law, \\
\poemll    an inheritance for the community of Jacob. \\
\poeml \v{5}The \divine{Lord}\fnote{\fbackref{33:5} The Heb. lacks \fbib{The \divine{Lord}}} was king of Jeshurun\fnote{\fbackref{33:5} I.e. a poetic term for national Israel; the Heb. name means \fbib{Upright One}} \\
\poemll    when the leaders of the people--- \\
\poemlll       all the tribes of Israel---gathered together.''
\passage{Reuben}
\poeml \v{6}``May Reuben live and not die, \\
\poemll    though his numbers are few.''
\end{poetry}
\passage{Judah}

\v{7}He declared this about Judah:

\begin{poetry}
\poeml ``Hear, \divine{Lord}, the voice of Judah \\
\poemll    and return him to his people. \\
\poeml With his own strength he fights for himself, \\
\poemll    and you will be of assistance\fnote{\fbackref{33:7} So LXX. MT reads \fbib{be his helper}} against\fnote{\fbackref{33:7} Lit. \fbib{from}} his enemies.''
\end{poetry}
\passage{Levi}

\v{8}About Levi he said:

\begin{poetry}
\poeml ``Let your Thummim and Urim\fnote{\fbackref{33:8} I.e. the jewel-encrusted breastplate worn by the high priest by which the will of God could be revealed; cf. Ezra 2:63, Neh 7:65} be with the man \\
\poemll    to whom you showed gracious love, \\
\poeml whom you tested at Massah \\
\poemll    and with whom you struggled \\
\poemlll       at the waters of Meribah, \\
\poeml \v{9}the one who told his mother and father, \\
\poemll    `I don't know\fnote{\fbackref{33:9} Lit. \fbib{see}} them,' \\
\poeml and who would neither acknowledge his brothers \\
\poemll    nor know his own children. \\
\poeml For they kept your word \\
\poemll    and guarded your covenant. \\
\poeml \v{10}They will teach your ordinances to Jacob, \\
\poemll    and your Law to Israel. \\
\poeml They will offer incense as a pleasant aroma to you\fnote{\fbackref{33:10} Lit. \fbib{for your nose}} \\
\poemll    and a whole burnt offering upon your altar. \\
\poeml \v{11}\divine{Lord}, bless his substance \\
\poemll    and approve the work that he undertakes.\fnote{\fbackref{33:11} Lit. \fbib{work of his hands}} \\
\poeml Shatter the legs\fnote{\fbackref{33:11} Or \fbib{loins}} of those who oppose against him; \\
\poemll    may those who hate him stand no more.''
\end{poetry}
\passage{Benjamin}

\v{12}About Benjamin he said:

\begin{poetry}
\poeml ``The beloved of the \divine{Lord} will live confidently, \\
\poemll    the Most High protecting\fnote{\fbackref{33:12} Or \fbib{shading}} him all day long, \\
\poemlll       and resting in his bosom.\fnote{\fbackref{33:12} Lit. \fbib{between his shoulders}}''
\end{poetry}
\passage{Joseph}

\v{13}About Joseph he said:

\begin{poetry}
\poeml ``May the blessing of the \divine{Lord} be on his land: \\
\poemll    dew from the choicest of the heavens, \\
\poemlll       and from the depths beneath; \\
\poeml \v{14}from the choicest products of the sun, \\
\poemll    the rich fruit of the harvest moon,\fnote{\fbackref{33:14} Lit. \fbib{the month}} \\
\poeml \v{15}the choicest portion\fnote{\fbackref{33:15} The Heb. lacks \fbib{portion}} of the eternal mountains, \\
\poemll    and the best of the everlasting hills; \\
\poeml \v{16}from the choicest of the earth and its fullness, \\
\poemll    and the favor of the one who lived in the burning\fnote{\fbackref{33:16} The Heb. lacks \fbib{burning}} bush. \\
\poeml May blessing\fnote{\fbackref{33:16} Lit. \fbib{it}} rest on Joseph's head, \\
\poemll    and on the crown of the head \\
\poemlll       of the one set apart from his brothers. \\
\poeml \v{17}May the firstborn of his bull be honorable to him, \\
\poemll    and may his horns be those of a wild ox. \\
\poeml With them may he push people all together, \\
\poemll    to the ends of the earth. \\
\poeml These are the myriads\fnote{\fbackref{33:17} Or \fbib{the countless ones}; Lit. \fbib{the ten thousands}} of Ephraim \\
\poemll    and the thousands of Manasseh.''
\end{poetry}
\passage{Zebulun and Issachar}

\v{18}About Zebulun he said:

\begin{poetry}
\poeml ``Zebulun, rejoice as you go out \\
\poemll    and Issachar, in being inside your tents. \\
\poeml \v{19}They will call the nations\fnote{\fbackref{33:19} Lit. \fbib{peoples}} to the mountain, \\
\poemll    and there they will offer righteous sacrifices, \\
\poeml for they'll draw from the abundance of the sea \\
\poemll    and from the hidden treasures of the sand.''
\end{poetry}
\passage{Gad}

\v{20}About Gad he said:

\begin{poetry}
\poeml ``Blessed be the one who enlarges Gad! \\
\poemll    Like a roaring lion, he crouches, \\
\poemlll       tearing arm and scalp. \\
\poeml \v{21}He chose the best part for himself, \\
\poemll    when the leader's portion was assigned. \\
\poeml He came at the head of the people, \\
\poemll    carrying out the \divine{Lord}'s justice \\
\poemlll       and his ordinances concerning Israel.''
\end{poetry}
\passage{Dan}

\v{22}About Dan he said:

\begin{poetry}
\poeml ``Dan is a lion's cub, \\
\poemll    leaping forth from Bashan.''
\end{poetry}
\passage{Naphtali}

\v{23}About Naphtali he said:

\begin{poetry}
\poeml ``Naphtali, full of favor and the \divine{Lord}'s blessing, \\
\poemll    take possession of the west\fnote{\fbackref{33:23} Or \fbib{sea}} and south.''
\end{poetry}
\passage{Asher}

\v{24}About Asher he said:

\begin{poetry}
\poeml ``May Asher be blessed, along with his descendants, \\
\poemll    may his brothers be pleased with him, \\
\poemlll       may he dip his feet in oil, \\
\poeml \v{25}may your bolts be made of iron and bronze, \\
\poemll    and may your strength be sufficient for each day you live.''
\passage{Israel's Defender}
\poeml \v{26}``There is no one like the God of Jeshurun,\fnote{\fbackref{33:26} I.e. a poetic term for national Israel; the Heb. name means \fbib{Upright One}} \\
\poemll    who rides through the heavens \\
\poemlll       with its lofty clouds to help you. \\
\poeml \v{27}The God of old is a dwelling place, \\
\poemll    with everlasting arms underneath. \\
\poeml He drove out your enemies before you \\
\poemll    and said: `Destroy them!' \\
\poeml \v{28}So Israel lives in confidence, \\
\poemll    isolated as the fountain of Jacob \\
\poeml in a land of grain and new wine, \\
\poemll    where the heavens rain down dew. \\
\poeml \v{29}How blessed are you, Israel! \\
\poemll    Who can be like you, \\
\poeml a people delivered by the \divine{Lord}, \\
\poemll    your shield of help and \\
\poemlll       your finely crafted sword. \\
\poeml May your enemies cower before you. \\
\poemll    You will tread down their high places.''
\end{poetry}
\labelchapt{34}
\passage{Moses Ascends Pisgah}

\chapt{34}
\v{1}Moses ascended from the desert plain of Moab toward Mount Nebo, to the top of Pisgah, across from Jericho. There\fnote{\fbackref{34:1} The Heb. lacks \fbib{There}} the \divine{Lord} showed him the entire land, from Gilgal as far as Dan, \v{2}all of Naphtali, the territories\fnote{\fbackref{34:2} Lit. \fbib{land}} of Ephraim and Manasseh, and the entire territory\fnote{\fbackref{34:2} Lit. \fbib{land}} of Judah all the way to out over the sea,\fnote{\fbackref{34:2} I.e. the Mediterranean Sea} \v{3}including the Negev,\fnote{\fbackref{34:3} I.e. the southern regions of the Sinai peninsula; cf. Josh 10:40} the Arabah, the valley of Jericho, and the city of the palm trees as far as Zoar. \v{4}Then the \divine{Lord} told him: ``This is the land that I promised to Abraham, Isaac, and Jacob by an oath when I said, `I'll give it to your descendants.' I'll let you see it with your eyes, but you won't cross over there.''
\passage{Moses Dies}

\v{5}So Moses, the servant of the \divine{Lord}, died there in the land of Moab, just as the \divine{Lord} had said.\fnote{\fbackref{34:5} Lit. \fbib{Moab, according to the word of the \divine{Lord}}} \v{6}He was buried in the valley opposite Beth Peor, in the land of Moab, but no one knows to this day where his burial place is. \v{7}Moses was 120 years old when he died. His eyesight wasn't impaired and he was still vigorous and strong. \v{8}The Israelis mourned for Moses at the desert plain of Moab for 30 days, after which the period of mourning for Moses was completed.
\passage{The Epitaph for Moses}

\v{9}Now Nun's son Joshua was full of the spirit of wisdom, because Moses had placed his hands on him, so Israelis listened to him and did what the \divine{Lord} had commanded Moses. \v{10}No prophet ever rose again in Israel like Moses, whom the \divine{Lord} knew with such great intimacy.\fnote{\fbackref{34:10} Lit. \fbib{knew face to face}}

\v{11}What signs and wonders the \divine{Lord} sent him to do throughout the land of Egypt, to Pharaoh, and to all of his servants who lived in the whole land!

\v{12}What great power\fnote{\fbackref{34:12} Lit. \fbib{What a mighty hand}} and great terror Moses displayed on behalf of all Israel!
