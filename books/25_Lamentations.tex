\bookheader{Lamentations}
\labelbook{Lam}

\bookpretitle{The Book of}
\booktitle{Lamentations}

\labelchapt{1}
\passage{The Sorrowful City\fnote{This book is an acrostic---successive verses begin with a consecutive letter of the Heb. alphabet except in chapter 3, where every three verses begin with the same consecutive Heb. letter.}}

\chapt{1}
\v{1}How lonely she lies,

\begin{poetry}
\poemll    the city that thronged with people! \\
\poeml Like a widow she has become, \\
\poemll    this great one among nations! \\
\poeml The princess among provinces \\
\poemll    has become a vassal. \\
\poeml \v{2}Bitterly she cries in the night, \\
\poemll    as tears stream down\fnote{\fbackref{1:2} The Heb. lacks \fbib{stream down}} her cheeks. \\
\poeml No one consoles her \\
\poemll    of all her friends. \\
\poeml All her neighbors have betrayed her; \\
\poemll    they have become her enemies. \\
\poeml \v{3}Judah has gone into exile \\
\poemll    to escape affliction and servitude. \\
\poeml She that sat among the nations, \\
\poemll    has found no rest. \\
\poeml All her pursuers overtook her \\
\poemll    amid narrow passes. \\
\poeml \v{4}The roads that lead to Zion are in mourning, \\
\poemll    because no one travels to the festivals. \\
\poeml All her gates are desolate; \\
\poemll    her priests are moaning. \\
\poeml Her young women\fnote{\fbackref{1:4} Lit. \fbib{virgins}} are grieving,\fnote{\fbackref{1:4} Or \fbib{are led away}.} \\
\poemll    and she is bitter. \\
\poeml \v{5}Her adversaries dominate her, \\
\poemll    her enemies prosper. \\
\poeml For the \divine{Lord} has made her suffer \\
\poemll    because of her many transgressions. \\
\poeml Her children have gone away, \\
\poemll    taken into captivity in the presence of the enemy. \\
\poeml \v{6}Fled from cherished\fnote{\fbackref{1:6} Lit. \fbib{from the daughter of}} Zion \\
\poemll    are all that were her splendor. \\
\poeml Her princes have become like deer \\
\poemll    that cannot find their feeding grounds. \\
\poeml They flee with strength exhausted \\
\poemll    from their pursuers. \\
\poeml \v{7}Jerusalem remembers\fnote{\fbackref{1:7} Or \fbib{Remember, \divine{Lord}, Jerusalem,}} \\
\poemll    her time of affliction and misery; \\
\poeml all her valued belongings\fnote{\fbackref{1:7} Or \fbib{Perished are all her valued belongings}} \\
\poemll    of days gone by, \\
\poeml when her people fell into enemy hands, \\
\poemll    with no one to help her, \\
\poeml and her enemies stared at her, \\
\poemll    mocking her downfall. \\
\poeml \v{8}Jerusalem sinned greatly, \\
\poemll    and she became unclean.\fnote{\fbackref{1:8} Lit. \fbib{has been removed}; i.e. due to ritual uncleanness} \\
\poeml All who honored her now despise her, \\
\poemll    because they saw her naked. \\
\poeml She herself groans \\
\poemll    and turns her face away. \\
\poeml \v{9}Uncleanness has soiled her skirts, \\
\poemll    and she gave no thought to what would follow. \\
\poeml She fell in such a startling way, \\
\poemll    with no one to comfort her. \\
\poeml Look, \divine{Lord}, upon my affliction, \\
\poemll    because my enemy is boasting. \\
\poeml \v{10}The adversary seized in his hands \\
\poemll    everything she valued. \\
\poeml She watched the nations\fnote{\fbackref{1:10} Or \fbib{watched foreigners}} \\
\poemll    enter her sanctuary; \\
\poeml those you forbade to enter \\
\poemll    your place of meeting. \\
\poeml \v{11}All her people groaned \\
\poemll    as they searched for food. \\
\poeml They traded their valuables in order to eat, \\
\poemll    to keep themselves alive.\fnote{\fbackref{1:11} Lit. \fbib{to eat to refresh the soul}} \\
\poeml Look, \divine{Lord}, and see \\
\poemll    how I have become dishonored. \\
\poeml \v{12}May it not befall you,\fnote{\fbackref{1:12} Lit. \fbib{It is not for you}} \\
\poemll    all who pass along the road! \\
\poeml Look and see: \\
\poemll    Is there any grief \\
\poeml like my grief \\
\poemll    dealt out to me, \\
\poeml by which the \divine{Lord} afflicted me \\
\poemll    in the time of his fierce wrath? \\
\poeml \v{13}He sent fire from on high,\fnote{\fbackref{1:13} Lit. \fbib{high into my bones}} \\
\poemll    making it penetrate my bones.\fnote{\fbackref{1:13} Lit. \fbib{overcoming her}} \\
\poeml He stretched out a net at my feet, \\
\poemll    forcing me to turn back. \\
\poeml He made me desolate; \\
\poemll    I'm fainting all day long. \\
\poeml \v{14}The yoke of my sins was bound on,\fnote{\fbackref{1:14} Lit. \fbib{was heavy}} \\
\poemll    fastened together by his hand. \\
\poeml They settled on my neck; \\
\poemll    he caused my strength to fail. \\
\poeml The \divine{Lord} placed me in the power \\
\poemll    of those I cannot resist. \\
\poeml \v{15}He rejected all the valiant men--- \\
\poemll    the \divine{Lord}, in my midst. \\
\poeml He set a time to meet with me \\
\poemll    to crush my young warriors. \\
\poeml The \divine{Lord} has trampled, as in a winepress, \\
\poemll    the fair virgin that is\fnote{\fbackref{1:15} Lit. \fbib{the virgin daughter of}} Judah. \\
\poeml \v{16}Because of all this, I weep; \\
\poemll    my eyes\fnote{\fbackref{1:16} Lit. \fbib{my eyes my eyes}} stream with tears \\
\poeml because far from me \\
\poemll    is the comforter of my soul. \\
\poeml My children are sorrowful, \\
\poemll    because the enemy has won. \\
\poeml \v{17}Zion spreads out her hands;\fnote{\fbackref{1:17} Or \fbib{Zion rent her linen garments}} \\
\poemll    no one is there to comfort her. \\
\poeml The \divine{Lord} has issued an order against\fnote{\fbackref{1:17} Or \fbib{\divine{Lord} kept watch over}} Jacob, \\
\poemll    that all who are around him are to be his enemies; \\
\poeml Jerusalem has become \\
\poemll    unclean among them. \\
\poeml \v{18}The \divine{Lord} is in the right, \\
\poemll    but I rebelled against his commands. \\
\poeml Listen, please, all you people, \\
\poemll    and look at my pain--- \\
\poeml my young men and women\fnote{\fbackref{1:18} Lit. \fbib{virgins}} \\
\poemll    have gone into captivity. \\
\poeml \v{19}I called out to my lovers,\fnote{\fbackref{1:19} Or \fbib{friends}} \\
\poemll    but they deceived me. \\
\poeml My priests and my elders \\
\poemll    have died within the city \\
\poeml while looking for something to eat \\
\poemll    to keep themselves alive. \\
\poeml \v{20}Look, \divine{Lord}, how distressed I am; \\
\poemll    all my insides are churning. \\
\poeml My heart is troubled within me, \\
\poemll    because I vigorously rebelled. \\
\poeml Outside the sword brings loss of life, \\
\poemll    while at home death rules. \\
\poeml \v{21}People\fnote{\fbackref{1:21} Lit. \fbib{They}} heard how I groan, \\
\poemll    with no one to comfort me. \\
\poeml All my adversaries have heard about my troubles; \\
\poemll    they rejoice that you have caused them. \\
\poeml Bring on the day you have promised, \\
\poemll    so my adversaries\fnote{\fbackref{1:21} Lit. \fbib{so they}} will become like me. \\
\poeml \v{22}May all of their wickedness come to your attention, \\
\poemll    and deal with them \\
\poeml as you have done with me \\
\poemll    because of all my transgressions. \\
\poeml For I am constantly groaning, \\
\poemll    and my heart is faint.
\end{poetry}
\labelchapt{2}
\passage{The Condition of Israel}

\begin{poetry}
\poeml \chapt{2}
\v{1}How the Lord in his wrath \\
\poemll    shamed\fnote{\fbackref{2:1} Or \fbib{enveloped}} cherished\fnote{\fbackref{2:1} Lit. \fbib{the daughter of}} Zion! \\
\poeml He cast down from heaven to earth \\
\poemll    the glory of Israel, \\
\poeml He did not remember his footstool\fnote{\fbackref{2:1} I.e. the Temple} \\
\poemll    in the time of his anger. \\
\poeml \v{2}The Lord swallowed up without pity \\
\poemll    all of Jacob's habitations. \\
\poeml In his wrath he tore down \\
\poemll    the strongholds of fair Judah.\fnote{\fbackref{2:2} Lit. \fbib{of the daughter of Judah}} \\
\poeml He cast to the ground in dishonor \\
\poemll    both her kingdom and its rulers. \\
\poeml \v{3}In his fierce wrath he cut off \\
\poemll    all the strength\fnote{\fbackref{2:3} Lit. \fbib{every horn}} of Israel. \\
\poeml He withdrew his protection\fnote{\fbackref{2:3} Lit. \fbib{his right hand}} \\
\poemll    as the enemy approached.\fnote{\fbackref{2:3} Lit. \fbib{in front of the enemy}} \\
\poeml He burned Jacob like a blazing fire \\
\poemll    consumes everything around it. \\
\poeml \v{4}He bent his bow against us\fnote{\fbackref{2:4} The Heb. lacks \fbib{against us}} as would an enemy, \\
\poemll    his right hand cocked as would an adversary. \\
\poeml He has killed everyone in whom we took pride; \\
\poemll    in the tent of cherished\fnote{\fbackref{2:4} Lit. of \fbib{the daughter of}} Zion he poured out \\
\poemlll       his anger like fire. \\
\poeml \v{5}The Lord has become like an enemy--- \\
\poemll    he has devoured Israel. \\
\poeml He has devoured all of her palaces, \\
\poemll    destroying her fortresses. \\
\poeml He filled cherished Judah\fnote{\fbackref{2:5} Lit. \fbib{the daughters of Judah}} \\
\poemll    with mourning and lament. \\
\poeml \v{6}He plowed under his Temple\fnote{\fbackref{2:6} Lit. \fbib{tent}} like a garden, \\
\poemll    spoiling his tent. \\
\poeml The \divine{Lord} abolished in Zion \\
\poemll    both festivals and Sabbaths. \\
\poeml In his fierce wrath he despised \\
\poemll    both king and priest. \\
\poeml \v{7}The Lord rejected his altar, \\
\poemll    disavowing his sanctuary. \\
\poeml He gave up her palace walls \\
\poemll    to the control of the enemy. \\
\poeml They shouted in the \divine{Lord}'s Temple, \\
\poemll    as though they were attending a day of celebration. \\
\poeml \v{8}The \divine{Lord} planned to destroy \\
\poemll    the walls of cherished\fnote{\fbackref{2:8} Lit. \fbib{of the daughter of}} Zion. \\
\poeml He measured them with his line. \\
\poemll    He did not withhold his hand from destruction. \\
\poeml He made both ramparts and defensive walls mourn; \\
\poemll    they languish together. \\
\poeml \v{9}Jerusalem's\fnote{\fbackref{2:9} Lit. \fbib{Her}} gates collapsed to the ground; \\
\poemll    he destroyed and broke the bars of her gates.\fnote{\fbackref{2:9} The Heb. lacks \fbib{gates}} \\
\poeml Both king and prince have gone into captivity.\fnote{\fbackref{2:9} Lit. \fbib{into the nations}} \\
\poemll    There is no instruction,\fnote{\fbackref{2:9} Or \fbib{Law}; or \fbib{The priests do not give their guidance}} \\
\poeml and the prophets receive \\
\poemll    no vision from the \divine{Lord}. \\
\poeml \v{10}The leaders of cherished\fnote{\fbackref{2:10} Lit. \fbib{of the daughter of}} Zion \\
\poemll    sit silently on the ground; \\
\poeml they throw dust on their heads \\
\poemll    and dress in mourning clothes. \\
\poeml The young women of Jerusalem \\
\poemll    bow their heads in sorrow.\fnote{\fbackref{2:10} Lit. \fbib{heads to the ground}} \\
\poeml \v{11}My eyes are worn out from crying, \\
\poemll    my insides are churning, \\
\poeml My emotions pour out in grief\fnote{\fbackref{2:11} Lit. \fbib{my liver empties upon the ground}} \\
\poemll    because my people are destroyed--- \\
\poeml Children and infants faint \\
\poemll    in the streets of the city. \\
\poeml \v{12}They ask their mothers, \\
\poemll    ``Is there anything to eat or drink?''\fnote{\fbackref{2:12} Lit. \fbib{any grain and wine}} \\
\poeml They faint in the streets of the city \\
\poemll    like wounded men. \\
\poeml Their life ebbs away \\
\poemll    while they lie on their mother's bosom. \\
\poeml \v{13}What can be said about you? \\
\poemll    To what should you be compared, fair\fnote{\fbackref{2:13} Lit. \fbib{daughter of}} Jerusalem? \\
\poeml To what may I liken you, \\
\poemll    so I may comfort you, fair one\fnote{\fbackref{2:13} Lit. \fbib{virgin daughter}} of Zion? \\
\poeml Indeed, your wound is as deep as the sea--- \\
\poemll    who can heal you? \\
\poeml \v{14}Your prophets look on your behalf; \\
\poemll    they see false and deceptive visions. \\
\poeml They did not expose your sins \\
\poemll    in order to restore what had been captured.\fnote{\fbackref{2:14} Lit. \fbib{restore your captivity}} \\
\poeml Instead, they crafted oracles for you \\
\poemll    that are false and misleading. \\
\poeml \v{15}Everyone who passes by on the road \\
\poemll    shake their fists\fnote{\fbackref{2:15} Or \fbib{road clap their hands}} at you. \\
\poeml They hiss and shake their heads \\
\poemll    at cherished\fnote{\fbackref{2:15} Lit. \fbib{at the daughter}} Jerusalem: \\
\poeml ``Is this the city men used to call `The Perfection of Beauty,' \\
\poemll    and `The Joy of the Entire Earth'\,''? \\
\poeml \v{16}All of your enemies \\
\poemll    insult you with gaping mouths. \\
\poeml They hiss and grind their teeth while saying, \\
\poemll    ``We have devoured her completely. \\
\poeml Yes, this is the day that we anticipated! \\
\poemll    We found it at last;\fnote{\fbackref{2:16} The Heb. lacks \fbib{at last}} we have seen it!'' \\
\poeml \v{17}The \divine{Lord} did what he planned. \\
\poemll    He carried out his threat. \\
\poeml Just as he commanded long ago, \\
\poemll    he has torn down without pity; \\
\poeml He let the enemy boast about you \\
\poemll    and has exalted the power\fnote{\fbackref{2:17} Lit. \fbib{horn}} of your enemies. \\
\poeml \v{18}Cry out from your heart to the Lord, \\
\poemll    wall of fair\fnote{\fbackref{2:18} Lit. \fbib{of the daughter of}} Zion! \\
\poeml Let your tears run down like a river \\
\poemll    day and night. \\
\poeml Allow yourself no rest, \\
\poemll    and don't stop crying. \\
\poeml \v{19}Get up and cry aloud in the night, \\
\poemll    at the beginning of every hour.\fnote{\fbackref{2:19} Lit. \fbib{of the night watches}} \\
\poeml Pour out your heart like water \\
\poemll    in the presence of the Lord! \\
\poeml Lift up your hands toward him \\
\poemll    for the lives of your children, \\
\poeml who are fainting away \\
\poemll    at every street corner. \\
\poeml \v{20}Look, \divine{Lord}, and take note: \\
\poemll    To whom have you done this? \\
\poeml Should women eat their offspring, \\
\poemll    the children they have cuddled? \\
\poeml Should priests and prophets be slain \\
\poemll    in the sanctuary of the Lord? \\
\poeml \v{21}Young men and the aged \\
\poemll    lie on the ground in the streets; \\
\poeml my young women and young men \\
\poemll    have fallen by the sword. \\
\poeml You killed them in your anger, \\
\poemll    slaughtering them without pity. \\
\poeml \v{22}You have invited those who terrorize me to come around, \\
\poemll    as if today were a festival. \\
\poeml No one has escaped or survived \\
\poemll    the time of the \divine{Lord}'s anger. \\
\poeml My enemy has finished off \\
\poemll    those whom I cuddled and raised.
\end{poetry}
\labelchapt{3}
\passage{The \divine{Lord}'s Purposes for Affliction}

\begin{poetry}
\poeml \chapt{3}
\v{1}I am a man familiar with affliction--- \\
\poemll    under the rod of God's\fnote{\fbackref{3:1} Lit. \fbib{his}} anger. \\
\poeml \v{2}He has led me---brought me \\
\poemll    into darkness, not into light. \\
\poeml \v{3}He truly turned his hand against me, \\
\poemll    again and again, all day long. \\
\poeml \v{4}He made my flesh and skin prematurely old; \\
\poemll    he broke my bones. \\
\poeml \v{5}He laid siege against me, \\
\poemll    surrounding me with bitterness and suffering. \\
\poeml \v{6}He has forced me to live in darkness, \\
\poemll    like those who are long dead. \\
\poeml \v{7}He has walled me in so I cannot escape; \\
\poemll    he placed heavy chains on me. \\
\poeml \v{8}Indeed, when I cry out, calling for help, \\
\poemll    he shuts out my prayer. \\
\poeml \v{9}He impeded my way with blocks of stone, \\
\poemll    making my paths uneven. \\
\poeml \v{10}He is like\fnote{\fbackref{3:10} The Heb. lacks \fbib{like}} a bear that lies in wait for me, \\
\poemll    a lion in hiding. \\
\poeml \v{11}He forced me off my path, \\
\poemll    tearing me to pieces and making me desolate. \\
\poeml \v{12}He bent his bow, \\
\poemll    aiming at me with his arrow. \\
\poeml \v{13}He caused his war arrows\fnote{\fbackref{3:13} Lit. \fbib{caused the sons of his arrows}} \\
\poemll    to pierce my vital organs. \\
\poeml \v{14}I have become a laughingstock to all my people, \\
\poemll    the object of their taunts throughout the day. \\
\poeml \v{15}He has filled me with bitterness, \\
\poemll    making me drink wormwood. \\
\poeml \v{16}He broke my teeth on gravel, \\
\poemll    covering me with dust. \\
\poeml \v{17}You have removed peace from my life; \\
\poemll    I have forgotten what prosperity is.\fnote{\fbackref{3:17} Lit. \fbib{forgotten prosperity}} \\
\poeml \v{18}So I say, ``My strength is gone \\
\poemll    as is my hope in the \divine{Lord}.'' \\
\poeml \v{19}Remember my affliction and homelessness--- \\
\poemll    wormwood and gall! \\
\poeml \v{20}My mind keeps reflecting on it, \\
\poemll    and I become depressed.\fnote{\fbackref{3:20} Lit. \fbib{and sinks within me}} \\
\poeml \v{21}This is what comes to mind, \\
\poemll    and therefore I have hope: \\
\poeml \v{22}Because of the \divine{Lord}'s gracious love we are not consumed, \\
\poemll    since his compassions never end. \\
\poeml \v{23}They are new every morning--- \\
\poemll    great is your faithfulness! \\
\poeml \v{24}``The \divine{Lord} is all I have,''\fnote{\fbackref{3:24} Lit. \fbib{is my portion}} says my soul, \\
\poemll    ``Therefore I will trust in him.'' \\
\poeml \v{25}The \divine{Lord} is good to those who wait for him, \\
\poemll    to the person who searches for him. \\
\poeml \v{26}It is good to hope and wait patiently \\
\poemll    for the \divine{Lord}'s salvation. \\
\poeml \v{27}It is good when a young man carries the yoke \\
\poemll    of discipline\fnote{\fbackref{3:27} The Heb. lacks \fbib{of discipline}} in his youth. \\
\poeml \v{28}He is to sit apart and remain silent, \\
\poemll    because the \divine{Lord}\fnote{\fbackref{3:28} Lit. \fbib{because he}} has laid it upon him. \\
\poeml \v{29}Let him fall face down in the dust, \\
\poemll    so there may yet be hope. \\
\poeml \v{30}He will endure being slapped in the face, \\
\poemll    bringing him public disgrace. \\
\poeml \v{31}Indeed, the Lord will not always \\
\poemll    reject us\fnote{\fbackref{3:31} The Heb. lacks \fbib{us}}--- \\
\poeml \v{32}though he causes grief, \\
\poemll    his compassion abounds according to his gracious love. \\
\poeml \v{33}For he does not deliberately hurt \\
\poemll    or grieve human beings. \\
\poeml \v{34}When any of the prisoners of the earth \\
\poemll    are crushed underfoot, \\
\poeml \v{35}when a person's rights are perverted \\
\poemll    in defiance of the Most High. \\
\poeml \v{36}When a man is thwarted in his appeal, \\
\poemll    does the Lord condone\fnote{\fbackref{3:36} Lit. \fbib{see}} it? \\
\poeml \v{37}Who can command, and it happens, \\
\poemll    without the Lord having ordered it? \\
\poeml \v{38}Do not both good and evil things proceed \\
\poemll    from the mouth of the Most High? \\
\poeml \v{39}Why should anyone living complain, \\
\poemll    any mortal, about being punished for sin? \\
\poeml \v{40}Let us examine our lifestyles, \\
\poemll    putting them to the test, \\
\poemlll       and turn back to the \divine{Lord}. \\
\poeml \v{41}Let us lift up our hearts \\
\poemll    and our hands \\
\poemlll       to God in heaven. \\
\poeml \v{42}As for us, we have sinned and rebelled; \\
\poemll    but you have not pardoned us.\fnote{\fbackref{3:42} The Heb. lacks \fbib{us}} \\
\poeml \v{43}Clothing yourself with anger, you pursued us. \\
\poemll    You killed without pity, \\
\poeml \v{44}You covered yourself with a cloud \\
\poemll    that prayer cannot pierce. \\
\poeml \v{45}You have reduced us to scum and garbage \\
\poemll    among the nations. \\
\poeml \v{46}All our enemies \\
\poemll    jeer at us with gaping mouths. \\
\poeml \v{47}Panic and pitfalls beset us, \\
\poemll    along with devastation and ruin. \\
\poeml \v{48}My eyes run with rivers of tears \\
\poemll    over the destruction of my cherished\fnote{\fbackref{3:48} Lit. \fbib{of the daughter of}} people. \\
\poeml \v{49}My tears pour\fnote{\fbackref{3:49} Lit. \fbib{My eye pours}} down ceaselessly; \\
\poemll    I am far from relief \\
\poeml \v{50}until the \divine{Lord} bends down \\
\poemll    to see from heaven. \\
\poeml \v{51}What I see\fnote{\fbackref{3:51} Lit. \fbib{My eye}} grieves my soul \\
\poemll    because of all the young women\fnote{\fbackref{3:51} Lit. \fbib{the daughters}} of my city. \\
\poeml \v{52}My enemies hunted me like a bird, \\
\poemll    viciously and without justification. \\
\poeml \v{53}They dumped me alive into a pit, \\
\poemll    sealing me in with stone.\fnote{\fbackref{3:53} Lit. \fbib{pit, casting a stone at me}} \\
\poeml \v{54}Water closed over my head, \\
\poemll    and I said, ``I'm a dead man.''\fnote{\fbackref{3:54} Lit. ``\fbib{I'm cut off.''}} \\
\poeml \v{55}I called on your name, \divine{Lord}, \\
\poemll    from the depths of the Pit,\fnote{\fbackref{3:55} I.e. the place of punishment in the afterlife} \\
\poeml \v{56}You heard my voice--- \\
\poemll    don't close your ear to my sighs and cries.\fnote{\fbackref{3:56} Lit. \fbib{my relief, to my cry}} \\
\poeml \v{57}You drew near when I called out to you. \\
\poemll    You said, ``Stop being afraid'' \\
\poeml \v{58}Lord, you have defended my cause; \\
\poemll    you have redeemed my life. \\
\poeml \v{59}\divine{Lord}, you observed how I have been wronged; \\
\poemll    now make your ruling in my case. \\
\poeml \v{60}You examined their plans for vengeance, \\
\poemll    all of their plots against me. \\
\poeml \v{61}\divine{Lord}, you listened to their insults--- \\
\poemll    all their plots against me, \\
\poeml \v{62}the whisperings of my opponents, \\
\poemll    their scheming against me all day long. \\
\poeml \v{63}Watch! Whether they sit down or stand up, \\
\poemll    they mock me with their songs. \\
\poeml \v{64}Pay them back, \divine{Lord}, \\
\poemll    according to their actions. \\
\poeml \v{65}Give them an anguished heart; \\
\poemll    may your curse be upon them! \\
\poeml \v{66}Pursue them in your anger \\
\poemll    and destroy them from under the \divine{Lord}'s heaven.
\end{poetry}
\labelchapt{4}
\passage{Zion's Punishment}

\begin{poetry}
\poeml \chapt{4}
\v{1}How tarnished the gold has become, \\
\poemll    the finest gold debased! \\
\poeml Sacred stones\fnote{\fbackref{4:1} Or \fbib{gems}} have been scattered \\
\poemll    at every street corner. \\
\poeml \v{2}Though the precious people of Zion \\
\poemll    were like fine gold, \\
\poeml how they are valued like clay vessels, \\
\poemll    the handiwork of a potter! \\
\poeml \v{3}Even wild animals nurse, \\
\poemll    suckling their young; \\
\poeml but the women of my people are cruel, \\
\poemll    like ostriches in the wilderness. \\
\poeml \v{4}The nursing child's tongue \\
\poemll    cleaves to its palate from thirst. \\
\poeml Young children beg for bread, \\
\poemll    but no one gives them any. \\
\poeml \v{5}Those who enjoyed delicacies \\
\poemll    lie desolate in the streets. \\
\poeml Those who were reared wearing purple \\
\poemll    scavenge in piles of trash. \\
\poeml \v{6}The guilt of my cherished people surpasses the sin of Sodom, \\
\poemll    which was overthrown in a moment, \\
\poemlll       without a hand to help her. \\
\poeml \v{7}Her princes\fnote{\fbackref{4:7} Or \fbib{Nazirites}} were purer than snow, \\
\poemll    whiter than milk. \\
\poeml Their bodies were more ruddy\fnote{\fbackref{4:7} I.e. reddish brown skin color} than rubies, \\
\poemll    their beards like the color of precious stones. \\
\poeml \v{8}Now their faces are blacker than coal; \\
\poemll    they are unrecognized in the streets. \\
\poeml Their skin clings to their bones; \\
\poemll    it has become dry like a stick. \\
\poeml \v{9}Those who die by the sword are better off \\
\poemll    than those who die from starvation, \\
\poeml who slowly waste away like those pierced through \\
\poemll    for lack of food from the fields. \\
\poeml \v{10}With their own hands, compassionate women \\
\poemll    boil their own children--- \\
\poeml they become their food--- \\
\poemll    when my beloved people were\fnote{\fbackref{4:10} Lit. \fbib{when the daughter of my people was}} destroyed. \\
\poeml \v{11}The \divine{Lord} has exhausted his wrath, \\
\poemll    pouring out his fierce anger. \\
\poeml He kindled a fire in Zion, \\
\poemll    consuming its foundations. \\
\poeml \v{12}None of the kings of the earth would have believed, \\
\poemll    nor the world's inhabitants, \\
\poeml that the adversary and the enemy \\
\poemll    could have breached the gates of Jerusalem. \\
\poeml \v{13}Due to the sins committed by her prophets, \\
\poemll    and the iniquities of her priests \\
\poeml who shed in her midst, \\
\poemll    the blood of the righteous, \\
\poeml \v{14}people stagger around in the streets like the blind, \\
\poemll    defiled by blood \\
\poeml unclean so that no one is able \\
\poemll    to touch their clothing. \\
\poeml \v{15}``Go away! Unclean!'' \\
\poemll    they shouted at them. \\
\poemlll       ``Go away! Go away! Don't touch!'' \\
\poeml When they fled away and wandered, \\
\poemll    those among the nations decreed, \\
\poemlll       ``They cannot live here!'' \\
\poeml \v{16}The \divine{Lord} himself separated them; \\
\poemll    he will do nothing more for them. \\
\poeml They did not respect their own priests; \\
\poemll    they did not honor their elders. \\
\poeml \v{17}Our eyes failed, \\
\poemll    searching in vain for hope; \\
\poeml we kept watching and looking \\
\poemll    for a nation that would not help. \\
\poeml \v{18}Our steps were closely stalked, \\
\poemll    so we couldn't travel on our own streets. \\
\poeml Our end is near, \\
\poemll    our days are over; \\
\poemlll       indeed, our end has come. \\
\poeml \v{19}Our pursuers were swifter \\
\poemll    than soaring eagles;\fnote{\fbackref{4:19} Lit. \fbib{than eagles of the heavens}} \\
\poeml they pursued us over the mountains, \\
\poemll    lying in wait for us in the wilderness. \\
\poeml \v{20}The \divine{Lord}'s anointed, \\
\poemll    the breath of our life, \\
\poemlll       was captured in their pits. \\
\poeml About him we had said, \\
\poemll    ``Under his protection we will survive \\
\poemlll       among the nations.'' \\
\poeml \v{21}Celebrate and rejoice, you women\fnote{\fbackref{4:21} Lit. \fbib{daughter}} of Edom, \\
\poemll    who live in the land of Uz. \\
\poeml But to you the cup also will pass--- \\
\poemll    you will become drunk and stripped naked. \\
\poeml \v{22}The punishment for your sin is complete, you women\fnote{\fbackref{4:22} Lit. \fbib{daughter}} of Zion, \\
\poemll    and God\fnote{\fbackref{4:22} Lit. \fbib{he}} will no longer exile you. \\
\poeml He will punish your iniquity, you women\fnote{\fbackref{4:22} Lit. \fbib{daughter}} of Edom, \\
\poemll    and he will expose your sins.
\end{poetry}
\labelchapt{5}
\passage{A Prayer for Deliverance}

\begin{poetry}
\poeml \chapt{5}
\v{1}\divine{Lord}, remember what has happened to us. \\
\poemll    Pay attention, and look at our shame! \\
\poeml \v{2}Our inheritance has\fnote{\fbackref{5:2} Or \fbib{possessions have}} been turned over to strangers, \\
\poemll    and our homes to foreigners. \\
\poeml \v{3}We are now orphans---without fathers--- \\
\poemll    and our mothers are like widows. \\
\poeml \v{4}We pay to drink our own water, \\
\poemll    and our own wood is sold to us at high price. \\
\poeml \v{5}Our pursuers breathe down\fnote{\fbackref{5:5} Lit. \fbib{pursuers are on}} our necks; \\
\poemll    we are weary, but there is no rest for us. \\
\poeml \v{6}We made a deal with the Egyptians and the Assyrians \\
\poemll    for the price of food.\fnote{\fbackref{5:6} Lit. \fbib{Assyrians, being satisfied with bread}} \\
\poeml \v{7}Our ancestors sinned and no longer exist \\
\poemll    yet we continue to bear the consequences of their sin. \\
\poeml \v{8}Slaves rule over us, \\
\poemll    and no one delivers us from their control.\fnote{\fbackref{5:8} Lit. \fbib{hand}} \\
\poeml \v{9}We risk our lives to obtain our food, \\
\poemll    facing death\fnote{\fbackref{5:9} Lit. \fbib{facing the sword}} in the desert. \\
\poeml \v{10}Our skin blisters\fnote{\fbackref{5:10} Lit. \fbib{blackens}} as from an oven, \\
\poemll    due to ravaging blasts of the famine. \\
\poeml \v{11}They have raped women in Zion, \\
\poemll    young women\fnote{\fbackref{5:11} Lit. \fbib{the virgins}} in the towns of Judah. \\
\poeml \v{12}Princes they have hung by their hands; \\
\poemll    elders\fnote{\fbackref{5:12} Lit. \fbib{the faces of the elders}} they have disrespected. \\
\poeml \v{13}Our\fnote{\fbackref{5:13} The Heb. lacks \fbib{Our}} young men must grind grain with a millstone; \\
\poemll    our\fnote{\fbackref{5:13} The Heb. lacks \fbib{our}} youths stumble under the weight of wood. \\
\poeml \v{14}Our\fnote{\fbackref{5:14} The Heb. lacks \fbib{Our}} elders have ceased ruling\fnote{\fbackref{5:14} The Heb. lacks \fbib{ruling}} at the gate; \\
\poemll    our\fnote{\fbackref{5:14} The Heb. lacks \fbib{our}} young men have abandoned\fnote{\fbackref{5:14} The Heb. lacks \fbib{have abandoned}} their music. \\
\poeml \v{15}The joy of our hearts has ceased, \\
\poemll    and our dancing has turned into dirges. \\
\poeml \v{16}The crown has fallen from our head--- \\
\poemll    woe to us, because we have sinned! \\
\poeml \v{17}This is why our hearts faint, \\
\poemll    and why our eyes grow dim: \\
\poeml \v{18}Because Mount Zion is desolate; \\
\poemll    foxes roam around it. \\
\poeml \v{19}You, \divine{Lord}, are forever--- \\
\poemll    your throne endures from generation to generation. \\
\poeml \v{20}So why have you completely forgotten us, \\
\poemll    forsaking us for so long? \\
\poeml \v{21}Restore us to yourself, \divine{Lord}, \\
\poemll    so that we may return. \\
\poeml Renew our days as before, \\
\poeml \v{22}unless you have utterly rejected us \\
\poemlll       and are angry with us without limit.\end{poetry}
